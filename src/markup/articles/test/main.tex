% ---
% title: Test Post
% snippet: A test post.
% publishedAt: 3 Dec 2024
% ---

%preamble
\documentclass[leqno]{article}

% \usepackage{CormorantGaramond}
\usepackage{csquotes} % smart quotes

% american mathematical society
\usepackage{amsmath}
\usepackage{amssymb}
\usepackage{amsthm}
\usepackage{mathtools} % extra symbols

\begin{document}
	\begin{align}
		\label{prop:experience}\forall s \forall o (Xso & \rightarrow Oo \land Ss)    \\
		\label{prop:object-dependence}\exists o (Oo)    & \rightarrow \exists s (Xso)
	\end{align}
	\begin{align}
		\label{test}\textnormal{test}  \\
		\label{test2}\textnormal{test}
	\end{align}
	\eqref{prop:experience} expresses the form of a subjective experience $X$. \eqref{prop:object-dependence}
	expresses a limit on the existence of objects, namely that the existence of an
	experience-object entails the existence of a subjective experience.
	Considering \eqref{prop:experience} and \eqref{prop:object-dependence}
	together, we can formalize this non-dangling property in a lemma:%

	(Not this tho)
	\[
		\int_{-\infty}^{\infty}\hat{f}\lparen{}x \rparen{}i\,e^{2\pi{} i\xi{} x}d\xi{}
	\]
	\[
		E=mc^{2}
	\]

	Here's a video: \video{640}{360}{/static/media/hot-reload.mp4}
\end{document}