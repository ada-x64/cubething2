%%%%%%%%%%%%%%%%%%%%%%%%%%%%%%% cubething.dev %%%%%%%%%%%%%%%%%%%%%%%%%%%%%%%%%

\documentclass[leqno, 12pt]{turabian-researchpaper}
\usepackage{../../cubething}
\addbibresource{../../library.bib}

%opening
\title{Logic and Metaphysics in Early 20th Century}
\author{Phoenix Ada Rose Mandala}

\begin{document}
	\maketitle

	The traditional interpretation of Carnap's 1932 paper, \enquote{The Elimination of Metaphysics Through Logical Analysis of Language},
	\nocite{carnap1966} is that is a poor reading of Heidegger. \nocite{heidegger2008}
	However, recent works (\cite{dambock2022}, \cite{stone2017}, and others) have come
	to reject this reading. Although the paper itself fails on several fronts, it is
	indicative of the overall character of Carnap's project, and there is good
	reason to suppose that Carnap would read Heidegger seriously. In fact, the two
	authors share many of the same predispositions and assumptions. In this paper,
	I wish to explore these undercurrents, and to expose the origin of the difference
	between these giants of \nth{20} century philosophy. First I will go over the shared
	philosophical backgrounds, from the dominant Neo-Kantian philosophy and the
	influence of Husserl, to the political influences which fuel their projects.
	Secondly, I will take a deeper dive into how their philosophical projects were
	fleshed out, focusing particularly on their conceptions of object, science,
	language, and the role of logic in philosophy. Finally, I will take a look at the
	famous debate on the nature of \enquote{metaphysics,} and hopefully draw some
	insightful conclusions and applications to contemporary approaches to
	philosophy.

	\subsection{The Origins}

	Heidegger and Carnap share a cultural background which informs their philosophical
	insights and methodologies. This background is Weberian. Both thinkers wish to
	answer the question: What do we do about the incommensurability of facts and
	meaning?

	Both thinkers are reacting to the Neo-Kantianism which was highly prevelant in
	the late \nth{19} and early \nth{20} century. Heidegger's teacher and the renowned
	logician Edmund Husserl had recently published his books \textit{Logical
	Investigations} \nocite{husserl2005} and \textit{Ideas}, which were designed to
	put formal logic on new grounds. These grounds were phenomenological in nature;
	that is, in order to ground logic, Husserl believed we needed a strong understanding
	of our intuitive capacity to reason. In Kantian terms, Husserl was responding to
	the sentiment that the intuitions of reason espoused in the first critique were
	in need of a grounding. Husserl's solution was to create an idealist structure
	of phenomena: abstract, absolute ideas of experience. These were to structure and
	govern our rational thinking in all realms of life.

	Carnap, as a founding member of logical positivism and a follower of Russell
	and Wittgenstein, was keen to find a logical structure by which to understand the
	world. He did so in his book, \textit{The Logical Structure of the World}. His
	project was constructivist. He would proceed from the atomic structures of our
	reason - the phenomenological basis, akin to Husserl's phenomenology, though
	far simpler. Instead of proposing a realm of abstract entities, Carnap would
	propose a single phenomenal structure, that of [FIXME]. His aim is to show the
	unified basis of all the sciences, that is, to show that there is only one
	proper science, and that the separation of the sciences into epistemologically
	and ontologically distinct subfields is a mistake. Additionally, his emphasis is
	on the prior ontological status of structure.

	\subsection{The Projects}

	First, some definitions. I will define the concepts of object, science,
	language, and logic as both thinker see it. We will need these definitions to
	work out just what they mean when they speak of metaphysics.

	Heidegger and Carnap share the same conception of an object. In Carnap:
	\blockquote[{\cite[5]{carnap2003}}]{The word \enquote{object} is here always used in its widest sense, namely, for anything about which a statement can be made. Thus, among objects we count not only things, but also properties and classes, relations in extension and intentions, states and events, what is actual as well as what is not.}
	And in Heidegger: \blockquote[{\cite[5]{heidegger2008b}}]{Whenever one cognizes anything or makes an assertion, whenever one comports oneself towards entities, even towards oneself, some use is made of \enquote{Being}; and this expression is held to be intelligible \enquote{without further ado}, just as everyone understands \enquote{The sky \emph{is} blue}, \enquote{I \emph{am} merry,} and the like.}
	So in both thinkers the concept of "being" or "object" is applied to both substantives
	(\enquote{what is actual;} \enquote{whenever one comports oneself towards entities}),
	and the objects of predication (\enquote{properties and classes, [etc.]}, \enquote{whenever one cognizes anything or makes an assertion}).\footnote{I
	take this reading from \enquote{Heidegger and the grammar of being} in \cite[chap.
	15]{priest2002}}

	Both authors wish to understand what \emph{is}, in the widest sense. Both authors
	have an interest in science and logic, though they take differing stances on
	its priority, and both authors have an interest in understanding the way we speak
	about beings. Both Heidegger and Carnap believe in a unified science. For Carnap,
	this takes the form of a constructional system. His emphasis is on the structure
	of language and its relation to things in the world, later turning to
	semantics to understand how the formal syntax we use relates us to the things
	we observe. Carnap's project is \blockquote[{\cite[7]{carnap2003}}]{an attempt \emph{to apply the theory of relations to the task of analyzing reality.}}
	Although he uses this phrase to characterize the project of the \textit{Aufbau},
	there is no reason to suppose his general project changed over the course of his
	career.

	\subsection{The Conflict}

	For Heidegger, \enquote{metaphysics} is characterized by a misuse of language,
	that is, it is the linguistic fallacy of speaking of Being as \emph{a} being.
	Where we need to speak of the characteristic essence of beings as a whole, we are
	instead speaking of essential beings. For Carnap, it's quite similar.
	Metaphysics is the misuse of language to speak of things which cannot be
	structurally analyzed, that is, properly thought.

	Carnap characterizes Heidegger's misuse of language in \enquote{The Elimination}
	as an attempt to substantiate something which cannot be substantiated. That is,
	Heidegger attempts to speak of \enquote{the Nothing} as if it is in fact a
	something, a substantial thing. However, this clearly is nonsense under the guise
	of first-order predicate logic. There is no way to adequately capture \enquote{the Nothing nothings}
	with the existential quantifier. Carnap outlines a transition from ordinary
	language to formal language. He characterizes the transition first by creating
	an object, which we call \enquote{nothing} ($no$). We then construct various
	predicates about \enquote{the nothing} by instantiating them with the pseudo-object.
	However, Carnap argues that what we mean by \enquote{nothing} cannot be
	instantiated, as it is represented by a quantifier ($\neg\exists x.Px$ for
	some property $P$). So he argues that we ought to dismiss Heidegger's talk as
	nonsense.

	It is fairly apparent to anyone with an understanding of higher-level logics
	that, in fact, we \emph{can} predicate over quantifiers. We could write
	\enquote{the Nothing nothings} as something like the following:
	$N(\neg\exists x . Px)$, and then express properties of this nothing-ing predicate
	as appropriate. So even on his own terms, given the advances in logic since the
	article was written, Carnap's argument fails. He acknowledges as much in later
	writings (SOURCE), but still maintains that his interpretation of Heidegger is
	correct. How does he claim this?

	Even if we are to predicate over nothing, this form is highly unwieldy, and does
	little to explain exactly what Heidegger is trying to do by substantiating nothing.

	\subsection{The Conclusion}

	\clearpage
	\printbibliography
\end{document}