%%%%%%%%%%%%%%%%%%%%%%%%%%%%%%% cubething.dev %%%%%%%%%%%%%%%%%%%%%%%%%%%%%%%%%

\documentclass[leqno, 12pt]{turabian-researchpaper}
\usepackage{../../cubething}
\addbibresource{../../library.bib}

%opening
\title{Logic and Metaphysics in Early 20th Century}
\author{Phoenix Ada Rose Mandala}

\begin{document}
	\maketitle

	Heidegger and Carnap share a cultural background which informs their
	philosophical insights and methodologies. This background is Weberian. Both
	thinkers wish to answer the question: What do we do about the incommensurability
	of facts and meaning?

	Both thinkers are reacting to the Neo-Kantianism which was highly prevelant in
	the late 19th and early 20th century in Germany. Heidegger's teacher and the
	renowned logician Edmund Husserl had recently published his books \textit{Logical
	Investigations} and \textit{Ideas}, which were designed to put formal logic on
	new grounds. These grounds were phenomenological in nature; that is, in order
	to ground logic, Husserl believed we needed a strong understanding of our intuitive
	capacity to reason. In Kantian terms, Husserl was responding to the sentiment that
	the intuitions of reason espoused in the first critique were in need of a grounding.
	Husserl's solution was to create an idealist structure of phenomena: abstract,
	absolute ideas of experience. These were to structure and govern our rational thinking
	in all realms of life.

	Carnap, as a founding member of logical positivism and a follower of Russell
	and Wittgenstein, was keen to find a logical structure by which to understand the
	world. He did so in his book, \textit{The Logical Structure of the World}. His
	project was constructivist. He would proceed from the atomic structures of our
	reason - the phenomenological basis, akin to Husserl's phenomenology, though
	far simpler. Instead of proposing a realm of abstract entities, Carnap would
	propose a single phenomenal structure, that of [FIXME]. His aim is to show the
	unified basis of all the sciences, that is, to show that there is only one
	proper science, and that the separation of the sciences into epistemologically
	and ontologically distinct subfields is a mistake. Additionally, his emphasis is
	on the prior ontological status of structure: \textquote[{\cite[10]{carnap2003}}]{...the object and its concept are one and the same. This identification does not amount to a reification of the concept, but, on the contrary, is a \enquote{functionalization} of the object.}

	\section{foobar}

	\clearpage
	\printbibliography
\end{document}