%%%%%%%%%%%%%%%%%%%%%%%%%%%%%%% cubething.dev %%%%%%%%%%%%%%%%%%%%%%%%%%%%%%%%%

\documentclass[leqno, 12pt]{turabian-researchpaper}
\usepackage{../../cubething}
\addbibresource{../../library.bib}

\renewcommand*{\bibfont}{\small}

%opening
\title{Logic and Metaphysics in the Early \nth{20} Century}
\author{Phoenix Ada Rose Mandala}

\begin{document}
	\maketitle

	The traditional interpretation of Carnap's 1932 paper, \enquote{The Elimination of Metaphysics Through Logical Analysis of Language,}
	\nocite{carnap1966}\footnote{I will refer to this paper as the \textit{Uberwindung}.}
	is that is a poor reading of Heidegger. However, recent works have come to
	reject this reading. Although Carnap's paper itself fails on several fronts,
	it is indicative of the overall character of his project, and there is good reason
	to suppose that Carnap would read Heidegger seriously. In fact, the two authors
	share many of the same predispositions and assumptions. In this paper, I wish to
	explore these undercurrents, and to expose the origin of the difference
	between these giants of \nth{20} century philosophy. First I will go over the
	shared philosophical backgrounds, from the dominant Neo-Kantian philosophy and
	the influence of Husserl, to the political influences which fuel their
	projects. Secondly, I will take a deeper dive into how their philosophical
	projects were fleshed out, focusing particularly on their conceptions of object,
	science, language, and the role of logic in philosophy. Finally, I will take a
	look at the famous debate on the nature of metaphysics, and hopefully draw some
	insightful conclusions and applications to contemporary approaches to philosophy.

	\section{Historical Background, Political Origins}

	Heidegger and Carnap share a cultural background which informs their
	philosophical insights and methodologies. Both thinkers are generally oriented
	by the same question: What do we do about the incommensurability of facts and
	meaning?

	By facts, I intend empirical facts, truths about the world disclosed to us by
	scientific research. By meaning, I intend meaning in life; not semantic meaning
	in the sense of pointing, but in the sense of fulfillment, guidance, and
	valuation.

	In his 1917 lecture, \enquote{Science as a Vocation,} Max Weber laid out the
	limitations of science for guidance in our lives. To Weber, there are only
	three ways in which science can guide our thought. First, it provides a clear
	means of inquiry about things in the world through its empirical method. Second,
	it gives us a theory of causality which we can use to determine with high
	accuracy the results of our actions. Finally, it provides us a means of
	analyzing our basic attitudes with mathematical logic. Although the sciences can
	and should provide us a method of analyzing our beliefs, it cannot then tell
	us which things to value. It is entirely beyond the scope of the sciences to provide
	us reasons to act. He brings these limitations to the fore in order to stop
	the student population's desire for an "academic prophet" who could quell
	their disenchantment. Industrial production and advances in empirical science had
	lead to a crisis of confidence in theology; without theological faith, there was
	no order by which to live. So, the question of the age is as above: What are
	we to do about the incommensurability of facts and meaning?

	\blockquote[{\cite[139]{weber2014}}]{Under these internal presuppositions, what is the meaning of science as a vocation, now after all these former illusions, the 'way to true being,' the 'way to true art,' the 'way to true nature,' the 'way to true God,' the 'way to true happiness,' have been dispelled? Tolstoy has given the simplest answer, with the words: 'Science is meaningless because it gives no answer to our question, the only question important for us: "What shall we do and how shall we live?" That science does not give an answer to this is indisputable. The only question that remains is the sense in which science gives 'no' answer, and whether or not science might yet be of some use to the one who puts the question correctly.}

	\iffalse
	% too sensational
	By the time Heidegger and Carnap would encounter each other at Davos in 1929,
	the world had run amok. Mathematicians had uncovered contradictions in the foundations
	of their field, leading to an explosion in the development of formal logic. Einstein
	had recently published his theory of relativity, turning established physical
	theory on its head. Heidegger's teacher and logician Edmund Husserl had recently
	published his books \textit{Logical Investigations} \nocite{husserl2005} and \textit{Ideas},
	which were designed to put logic on new grounds, and accidentally discovered
	the field of phenomenology. Max Weber had delivered his lecture \enquote{Science as Vocation}
	to the Jena youth movement; experimental psychologists and sociologists were beginning
	to find empirical answers to questions once deemed the stuff of philosophy.
	All of this had lead to a crisis in intellectualism, and Heidegger was determined
	to position himself at the top of the new pecking order.

	% odd transition
	At Davos, Heidegger was set to debate the neokantian scholar Casserier. Neokantian
	scholars, predominant at the time, had been concerned with the foundations of our
	intuitions, which themselves were supposed to ground logic. Husserl's solution
	was to create an idealist structure of phenomena: abstract, absolute ideas of experience.
	But his students, having grown up in an era of revolution in Europe, and
	having been part of various youth movements throughout Germany and Austria, were
	none too keen on the idea of absolute, essential abstracta. At this time,
	materialist dialectic had made its way into the mainstream through the German
	socialist revolution of the 1920s.\footnote{See \autocite{dambock2022}.}

	Weber argues for a disenchanted understanding of the scientific discipline. To
	Weber, there are three distinct ways science can be used. (1) Science provides
	practical means of inquiry through its method. (2) Science is able to accurately
	predict the outcome of actions through causal theory. (3) Science provides us
	means of analyzing our basic attitudes through logic. \fi

	\iffalse This enumeration limits the scope of the sciences. Science does all this
	and nothing further. All valuation, to Weber, is outside the scope of the
	sciences. It can help us to analyze these evaluations by providing us logic and
	material prediction, but it cannot then say that such outcomes are desirable. Because
	of this limitation, value theory could not be done in the laboratory. Instead,
	it must take to the streets and prioritize its facticity. In this way Weber's ethical
	theory has an existentialist dimension; and it is this dimension which binds
	our opposed thinkers. Weber theorized that the modern era saw a shift from religious
	certainty into disenchantment, a move from theological faith and autocratic
	rule toward scientific reason and democratic liberalism. For Carnap, this was
	a sign that metaphysics \blockquote[{\citetitle[xvii]{carnap2003}}]{belongs to the past.}
	To Heidegger, disenchantment signalled a crisis, and an opportunity to rethink
	the whole tradition of philosophical thought. This move was inspired by Gogarten
	and his dialectical theology \autocite{dambock2022}. This movement sought to found
	ethical theory in transcendent experience of God \textit{beyond} scientific
	rationality. \fi

	For Carnap and his Vienna Circle, the limitations of science are themselves the
	limitations of knowledge. Anything which can be said about material things (including
	individuals and their societies) can be stated clearly, in scientific language.
	The rise of sociology as a discipline, with Weber and Durkheim empiricizing questions
	typically reserved for philosophy, alongside theories of linguistic logic by
	Russell and Wittgenstein, provided further fuel for their attitude.
	Metaphysical pronouncements outside the scope of logical analysis -- about abstract
	essences, ethics, or politics -- are considered \enquote{nonsense.} This term is
	not to be considered a complete dismissal, however; the scientific worldview is
	meant to \enquote{serve life}, not the other way around. In the \enquote{manifesto}
	of their group, \enquote{The Scientific Conception of the World} \autocite{hahn1973},
	they explicitly link their worldview with the socialist movements of the early
	\nth{20} century. They acknowledge a widespread disenchantment among Weberian lines,
	connected with the industrial mode of production. The European public \blockquote[{\cite[21-22]{hahn1973}}]{with their socialist attitudes tend to lean towards a down-to-earth empricist view. In previous times \emph{materialism} was the expression of this view; meanwhile, however, modern empiricism has ... taken a strong shape in the \emph{scientific world-conception.}}

	Meanwhile, Heidegger, influenced by Gogarten and his dialectical theology, would
	argue for a radical conservatism which sought definite meaning beyond scientific
	method. In order to preserve \blockquote[]{the particular rigour of the humanities,}
	Heidegger would offer an analysis of our emotive, factical nature, and in so doing
	would offer an analysis of the essential nature of being. If the sciences can
	provide us with the particular facts of their disciplines, and nothing further,
	Heidegger calls on us to examine the \enquote{nothing further} which by its nature
	sets the limitations of what can be known about beings. Similar to Kant, he is
	asking us to examine our intuitions in order to found our scientific knowledge.
	Influenced by Edmund Husserl, he takes on a phenomenological hermeneutic of our
	lived experience in order to found our intuitions, and thus to examine what
	gives life meaning, in addition to examining the basis for our scientific attitude.
	The scientific worldview is, after all, one among many. To Carnap, as to
	Wittgenstein, ethics seems to be matter of preference. If we are to save
	ethics, then, we need to step beyond science. \autocite{stone2017} provides an
	extensive analysis of the general project of overcoming Kantian metaphysics, through
	Husserl, with the intention of saving ethics.

	The general conception of ethics which our authors offer differ dramatically, though
	both are only implied in their philosophy, and made explicit in their lives.

	So we have shown that both Heidegger and Carnap were driven to exclude ethics
	from their projects by a preference for an analysis of factical existence. Both
	thinkers agree with Weber that science is limited in its application to affairs
	of life, but their thoughts differ dramatically in application of and
	justification for this notion. In the next section I will examine Heidegger and
	Carnap's relation to material analysis through this existential lens.

	\section{Existential Material Analysis}

	An important aspect of Heidegger's philosophy is his anti-Cartesian project. In
	both Heidegger and Carnap, there is a focus on lived experience. Descartes'
	philosophical project - which highly influenced Enlightenment thinking -
	begins with radical doubt. Doubting all experience, we are left with only a
	subject, the I of experience. From this radical subjectivity, we build up the external
	world. So, for Descartes, essence precedes existence; metaphysics begins from an
	intellectual basis. Sensory experience is justified through an appeal to our
	mental faculties. In contrast to this, Heidegger and Carnap both reject the
	primacy of essence. In the scientific world-conception, we proceed from our
	senses to make general claims about the world through logical analysis. Essence
	is rejected; only what can be verified empirically is accepted as real. Metaphysics,
	as purely intellectual, is rejected as a comportment, an attitude toward the
	world. As Heidegger saw, and as many critics of the Vienna Circle have pointed
	out, this rejection of the metaphysical does no good to explain why the scientific
	world-conception should itself be accepted; it is a metaphysical claim without
	justification. The positivist would of course respond that there \emph{could}
	be no verifiable justification; while we are free to debate how we ought to live
	and comport ourselves, this goes beyond the scope of knowledge, and to claim otherwise
	is empty pretension. Heidegger, on the other hand, does not proceed from the
	strict demarcations of science, but rather finds his existential subject among
	other subjects in the world; it is the subject's existence among other
	subjects which imbues it with meaning, and which allows for it to comport itself
	toward the world in various ways. Without this being-in-the-world, being-among-others,
	and being-towards-death, there is no metaphysics (for Heidegger, fundamental ontology);
	and without metaphysics, there is no justification for science.

	Husserl, like Descartes, had proceeded from a bracketing of experience, and naturally
	found himself postulating essences. It is useful for us to see how Carnap and
	Heidegger differ in the critiques of Husserl.

	I will define the concepts of object, science, language, and logic as both thinker
	see it. We will need these definitions to work out just what they mean when
	they speak of metaphysics.

	Heidegger and Carnap share the same conception of an object. In Carnap:
	\blockquote[{\cite[5]{carnap2003}}]{The word \enquote{object} is here always used in its widest sense, namely, for anything about which a statement can be made. Thus, among objects we count not only things, but also properties and classes, relations in extension and intentions, states and events, what is actual as well as what is not.}
	And in Heidegger: \blockquote[{\cite[5]{heidegger2008b}}]{Whenever one cognizes anything or makes an assertion, whenever one comports oneself towards entities, even towards oneself, some use is made of \enquote{Being}; and this expression is held to be intelligible \enquote{without further ado}, just as everyone understands \enquote{The sky \emph{is} blue}, \enquote{I \emph{am} merry,} and the like.}
	So in both thinkers the concept of "being" or "object" is applied to both substantives
	(\enquote{what is actual;} \enquote{whenever one comports oneself towards entities}),
	and the objects of predication (\enquote{properties and classes, [etc.]}, \enquote{whenever one cognizes anything or makes an assertion}).\footnote{I
	take this reading from \enquote{Heidegger and the grammar of being} in \cite[chap.
	15]{priest2002}}

	Both authors wish to understand what \emph{is}, in the widest sense. Both authors
	have an interest in science and logic, though they take differing stances on
	its priority, and both authors have an interest in understanding the way we speak
	about beings. Heidegger and Carnap believe in a unified science. For Carnap,
	this takes the form of a constructional system. His emphasis is on the structure
	of language and its relation to things in the world, later turning to
	semantics to understand how the formal syntax we use relates us to the things
	we observe. Carnap's project is \blockquote[{\cite[7]{carnap2003}}]{an attempt \emph{to apply the theory of relations to the task of analyzing reality.}}
	Although he uses this phrase to characterize the project of the \textit{Aufbau},
	there is no reason to suppose his general project changed over the course of his
	career.

	\section{Metaphysics - How to Found the Sciences?}

	For Heidegger, \enquote{metaphysics} is characterized by a misuse of language,
	that is, it is the linguistic fallacy of speaking of Being as \emph{a} being.
	Where we need to speak of the characteristic essence of beings as a whole, we are
	instead speaking of essential beings. For Carnap, it's quite similar.
	Metaphysics is the misuse of language to speak of things which cannot be
	structurally analyzed, that is, properly thought.

	Carnap characterizes Heidegger's misuse of language in \enquote{The Elimination}
	as an attempt to substantiate something which cannot be substantiated. That is,
	Heidegger attempts to speak of \enquote{the Nothing} as if it is in fact a
	something, a substantial thing. However, this clearly is nonsense under the guise
	of first-order predicate logic. There is no way to adequately capture \enquote{the Nothing nothings}
	with the existential quantifier. Carnap outlines a transition from ordinary
	language to formal language. He characterizes the transition first by creating
	an object, which we call \enquote{nothing} ($no$). We then construct various
	predicates about \enquote{the nothing} by instantiating them with the pseudo-object.
	However, Carnap argues that what we mean by \enquote{nothing} cannot be
	instantiated, as it is represented by a quantifier ($\neg\exists x.Px$ for
	some property $P$). So he argues that we ought to dismiss Heidegger's talk as
	nonsense.

	It is fairly apparent to anyone with an understanding of higher-level logics
	that, in fact, we \emph{can} predicate over quantifiers. We could write
	\enquote{the Nothing nothings} as something like the following:
	$N(\neg\exists x . Px)$, and then express properties of this nothing-ing predicate
	as appropriate. So even on his own terms, given the advances in logic since the
	article was written, Carnap's argument fails. He acknowledges as much in later
	writings (SOURCE), but still maintains that his interpretation of Heidegger is
	correct. How does he claim this?

	Even if we are to predicate over nothing, this form is highly unwieldy, and does
	little to explain exactly what Heidegger is trying to do by substantiating nothing.

	\section{What Must Be Left Unsaid, and Saying It Anyways}

	Carnap and Heidegger both emphasized the importance of language to our
	understanding of the world. While Heidegger explicitly used poetic language to
	speak about what lies beyond science, Carnap believed that that which science
	cannot formalize should explicitly be avoided in philosophical discourse. Any extrascientific
	influence is akin to metaphysics: it cannot be formalized, and so should be avoided.
	In fact, it can be actively dangerous. Carnap worried that metaphysical
	language, though its claims are unverifiable, could have \blockquote[]{psychological impacts}
	which could influence by an appeal to authority rather than to reason. This
	was not without good cause: Gogarten and Heidegger would both endorse the Nazi
	party. \autocite{dambock2022}

	The logical positivists preferred to analyze the logical structure of
	scientific language over material analysis. This is directly reflected in the title
	of Carnap's first major work, \textit{The Logical Structure of the World} \nocite{carnap2003}.\footnote{I
	will refer to this book as the \textit{Aufbau}.} In this book he would proceed
	from the atomic structures of our reason to construct an entire empirical
	worldview supposedly capable of recreating the world of science. His aim is to
	show the unified basis of all the sciences, that is, to show that there is only
	one proper science, and that the separation of the sciences into
	epistemologically and ontologically distinct subfields is a mistake. In line with
	the rest of the circle, this was to be done through conceptual reduction and
	reconstruction. At the time of writing, Carnap's focus was to create a single
	constructional system which would underlie the sciences. As his work matured,
	and by the time of his later major work, \textit{The Logical Syntax of
	Language} \nocite{carnap2000}, he would shift his view to accommodate
	flexibility in the choice of the formal system. He calls this the principle of
	tolerance. \blockquote[{\citetitle[{\S 17}]{carnap2000}}]{Everyone is at liberty to build up his own logic, i.e. his own form of language, as he wishes. All that is required of him is that, if he wishes to discuss it, he must state his methods clearly, and give syntactical rules instead of philosophical arguments.}
	Nonetheless, the method of logical reduction and the autopsychological basis of
	scientific theory holds. This basis is directly related to Husserl's through
	the use of the \textit{epoche}, or the bracketing of psychological experiences
	as real or unreal (see \citetitle[{\S 64}]{carnap2003}). Instead of proposing a
	realm of abstract entities alongside the conceptual connections between them,
	Carnap would propose only the logical structure, with the objects of that structure
	taken to be entirely arbitrary.

	Carnap and his circle's strict adherence to their scientific aims, and their
	rejection of metaphysics, is directly influenced by their political goals.
	Carnap writes in the preface to the \textit{Aufbau}, \blockquote[{\cite[xvii-xviii]{carnap2003}}]{our confidence ... stems from ... the belief that [metaphysics and theology] belong to the past. We feel that there is an inner kinship between the attitude on which our philosophical work is founded and the intellectual attitude which presently manifests in entirely different walks of life. ... It is an orientation which demands clarity everywhere, but which realizes that the fabric of life can never quite be comprehended. ... Our work is carried by the faith that this attitude will win the future.}
	The work of the \textit{Aufbau} and the rest of his corpus can only be explained
	by a pragmatic choice, the choice to pursue the method which has so far given
	us the most progress in science, and which promises, according to the value
	theory held by the group, to provide the most equitable and prosperous conditions
	for mankind. His justification for the scientific worldview thus appeals to
	the political and social aims of the leftist movements of the \nth{20} century.

	\section{Conclusion}

	\clearpage
	\printbibliography
\end{document}