%%%%%%%%%%%%%%%%%%%%%%%%%%%%%%% cubething.dev %%%%%%%%%%%%%%%%%%%%%%%%%%%%%%%%%

\documentclass[leqno, 12pt]{turabian-researchpaper}
\usepackage{../../cubething}
\addbibresource{../../library.bib}

\renewcommand*{\bibfont}{\small}

%opening
\title{What Must Be Left Unsaid, and Saying It Anyways: \\ Logic and Metaphysics
in the Early \nth{20} Century}
\author{Phoenix Ada Rose Mandala}

\begin{document}
	\maketitle

	Carnap's 1932 \enquote{Elimination of Metaphysics} \nocite{carnap1966}\footnote{I
	will refer to this paper as the \textit{Uberwindung}.} is typically criticized
	as a poor reading of Heidegger. However, recent works have come to reject this
	reading. Although Carnap's paper itself fails on several fronts, it is indicative
	of the overall character of his project, and there is good reason to suppose
	that Carnap would read Heidegger seriously. In fact, the two authors share
	many of the same predispositions and assumptions. In this paper, I wish to
	explore these undercurrents, and to expose the origin of the difference between
	these giants of \nth{20} century philosophy. First I will go over the shared philosophical
	backgrounds, from the dominant Neo-Kantian philosophy and the influence of
	Husserl, to the political influences which fuel their projects. Secondly, I will
	take a deeper dive into how their philosophical projects were fleshed out,
	focusing particularly on their conceptions of object, science, language, and the
	role of logic in philosophy. Finally, I will take a look at the famous debate on
	the nature of metaphysics, and hopefully draw some insightful conclusions and
	applications to contemporary approaches to philosophy.

	%%%%%%%%%%%%%%%%%%%%%%%%%%%%%%%%%%%%%%%%%%%%%%%%%%%%%%%%%%%%%%%%%%%%%%%%%%%
	% Main resources: Weber, Dambock
	\section{Historical Background, Political Origins}

	Heidegger and Carnap share a cultural background which informs their philosophical
	insights and methodologies. Both thinkers are generally oriented by the same
	question: What do we do about the incommensurability of facts and meaning?

	By facts, I intend empirical facts, truths about the world disclosed to us by scientific
	research. By meaning, I intend meaning in life; not semantic meaning in the
	sense of pointing, but in the sense of fulfillment, guidance, and valuation.

	In his 1917 lecture, \enquote{Science as a Vocation,} Max Weber laid out the limitations
	of science for guidance in our lives. To Weber, there are only three ways in which
	science can guide our thought. First, it provides a clear means of inquiry about
	things in the world through its empirical method. Second, it gives us a theory
	of causality which we can use to predict the results of our actions. Finally, it
	provides us a means of analyzing our basic attitudes with mathematical and logical
	rigor. Although the sciences can and should provide us a method of analyzing
	our beliefs, it cannot then tell us which things to value. It is entirely
	beyond the scope of the sciences to provide axioms for practical philosophy. Weber
	brings these limitations to the fore in order to dispel the student population's
	desire for an "academic prophet" who could quell the disenchantment brought on
	by the decline of the religious worldview. Industrial production and advances
	in empirical science had lead to a crisis of confidence in theology; without theological
	faith, there was no order by which to live. So, the question of the age is as
	above: What are we to do about the incommensurability of (empirical-scientific)
	facts and (theological-existential) meaning?

	\blockquote[{\cite[139]{weber2014}}]{Under these internal presuppositions, what is the meaning of science as a vocation, now after all these former illusions, the 'way to true being,' the 'way to true art,' the 'way to true nature,' the 'way to true God,' the 'way to true happiness,' have been dispelled? Tolstoy has given the simplest answer, with the words: 'Science is meaningless because it gives no answer to our question, the only question important for us: "What shall we do and how shall we live?" That science does not give an answer to this is indisputable. The only question that remains is the sense in which science gives 'no' answer, and whether or not science might yet be of some use to the one who puts the question correctly.}

	For Carnap and his Vienna Circle, the limitations of science are themselves the
	limitations of knowledge. Anything which can be said about material things (including
	individuals and their societies) can be stated clearly, in scientific language.
	The rise of psychology and sociology as disciplines, with figures like Weber,
	Durkheim, James, and Wundt empiricizing questions typically reserved for
	philosophy, alongside theories of linguistic logic by Russell and Wittgenstein,
	provided further fuel for their attitude. Metaphysical pronouncements outside the
	scope of logical analysis -- about abstract essences, ethics, or politics -- were
	considered \enquote{nonsense.} This term is not to be taken as a complete
	dismissal, however; the scientific worldview is meant to \enquote{serve life},
	not the other way around.

	In the \enquote{manifesto} of their group, \enquote{The Scientific Conception of the World}
	\autocite{hahn1973}, they explicitly link their worldview with the socialist movements
	of the early \nth{20} century. They acknowledge a widespread disenchantment among
	Weberian lines, connected with the industrial mode of production. The European
	public \blockquote[{\cite[pp. 21-22]{hahn1973}}]{with their socialist attitudes tend to lean towards a down-to-earth empricist view. In previous times \emph{materialism} was the expression of this view; meanwhile, however, modern empiricism has ... taken a strong shape in the \emph{scientific world-conception}}.

	The Vienna Circle could be seen as the culmination of the liberal Enlightenment
	project, the scientific worldview winding its way from Descartes to Kant to
	Marx. If the decisions we make in life are \enquote{irrational instincts}
	\autocite[{p. 4}]{carnap1929}, then we ought to guide those instincts through as
	rational a process as we can. The scientific worldview was designed to clear
	up pseudo-problems in philosophy in order to give us the space to think as
	clearly as we could about how to live our lives. Metaphysics was deemed a confusion,
	a misuse of language to describe in scientific terms what went beyond the
	scope of the sciences, and thus was irrational nonsense, poetic at best. The misuse
	of language would take center stage in the later Carnap's thought, and would
	drive much of the activity of the early \nth{20} century Analytic tradition.
	We will discuss language later; what matters now is the goal the Vienna Circle
	had in mind when pursuing this project. They sought to replace the dead God
	and his delusions, with a clear, scientific, collectivist attitude toward the world.

	\blockquote[{\cite[xvii-xviii]{carnap2003}}]{We do not deceive ourselves about the fact that movements in metaphysical philosophy and religion which are critical of such an orientation have again become very influential of late. Whence then our confidence that our call for clarity, for a science that is free from metaphysics, will be heard? It stems from the knowledge or, to put it somewhat more carefully, from the belief that these opposing powers belong to the past.}

	Heidegger, meanwhile, aims to destruct this Cartesian project, clearing the space
	for a new world-conception which can provide us an ontological ground on which
	to found both the sciences and practical philosophy. Heidegger's critique of
	Descartes hinges on his conception of being-in-the-world, as opposed to beings
	present-at-hand. He claims that Descartes mischaracterizes the essential
	nature of beings as Nature, that is, as founded on \enquote{an entity within-the-world which is proximally present-at-hand}
	\autocite[H. 95]{heidegger2008a}. As in the \enquote{Question Concerning Technology}
	\autocite{heidegger2008c}, Heidegger claims that Descartes has essentially
	made the world ready-to-hand, thus enclosing it and making all of Nature,
	including man, into \enquote{standing reserve.} This view is somewhat ironic
	considering Heidegger's perceived Nazi sympathies, though reading Heidegger as
	a clear-cut Nazi hardliner is naive. In \enquote{Only a God can Save Us,} Heidegger
	explicitly denies any ties with the Nazi party, considering himself an
	\enquote{unpolitical} person. In fact, he characterizes himself as \enquote{confronting the Nazi party}
	in his Nietzsche lectures, and as the subject of scrutiny under Nazi rule.
	Whether or not he was truly sympathetic to Nazi politics is a matter of debate,
	but in any case he is explicitly opposed to socialism as \enquote{a form of planetary technicity}
	\autocite[p. 206]{heidegger1981}.

	If the sciences can provide us with the facts of their particular disciplines,
	and nothing further, Heidegger calls on us to examine the \enquote{nothing further}
	which by its nature sets the limitations of what can be known about beings. Similar
	to Kant, he is asking us to examine our intuitions in order to found our
	scientific knowledge. Influenced by Edmund Husserl, he takes on a
	phenomenological hermeneutic of our lived experience in order to found our
	intuitions, and thus to examine what gives life meaning, in addition to
	examining the basis for our scientific attitude. The scientific worldview is,
	after all, one among many.

	Heidegger and Carnap are both seated within the Western philosophical tradition,
	and both are attempting to overcome it. Their methods differ quite dramatically,
	however, as do their end goals. Carnap wishes to clear away the cruft of the
	past few millennia in order to see rightly; fundamentally, he believed
	Descartes was on the right track. Heidegger wants to destruct the method and the
	worldview which evolved from Plato through Kant, and especially the scientific
	project began by Descartes. Both do so in order to found a worldview which is capable
	of overcoming the existential conditions which underlie false metaphysical inquiry.
	These conditions, being existential, are immediately tied up in both material
	analysis and practical philosophy, in questions concerning both facts and the meaning
	of life. Nietzsche declared that God is dead, that facts and meaning no longer
	lined up; disenchantment in the early \nth{20} century called for a radical reconstitution
	of the Western worldview, and both authors were determined to find it. In the next
	section we will take a look at how they each went about this destruction.

	%%%%%%%%%%%%%%%%%%%%%%%%%%%%%%%%%%%%%%%%%%%%%%%%%%%%%%%%%%%%%%%%%%%%%%%%%%%
	% Main resource: Stone 2007
	\section{Facts and Values}

	Descartes' philosophical project begins with radical doubt. Putting aside all
	experience in a skeptical \textit{epoch\'e}, we are left with the Cartesian
	ego as the subject of experience. From this radical subjectivity, we build up
	the external world. So, for Descartes, essence precedes existence: ontology
	begins from an intellectual basis. Sensory experience is justified through an appeal
	to our mental faculties. Husserl, like Descartes, proceeded from a bracketing
	of experience, and naturally found himself postulating essences. In contrast
	to this, Heidegger and Carnap both reject the primacy of the mental,
	preferring instead to proceed from the material to the essential. It is useful
	for us to see how Carnap and Heidegger differ in the critiques of Husserl.

	Stone argues that Heidegger and Carnap both wished to save ethics by limiting the
	pretensions of metaphysics. For both, this manifest itself in political action.
	However, both authors operate off of a kind of \enquote{faith} - a faith in
	the future of humanity which they as of yet cannot see or determine. This faith
	is Nietzschean in nature - for Heidegger this is quite explicit, given his
	famous lectures. For Carnap this faith is less obvious. \autocite{sachs2011a}

	In the scientific world-conception, we proceed from our senses to make general
	claims about the world through logical analysis. Essence is rejected; only what
	can be verified empirically is accepted as real. Metaphysics, as purely intellectual,
	is rejected as an unnatural attitude toward the world. As Heidegger saw, and
	as many critics of the Vienna Circle have pointed out, this rejection of the metaphysical
	does no good to explain why the scientific world-conception should itself be accepted;
	it is a metaphysical claim without justification. A positivist might respond that
	there \emph{could} be no verifiable justification; while we are free to debate
	how we ought to live, this goes beyond the scope of knowledge, and to claim otherwise
	is empty pretension. Heidegger, on the other hand, does not proceed from the
	strict demarcations of science, but rather finds his existential subject among
	other subjects in the world; it is the subject's existence among other
	subjects which imbues it with meaning, and which allows for it to relate to the
	world in various ways. Without this being-in-the-world, being-among-others,
	and being-towards-death, there is no metaphysics (for Heidegger, fundamental ontology);
	and without metaphysics, there is no justification for science.

	I will define the concepts of object, science, language, and logic as both thinker
	see it. We will need these definitions to work out just what they mean when
	they speak of metaphysics.

	Heidegger and Carnap share the same conception of an object. In Carnap:
	\blockquote[{\cite[5]{carnap2003}}]{The word \enquote{object} is here always used in its widest sense, namely, for anything about which a statement can be made. Thus, among objects we count not only things, but also properties and classes, relations in extension and intentions, states and events, what is actual as well as what is not.}
	And in Heidegger: \blockquote[{\cite[5]{heidegger2008b}}]{Whenever one cognizes anything or makes an assertion, whenever one comports oneself towards entities, even towards oneself, some use is made of \enquote{Being}; and this expression is held to be intelligible \enquote{without further ado}, just as everyone understands \enquote{The sky \emph{is} blue}, \enquote{I \emph{am} merry,} and the like.}
	So in both thinkers the concept of "being" or "object" is applied to both substantives
	(\enquote{what is actual;} \enquote{whenever one comports oneself towards entities}),
	and the objects of predication (\enquote{properties and classes, [etc.]}, \enquote{whenever one cognizes anything or makes an assertion}).\footnote{I
	take this reading from \enquote{Heidegger and the grammar of being} in \cite[chap.
	15]{priest2002}}

	Both authors wish to understand what \emph{is}, in the widest sense. Both authors
	have an interest in science and logic, though they take differing stances on
	its priority, and both authors have an interest in understanding the way we speak
	about beings. Heidegger and Carnap believe in a unified science. For Carnap,
	this takes the form of a constructional system. His emphasis is on the structure
	of language and its relation to things in the world, later turning to
	semantics to understand how the formal syntax we use relates us to the things
	we observe. Carnap's project is \blockquote[{\cite[7]{carnap2003}}]{an attempt \emph{to apply the theory of relations to the task of analyzing reality.}}
	Although he uses this phrase to characterize the project of the \textit{Aufbau},
	there is no reason to suppose his general project changed over the course of his
	career.

	%%%%%%%%%%%%%%%%%%%%%%%%%%%%%%%%%%%%%%%%%%%%%%%%%%%%%%%%%%%%%%%%%%%%%%%%%%%
	% Main resources: Sachs 2011
	\section{Metaphysics - How to Found the Sciences?}

	For Heidegger, \enquote{metaphysics} is characterized by a misuse of language,
	that is, it is the linguistic fallacy of speaking of Being as \emph{a} being.
	Where we need to speak of the characteristic essence of beings as a whole, we are
	instead speaking of essential beings. For Carnap, it's quite similar.
	Metaphysics is the misuse of language to speak of things which cannot be
	structurally analyzed, that is, properly thought.

	Carnap characterizes Heidegger's misuse of language in \enquote{The Elimination}
	as an attempt to substantiate something which cannot be substantiated. That is,
	Heidegger attempts to speak of \enquote{the Nothing} as if it is in fact a
	something, a substantial thing. However, this clearly is nonsense under the guise
	of first-order predicate logic. There is no way to adequately capture \enquote{the Nothing nothings}
	with the existential quantifier. Carnap outlines a transition from ordinary
	language to formal language. He characterizes the transition first by creating
	an object, which we call \enquote{nothing} ($no$). We then construct various
	predicates about \enquote{the nothing} by instantiating them with the pseudo-object.
	However, Carnap argues that what we mean by \enquote{nothing} cannot be
	instantiated, as it is represented by a quantifier ($\neg\exists x.Px$ for
	some property $P$). So he argues that we ought to dismiss Heidegger's talk as
	nonsense.

	It is fairly apparent to anyone with an understanding of higher-level logics
	that, in fact, we \emph{can} predicate over quantifiers. We could write
	\enquote{the Nothing nothings} as something like the following:
	$N(\neg\exists x . Px)$, and then express properties of this nothing-ing predicate
	as appropriate. So even on his own terms, given the advances in logic since the
	article was written, Carnap's argument fails. He acknowledges as much in later
	writings (SOURCE), but still maintains that his interpretation of Heidegger is
	correct. How does he claim this?

	Even if we are to predicate over nothing, this form is highly unwieldy, and does
	little to explain exactly what Heidegger is trying to do by substantiating nothing.

	%%%%%%%%%%%%%%%%%%%%%%%%%%%%%%%%%%%%%%%%%%%%%%%%%%%%%%%%%%%%%%%%%%%%%%%%%%%
	\section{What Must Be Left Unsaid, and Saying It Anyways}

	Carnap and Heidegger both emphasized the importance of language to our
	understanding of the world. While Heidegger explicitly used poetic language to
	speak about what lies beyond science, Carnap believed that that which science
	cannot formalize should explicitly be avoided in philosophical discourse. Any extrascientific
	influence is akin to metaphysics: it cannot be formalized, and so should be avoided.
	In fact, it can be actively dangerous. Carnap worried that metaphysical
	language, though its claims are unverifiable, could have \blockquote[]{psychological impacts}
	which could influence by an appeal to authority rather than to reason. This
	was not without good cause: Gogarten and Heidegger would both endorse the Nazi
	party. \autocite{dambock2022}

	The logical positivists preferred to analyze the logical structure of
	scientific language over material analysis. This is directly reflected in the title
	of Carnap's first major work, \textit{The Logical Structure of the World} \nocite{carnap2003}.\footnote{I
	will refer to this book as the \textit{Aufbau}.} In this book he would proceed
	from the atomic structures of our reason to construct an entire empirical
	worldview supposedly capable of recreating the world of science. His aim is to
	show the unified basis of all the sciences, that is, to show that there is only
	one proper science, and that the separation of the sciences into
	epistemologically and ontologically distinct subfields is a mistake. In line with
	the rest of the circle, this was to be done through conceptual reduction and
	reconstruction. At the time of writing, Carnap's focus was to create a single
	constructional system which would underlie the sciences. As his work matured,
	and by the time of his later major work, \textit{The Logical Syntax of
	Language} \nocite{carnap2000}, he would shift his view to accommodate
	flexibility in the choice of the formal system. He calls this the principle of
	tolerance. \blockquote[{\citetitle[{\S 17}]{carnap2000}}]{Everyone is at liberty to build up his own logic, i.e. his own form of language, as he wishes. All that is required of him is that, if he wishes to discuss it, he must state his methods clearly, and give syntactical rules instead of philosophical arguments.}
	Nonetheless, the method of logical reduction and the autopsychological basis of
	scientific theory holds. This basis is directly related to Husserl's through
	the use of the \textit{epoche}, or the bracketing of psychological experiences
	as real or unreal (see \citetitle[{\S 64}]{carnap2003}). Instead of proposing a
	realm of abstract entities alongside the conceptual connections between them,
	Carnap would propose only the logical structure, with the objects of that structure
	taken to be entirely arbitrary.

	Carnap and his circle's strict adherence to their scientific aims, and their
	rejection of metaphysics, is directly influenced by their political goals.
	Carnap writes in the preface to the \textit{Aufbau}, \blockquote[{\cite[xvii-xviii]{carnap2003}}]{our confidence ... stems from ... the belief that [metaphysics and theology] belong to the past. We feel that there is an inner kinship between the attitude on which our philosophical work is founded and the intellectual attitude which presently manifests in entirely different walks of life. ... It is an orientation which demands clarity everywhere, but which realizes that the fabric of life can never quite be comprehended. ... Our work is carried by the faith that this attitude will win the future.}
	The work of the \textit{Aufbau} and the rest of his corpus can only be explained
	by a pragmatic choice, the choice to pursue the method which has so far given
	us the most progress in science, and which promises, according to the value
	theory held by the group, to provide the most equitable and prosperous conditions
	for mankind. His justification for the scientific worldview thus appeals to
	the political and social aims of the leftist movements of the \nth{20} century.

	%%%%%%%%%%%%%%%%%%%%%%%%%%%%%%%%%%%%%%%%%%%%%%%%%%%%%%%%%%%%%%%%%%%%%%%%%%%
	\section{Conclusion}

	Carnap and Hiedegger come to opposite conclusions from the same background. Both
	wish to confront the incommensurability of facts and meaning. Carnap does so
	by expunging meaning from scientific discourse, leaving it to the poets to sort
	out. For Carnap, scientific discourse is central. Logic ought to guide us more
	than the irrational whims of desire; it is left to the poets to express these \enquote{attitudes toward life.}
	Although both agree that the sciences are preceeded by the emotions, and that in
	life we are always already experiencing some attitude toward the world, they
	disagree about what this means for human life. Heidegger takes the ontical primacy
	of the emotions as evidence of their philosophical priority. If we are to
	orient ourselves to the world in a meaningful manner, we must think through our
	moods and our emotions; rather than viewing them as a useful but irrational
	orientatating schema, we ought to consider them as precisely indicative of a relation
	to the world. Only through an analysis of this sort can we come to understand
	how we ought to relate to the world. However, the primacy of emotive, orienting
	rationality is not to deny the categorical rationality of logic. The
	difference in our author's positions becomes clear in their stances towards
	language. For Carnap, formal syntax is the proper mode of scientific
	communication, and thus the only way to properly enunciate an idea (where by
	\enquote{idea} we mean the formal, Kantian notion of an idea: that of an empirical
	statement about the world). %This is incorrect.

	Political and geospatial differences fuelled the distance that Carnap and Heidegger
	felt from each other, despite working from similar grounds. If we are able to overcome
	these differences to see the throughline, perhaps we could reinvigorate a more
	linguistic turn in analytic value theory. % Sellars.

	\clearpage
	\printbibliography
\end{document}