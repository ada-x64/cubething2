%%%%%%%%%%%%%%%%%%%%%%%%%%%%%%% cubething.dev %%%%%%%%%%%%%%%%%%%%%%%%%%%%%%%%%

\documentclass[leqno, 12pt]{turabian-researchpaper}
\usepackage{../../cubething}
\addbibresource{../../library.bib}

\renewcommand*{\bibfont}{\small}

%opening
\title{What Must Be Left Unsaid, and Saying It Anyways: \\ Logic and Metaphysics
in the Early \nth{20} Century}
\author{Phoenix Ada Rose Mandala}

\begin{document}
	\maketitle

	Carnap's 1932 \enquote{Elimination of Metaphysics} \nocite{carnap1966} is
	typically criticized as a poor reading of Heidegger. However, recent works
	have come to reject this reading. Although Carnap's paper itself fails on several
	fronts, it is indicative of the overall character of his project, and there is
	good reason to suppose that Carnap would read Heidegger seriously. In fact, the
	two authors share many of the same predispositions and assumptions. In this
	paper, I wish to explore these undercurrents, and to expose the origin of the
	difference between these giants of \nth{20} century philosophy.

	First, I will go over the historical and political movements which fuel their
	projects. The philosophers' positions will then be discussed against the then-dominant
	Neokantian school and the influence of Edmund Husserl's phenomenology. I will then
	discuss the matters on which they agree and find themselves in contrast. The
	discussion will proceed through an analysis of the author's conceptions of
	fundamental epistemology, metaethics, and their relation to language.

	%=========================================================================%
	\section{Historical Background, Political Origins}

	Carnap and Heidegger were young men in 1914, when the world went to war:. The
	horror of industrial warfare and the decline of Western religious practices
	scarred a generation, prompting a radical political split. The classic dilemma
	of is and ought had come to engross the age: What are we to do about the incommensurability
	of (empirical-scientific) facts and (theological-existential) meaning? In the
	midst of the great war, the sociologist Max Weber wrote the following.

	\begin{displayquote}
		[{\cite[139]{weber2014}}] \dots what is the meaning of science as a vocation,
		now after all these former illusions \dots have been dispelled? Tolstoy has given
		the simplest answer, with the words: \enquote{Science is meaningless because it gives %
		no answer to our question, the only question important for us: \enquote{%
		What shall we do and how shall we live?}}
		That science does not give an answer to this is indisputable. The only question
		that remains is the sense in which science gives \enquote{no} answer, and whether
		or not science might yet be of some use to the one who puts the question correctly.
	\end{displayquote}
	This essay aims to clarify why science gives no answer.

	For Carnap and his Vienna Circle, the limitations of science are themselves
	the limitations of knowledge. Anything which can be said about material things
	(including individuals and their societies) can be stated clearly, in
	scientific language. The rise of psychology and sociology as disciplines, with
	figures like Weber, Durkheim, James, and Wundt empiricizing questions typically
	reserved for philosophy, alongside theories of linguistic logic by Russell and
	Wittgenstein, provided further fuel for their attitude. Metaphysical
	pronouncements outside the scope of logical analysis -- about abstract essences
	or universal ethics -- were considered \enquote{nonsense.} This term is not to
	be taken as a complete dismissal, however; the scientific worldview is meant to
	\enquote{serve life}, not the other way around. Ethics, and politics, is to
	proceed from empirical grounds, as a sociopsychological science. Of course, they
	were not without their bias.

	In \enquote{The Scientific Conception of the World,} \nocite{hahn1973} they explicitly
	link their worldview with the socialist movements of the early \nth{20}
	century. They acknowledge a widespread disenchantment among Weberian lines,
	connected with the industrial mode of production. The European public \blockquote[{\cite[21f]{hahn1973}}]{with their socialist attitudes tend to lean towards a down-to-earth empricist view. In previous times \emph{materialism} was the expression of this view; meanwhile, however, modern empiricism has ... taken a strong shape in the \emph{scientific world-conception}}.

	While strict empiricism takes the priority for the positivists, Heidegger
	focuses on the lost sense of meaning. For him, the Western philosophical
	project is extremely fruitful, but fatally flawed. He aims to destruct the
	Western canon, clearing the space for a new world-conception which can provide
	us the ground on which to found a new kind of philosophy.

	If the sciences can provide us with the facts of their particular disciplines,
	and nothing further, Heidegger calls on us to examine the \enquote{nothing further}
	which by its nature sets the limitations of what can be known about beings. Similar
	to Kant, he is asking us to examine our intuitions in order to found our
	scientific knowledge. He takes on a phenomenological hermeneutic of our lived experience
	in order to found our intuitions. In so doing, he engages with the emotions and
	our relationships with other people, in addition to our relationships to mere
	objects. We relate to the world in myriad ways, and Heidegger believes these
	are worth investigating. The scientific worldview is, after all, one among many.

	It is well-known that Heidegger took the rectorate at Freiburg under Nazi reign,
	and delivered several lectures expounding on the \enquote{inner greatness} of the
	Nazi party. Throughout his life, he was taciturn about his involvements in the
	party. In a late interview, published only posthumously, Heidegger claims his
	involvement with the Nazi party was taken out of a sense of obligation, in order
	to keep the university intact, and to secure his position as rector. Regardless
	of the nature of his involvement with fascism, it is clear that Heidegger was
	against communism in any form, especially the technocratic version which the Vienna
	Circle would undoubtedly espouse. He argues that communism and \enquote{Americanism}
	(what I assume to be technocratic free-market capitalism) are forms of global technicity,
	that is, a movement away from viewing human beings in their full nature towards
	the total organization and technologization of human behavior in cybernetics.
	In his language, all beings, even Dasein itself, becomes uprooted from the earth
	and becomes ready-to-hand, no longer capable of being seen in its true essence.
	Although he does not make his political end-goal explicit, it is against this
	technicity, and towards a \textit{Volkish} ideal that Heidegger strives
	\autocite{heidegger1981}.

	Heidegger and Carnap are both seated within the Western philosophical tradition,
	and both are attempting to overcome it. Their methods differ quite dramatically,
	however, as do their end goals. Carnap wishes to clear away the cruft of the
	past few millennia in order to see rightly; fundamentally, he believed the
	Western canon was on the right track. Heidegger, however, wants to destruct the
	method and the worldview which evolved from Plato through Kant. Both do so in
	order to found a worldview which is capable of overcoming the existential
	conditions which underlie false metaphysical inquiry. These conditions are immediately
	tied up in both material analysis and practical philosophy, in questions
	concerning both empirical facts and lived meaning. In the next section we will
	take a look at how they each went about this destruction.

	%=========================================================================%
	\section{Husserl and Neokantianism}

	Heidegger and Carnap were both educated in the Neokantian tradition. In keeping
	with the post-Kantian idealists, the Neokantians rejected the Kantian
	distinction between the realms of pure sensibility and understanding. Over spatio-temporal
	intuition, they preferred a pure logical basis for the object of knowledge. \autocite[28]{friedman2000}
	The tradition was divided between several schools, most notably the Marburg
	school and the Southwest school. While there are many differences between these
	camps, the primary distinction I will focus on is their stance on the
	distinction between logic and values.

	Logic, being a discursive science aimed at proper thinking, was considered
	\emph{normative.} For Rickert, a key player of the Southwest school, this leads
	to the claim that logic belongs to the realm of transcendental value. Pure
	logic is then to be distinguished from psychological knowledge, which proceeds
	by empirical means. This leaves us with the gap between \emph{logic} as value,
	and \emph{psychology} as fact \autocite[34ff]{friedman2000}.

	Cassirer of the Marburg school rejected this distinction. Instead of cleaving
	the mind between transcendental psychology and pure logic, Cassirer claims that
	these gaps are \emph{moments} of cognition. They are not essential features which
	then are synthesized in to a single cognition, but a unitary relation which
	has been artificially analyzed into parts. Value and fact are analogous to the
	universal and the particular, but for Cassirer this is a difference in matter
	of \emph{viewpoint,} between understanding a thing in its \enquote{thinghood}
	(as it relates to universal/transcendental forms), and understanding a thing in
	its \enquote{context} (as a being among beings). This separation between viewpoints
	is \enquote{artificial,} only occurring after the initial unitary relation
	between mind and object. The originary perception finds these two aspects in unison
	\autocite[34ff]{friedman2000}.

	The parallels between Husserl and the Neokantian schools is fairly clear.
	Rickert writes positively of the \textit{Logical Investigation's}\nocite{husserl2005}
	polemic against psychologism, and there are clear parallels between the
	unitary relation of cognition in Cassirer and Husserl's unitary field of
	phenomenological consciousness (see \textit{Ideas} \S38)\nocite{husserl2012}. Akin
	to Rickert, Husserl ends up proposing transcendental phenomenological ideas as
	the basis of his ontology, but akin to Cassirer he proposes that value and fact
	are part of the same conscious process.

	The phenomenon was proposed by Husserl as a way of bridging the gap between the
	conscious being and the thing-in-itself. If the phenomenon is understood in
	Kant's way as the empirical intuition of a thing, then it could be said that Husserl's
	phenomenon is closer to the transcendental intuition itself, the conditions
	for the possibility of experience in the first place. Reality shows itself through
	these conditions and through phenomenal appearances. It is the work of
	phenomenology to transcend the appearances and move \enquote{to the things themselves.}

	Drawing inspiration from this tradition, but moving past it, there are a few
	shared concerns for Carnap and Heidegger: (1) Scientific thought is formal in nature,
	so it cannot capture values; (2) Language is key to thought; and, (3) The manner
	in which we use language corresponds directly to our practical aims.

	%-------------------------------------------------------------------------%
	\section{Rational Construction}

	The scientific worldview lends itself to the notion of a unified science. It is
	Carnap's goal to place science on a unified ground, and in so doing to
	eliminate the confusions which traditional philosophy (\enquote{metaphysics}) has
	perpetuated. The goal of science is to \blockquote[{\citetitle[\S179]{carnap2003}}]{find and order the true statements about the objects of experience}.
	This can be split into two further goals: to create a constructional system
	corresponding to formal logic, and to investigate the relationships between objects
	of non-constructed experience. We then have two foundational aspects of
	construction theory: conventional stipulation of logical syntax and verification
	through empirical investigation.

	Carnap proceeds from unanalyzable phenomenological units which he calls basic experiences:
	these are whole-field conscious experiences which appear epistemically prior
	to the analysis of their constitutive parts. They are precisely the object of phenomenological
	analysis in Husserl's sense (\citetitle[\S64]{carnap2003}).\footnote{Carnap
	believed his system to be compatible with the three major epistemological
	movements: realism, idealism, and phenomenalism (\citetitle[\S177f]{carnap2003}).
	Nonetheless, he chose the Husserlian phenomena as his starting point.} However,
	basic experiences do not have any relation to a subject; the \enquote{for-me-ness}
	is analyzed into it after the fact. Similar to how we can only call integers
	\enquote{integers} in contradistinction to the real numbers once we have constructed
	them, we only understand our basic experiences as autopsychological in
	contrast with the later-constructed physical and heteropsychological objects. (\citetitle[{III.C \textit{passim.}}]{carnap2003}).

	Despite the inability to analyze the experiences themselves, he \enquote{quasi-analyzes}
	them by creating a formal allegory and examining the structures thereof. In
	this way he constructs the world through a purely structural analysis of basic
	experience. Through this formal analysis he claims to transform foundational epistemology
	from a speculative practice into a mathematically rigorous, logistical process.

	% In adopting the phenomenological \textit{epoch\'e} and proceeding from purely
	% formal analysis, Carnap obviates the distinction between the real and the ideal.
	% Ultimately, both \enquote{real} and \enquote{ideal} objects are constructed from
	% the same basic experiences. It is important to note that Carnap is not here
	% discussing ideals as essences. Quite famously, essences are the stuff of
	% metaphysics, which Carnap rejects outright. The difference is that \enquote{ideal}
	% objects are purely autopsychological, preceding physical construction, while \enquote{real}
	% objects are those that rely on physics and its objectivity \autocite[77]{friedman2000}.

	% The general structure here should sound similar to Cassirer. Both construct the
	% objects of experience through formal analysis. However, while Cassirer's construction
	% proceeds indefinitely, Carnap constructs definite, finite ranks within the type
	% hierarchy of his system \autocite[80]{friedman2000}.

	% For Carnap, as for the Marburg school, all objects are generated through a logical
	% (cognitive) faculty through direct experience. \autocite[75]{friedman2000}. However,
	% the difference between what is given (the real) and what is constructed (the ideal)
	% is obviated in Carnap. For him, \enquote{a concept and its object are the same.}
	% This amounts to a \enquote{functionalization} of the object. Carnap flattens the
	% distinction between real and ideal, as both are ultimately constituted objects.
	% The real-ideal distinction amounts to the distinction between the spatio-temporal
	% order and all others. \autocite[78]{friedman2000}

	% By redefining philosophy as logistics -- within the \textit{Aufbau} as the
	% formal logic of the \textit{Principia Mathematica} of Russell and Whitehead,
	% and later in the \textit{Syntax} as \emph{any} formal langauge -- Carnap aimed
	% to remove the metaphysical notion of the synthetic \textit{a priori.} In doing
	% so he would overcome the infinite regress of the Marburg school's object of
	% knowledge, thus rooting out the unknowable thing-in-itself. Clearing up these
	% final metaphysical notions would leave no room for philosophical
	% pseudoproblems which could not be solved by the logical analysis of language.

	In the \textit{Aufbau,} Carnap aims to formulate a \enquote{one true language}
	from which to construct the world. In response to formal difficulties arising from
	the construction of the physical from the autopsychological, he is forced to
	reconsider his stance on logic, eventually realizing that there are many valid
	formal logics. Once we reach the \textit{Syntax,} this stipulatory nature is formalized
	into the principle of tolerance.
	%
	\begin{displayquote}
		[{\citetitle[\S17]{carnap2000}}] \emph{In logic, there are no morals.}
		Everyone is at liberty to build up his own logic, i.e. his own form of
		language, as he wishes. All that is required of him is that, if he wishes to
		discuss it, he must state his methods clearly, and give syntactial rules instead
		of philosophical arguments.
	\end{displayquote}
	%
	Since logic is the ideal evaluative framework, but there are multiple valid logical
	systems, it is up to us to find the ideal logical system to relate to our
	practical aims, whatever they may be.

	Crucially, Carnap distinguishes between two modes of speech, the material and the
	formal. The material mode is defined by a casual transposition of complicated
	syntactical statements of the kind discussed in the \textit{Syntax} into ordinary
	speech. The move from discussion of syntax to discussion of material
	properties often leads to shorter and clearer explanations, but easily lends itself
	to confusion. While Carnap claims that this mode of speech is perfectly
	acceptable, and in fact unavoidable, he warns us that this translation leads
	us to speak of spurious universals - i.e., metaphysical essences. Additionally,
	some material sentences simply cannot be translated into formal syntax. Although
	these statements are useful for expressing ourselves, they cannot be subject
	to claims about truth and falsity, and so have no place in a scientific philosophy.

	\begin{displayquote}
		[{\citetitle[\S81]{carnap2000}}] If a sentence of the material mode of speech
		is given \dots, then the translation into the formal mode of speech \dots
		must always be possible. \emph{Translatability into the formal mode of
		speech constitutes the touchstone for all philosophical sentences}, or more generally,
		for all sentences which do not belong to the language of any one of the
		empirical sciences. \dots Sentences which do not give even a slight indication
		to determine their translation are outside the realm of the language of science
		and therefore incapable of discussion, no matter what depths or heights of
		feeling they may stir.
	\end{displayquote}

	So Carnap has his linguistic prescription: if you wish to be intelligible, you
	must speak in a manner consistent with the formal mode of speech. This relates
	directly to his ethics. Carnap claims in a 1932 lecture that ethical claims
	are in essence prescriptive statements, specifically imperatives. These
	statements are manifestations of an attitude toward life. Imperatives are not translatable
	into formal language, but the material mode of speech allows us to slip up and
	propose spurious ethical essences. \enquote{Do not kill,} for example, is clearly
	an imperative, while \enquote{killing is evil} is an assertion about a universal
	property which does not admit of empirical evidence \autocite[\S1.4]{carnap1996}.
	Carnap's own prescription, then, must be an expression of an attitude toward life,
	and admits of no proper evidence other than a particular kind of faith in
	science.

	\begin{displayquote}
		[{\citetitle[xvii.f]{carnap2003}}] Whence then our confidence that our call for
		clarity, for a science that is free from metaphysics, will be heard? It
		stems from the knowledge or, to put it somewhat more carefully, from the
		belief that these opposing powers belong to the past. \dots Our work is carried
		by the faith that this attitude will win the future.
	\end{displayquote}

	His inability to admit of value-statements leads him directly into the famous argument
	against positivism, that the verification principle itself is a prescription, and
	cannot be verified. Any justification for normative ethics must be made within
	the language of science, but the objects of its study are a matter of \emph{psychology}
	and not something which can be derived from universal principles of reason. Any
	ethical claim must take the shape of formal-logical arguments proceeding from psychological
	principles of emotion, ultimately amounting to an arbitrary (set?) of formal languages
	which capture the ethical feelings we experience as part of our basic experience.
	In all cases, logic comes first, and the emotions come second.

	It is precisely this attitude which spells out the dangers of metaphysics. Metaphysical
	speech has the appearance of scientific rationality, while in fact only
	prescribing an orientation toward the world. This can lead to an unearned sense
	of authority for the individual. Carnap's argument against metaphysics - and
	especially against Heidegger - consists in this: speculative metaphysics is inherently
	authoritarian by lack of verifiably. In contrast, a communist ethics undoubtedly
	places a large emphasis on the authority of the group and the sentiment of comradery
	which is essential for the scientific enterprise.

	In summary: (1) Science is logistic in nature, proceeding from basic experiences
	to the construction of the world, while value postulation is a matter of
	irrational prescription, an animal reaction to chance situation. (2) The
	material mode of speech is natural but insufficient, leading us into confusion.
	If one is to proceed scientifically, they must speak in a way which is translatable
	into the formal mode of speech. (3) Material language has potentially
	deleterious psychological effects. The formal mode of speech most clearly aligns
	itself with a communitarian ethic, and specifically, the communist politic,
	while the material mode of speech aligns itself most easily with
	authoritarianism, specifically, the fascist politic.

	%-------------------------------------------------------------------------%
	\section{Beyond Rationality}

	Heidegger, like Carnap, begins his analysis from phenomenological grounds. But
	his idea of a phenomenon is quite different. His system is very intricate, and
	we will only have the space to cover a small fraction of it here. We will focus
	only on what is necessary to contrast his understanding of scientific rationality
	from Carnap's.

	Heidegger's project is to understand the essence of Being in its most general form.
	Already Carnap would stop us, claiming that the entire project is based on a delusion:
	we are here beginning our investigation in the material mode of speech. Already
	\enquote{the essence of Being} is untranslatable. Heidegger knows this, and in
	fact this is his point. In \textit{Being and Time}, Heidegger engages in the form
	of a scientific treatise. Later, he will forsake this method in favor of a more
	poetic style, essentially agreeing with Carnap that the scientific form is
	inappropriate for \enquote{the task of thinking.} Nonetheless, let us continue
	outlining Heidegger's fundamental ontology by examining his phenomenological
	method.

	The method hinges on his understanding of apophantic logos, that is, the
	discursive basis for analysis and truth-bearing in general. Logos is a letting-something-be-seen.
	In order for something to be seen as true or false, it must be placed into a discourse
	which allows it to be seen as it really is. The way we speak then plays a
	large role in how we confront Being. Truth in Heidegger's sense relates more closely
	to \blockquote[{\citetitle[32]{heidegger2008b}}]{the sheer sensory perception of something}\footnote{References
	to Being and Time are to the original page numbers, prefixed with an H. in the
	cited edition.} than to any universal abstraction. Logic then operates on basic
	phenomenal experiences, and phenomenology is the analysis of basic experiences
	as they are. But the phenomenon is unanalyzable in the sense that it has no constitutive
	parts; there is nothing \enquote{behind} the phenomenon to be uncovered (\citetitle[36]{heidegger2008b}).

	If the phenomenon makes itself manifest as it is in itself, then what work is
	phenomenology to do? The work of phenomenology is to distinguish the phenomenon
	from the appearance. The appearance is manifestly what it is not; it hides its
	true nature, the Being of the phenomenon from which it originates.

	Phenomenology cannot proceed in the form of propositional assertions. This form
	of logic separates the essential character of the Being from its onto-historical
	context. By analyzing it in the abstract, the essence \enquote{gets understood in an empty way}
	and \enquote{loses its indigenous character} (\citetitle[36]{heidegger2008b}).
	So \emph{logos} cannot merely be \emph{logistics}. But what is this indigenous
	character?

	Heidegger's phenomenology is laid out as a hermeneutic of lived experience. He
	rejects abstract ideals and timeless essences in favor of a factical and
	historical study of human nature. For Heidegger, Being itself is situated in
	time, as it is \emph{through} us and \emph{for} us that Being takes its shape in
	the first place. Being, and thus truth, is historical in nature.

	% The dialectical characteristic of logos presupposes the being-with-others of Dasein,
	% its relation to \emph{das Man.} All of the essential characteristics of Dasein's
	% Being seem to occur simulatneously; they characterize it rather than appear to
	% it. What is Heidegger's epistemology?

	Heidegger's criticism of the scientific attitude can be most clearly seen in his
	criticism of Descartes (\citetitle[{I.3 B, espc. \P21}]{heidegger2008b}). Here
	he talks fairly explicitly about values. The critique of Descartes hinges on his
	conception of being-in-the-world, as opposed to beings present-at-hand. He claims
	that Descartes mischaracterizes the essential nature of beings as Nature, that
	is, as founded on \enquote{an entity within-the-world which is proximally present-at-hand}
	(\citetitle[95]{heidegger2008b}). As in the \enquote{Question Concerning Technology,}
	\nocite{heidegger2008c} Heidegger claims that Descartes has essentially made the
	world ready-to-hand, thus enclosing it and making all of Nature, including man,
	into \enquote{standing reserve.} This view is somewhat ironic considering Heidegger's
	perceived Nazi sympathies, though reading Heidegger as a clear-cut Nazi
	hardliner is naive. In \enquote{Only a God can Save Us,} Heidegger explicitly
	denies any ties with the Nazi party, considering himself an \enquote{unpolitical}
	person. In fact, he characterizes himself as \enquote{confronting the Nazi party}
	in his Nietzsche lectures, and as the subject of scrutiny under Nazi rule. Whether
	or not he was truly sympathetic to Nazi politics is a matter of debate, but in
	any case he is explicitly opposed to socialism as \enquote{a form of planetary technicity}
	\autocite[206]{heidegger1981}.

	Carnap actually would agree with Heidegger that Descartes' project is
	insufficient, but his critique would be quite different. Instead of offering
	an analysis of different \enquote{modes of being} as a criticism of the Cartesian
	worldview, he would offer the gestalt-theoretical experience as having nothing
	to do with an ego. This has to be constructed from the aspects of the experience;
	\textit{res extensa} is not the most fundamental

	--

	Heidegger prefers a holistic approach. The analytic of Dasein occurs piecemeal,
	but each piece influences and is influenced by the whole. Carnap prefers a constructivist
	approach, where the basic units of experience are assumed, and the rest is built
	from there.

	% %%%%%%%%%%%%%%%%%%%%%%%%%%%%%%%%%%%%%%%%%%%%%%%%%%%%%%%%%%%%%%%%%%%%%%%%%%%
	% % Main resource: Stone 2007
	% \section{Facts and Values}

	% % Marburg school's constructionism to avoid Southwest school's schism between fact and value.
	% % Heidegger's direct realism - adjoining fact and value through the existential hermeneutic.
	% % Carnap's empiricism, intersubjectivity through purely formal construction.

	% Descartes' philosophical project begins with radical doubt. Putting aside all
	% experience in a skeptical \textit{epoch\'e}, we are left with the Cartesian
	% ego as the subject of experience. From this radical subjectivity, we build up
	% the external world. So, for Descartes, essence precedes existence: ontology
	% begins from an intellectual basis. Sensory experience is justified through an appeal
	% to our mental faculties. Husserl, like Descartes, proceeded from a bracketing
	% of experience, and naturally found himself postulating essences. In contrast
	% to this, Heidegger and Carnap both reject the primacy of the mental,
	% preferring instead to proceed from the material to the essential. It is useful
	% for us to see how Carnap and Heidegger differ in the critiques of Husserl.

	% Stone argues that Heidegger and Carnap both wished to save ethics by limiting the
	% pretensions of metaphysics. For both, this manifest itself in political action.
	% However, both authors operate off of a kind of \enquote{faith} - a faith in
	% the future of humanity which they as of yet cannot see or determine. This faith
	% is Nietzschean in nature - for Heidegger this is quite explicit, given his
	% famous lectures. For Carnap this faith is less obvious. \autocite{sachs2011}

	% In the scientific world-conception, we proceed from our senses to make general
	% claims about the world through logical analysis. Essence is rejected; only what
	% can be verified empirically is accepted as real. Metaphysics, as purely intellectual,
	% is rejected as an unnatural attitude toward the world. As Heidegger saw, and
	% as many critics of the Vienna Circle have pointed out, this rejection of the metaphysical
	% does no good to explain why the scientific world-conception should itself be accepted;
	% it is a metaphysical claim without justification. A positivist might respond that
	% there \emph{could} be no verifiable justification; while we are free to debate
	% how we ought to live, this goes beyond the scope of knowledge, and to claim otherwise
	% is empty pretension. Heidegger, on the other hand, does not proceed from the
	% strict demarcations of science, but rather finds his existential subject among
	% other subjects in the world; it is the subject's existence among other
	% subjects which imbues it with meaning, and which allows for it to relate to the
	% world in various ways. Without this being-in-the-world, being-among-others,
	% and being-towards-death, there is no metaphysics (for Heidegger, fundamental ontology);
	% and without metaphysics, there is no justification for science.

	% I will define the concepts of object, science, language, and logic as both thinker
	% see it. We will need these definitions to work out just what they mean when
	% they speak of metaphysics.

	% Heidegger and Carnap share the same conception of an object. In Carnap:
	% \blockquote[{\cite[5]{carnap2003}}]{The word \enquote{object} is here always used in its widest sense, namely, for anything about which a statement can be made. Thus, among objects we count not only things, but also properties and classes, relations in extension and intentions, states and events, what is actual as well as what is not.}
	% And in Heidegger: \blockquote[{\cite[5]{heidegger2008b}}]{Whenever one cognizes anything or makes an assertion, whenever one comports oneself towards entities, even towards oneself, some use is made of \enquote{Being}; and this expression is held to be intelligible \enquote{without further ado}, just as everyone understands \enquote{The sky \emph{is} blue}, \enquote{I \emph{am} merry,} and the like.}
	% So in both thinkers the concept of "being" or "object" is applied to both substantives
	% (\enquote{what is actual;} \enquote{whenever one comports oneself towards entities}),
	% and the objects of predication (\enquote{properties and classes, [etc.]}, \enquote{whenever one cognizes anything or makes an assertion}).\footnote{I
	% take this reading from \enquote{Heidegger and the grammar of being} in \cite[chap.
	% 15]{priest2002}}

	% Both authors wish to understand what \emph{is}, in the widest sense. Both authors
	% have an interest in science and logic, though they take differing stances on
	% its priority, and both authors have an interest in understanding the way we speak
	% about beings. Heidegger and Carnap believe in a unified science. For Carnap,
	% this takes the form of a constructional system. His emphasis is on the structure
	% of language and its relation to things in the world, later turning to
	% semantics to understand how the formal syntax we use relates us to the things
	% we observe. Carnap's project is \blockquote[{\cite[7]{carnap2003}}]{an attempt \emph{to apply the theory of relations to the task of analyzing reality.}}
	% Although he uses this phrase to characterize the project of the \textit{Aufbau},
	% there is no reason to suppose his general project changed over the course of his
	% career.

	%%%%%%%%%%%%%%%%%%%%%%%%%%%%%%%%%%%%%%%%%%%%%%%%%%%%%%%%%%%%%%%%%%%%%%%%%%%
	% Main resources: Sachs 2011
	% \section{Metaphysics - How to Found the Sciences?}

	% For Heidegger, \enquote{metaphysics} is characterized by a misuse of language,
	% that is, it is the linguistic fallacy of speaking of Being as \emph{a} being. Where
	% we need to speak of the characteristic essence of beings as a whole, we are
	% instead speaking of essential beings. For Carnap, it's quite similar. Metaphysics
	% is the misuse of language to speak of things which cannot be structurally analyzed,
	% that is, properly thought.

	% Carnap characterizes Heidegger's misuse of language in \enquote{The Elimination}
	% as an attempt to substantiate something which cannot be substantiated. That is,
	% Heidegger attempts to speak of \enquote{the Nothing} as if it is in fact a something,
	% a substantial thing. However, this clearly is nonsense under the guise of
	% first-order predicate logic. There is no way to adequately capture \enquote{the Nothing nothings}
	% with the existential quantifier. Carnap outlines a transition from ordinary language
	% to formal language. He characterizes the transition first by creating an
	% object, which we call \enquote{nothing} ($no$). We then construct various predicates
	% about \enquote{the nothing} by instantiating them with the pseudo-object. However,
	% Carnap argues that what we mean by \enquote{nothing} cannot be instantiated,
	% as it is represented by a quantifier ($\neg\exists x.Px$ for some property $P$).
	% So he argues that we ought to dismiss Heidegger's talk as nonsense.

	% It is fairly apparent to anyone with an understanding of higher-level logics that,
	% in fact, we \emph{can} predicate over quantifiers. We could write \enquote{the Nothing nothings}
	% as something like the following: $N(\neg\exists x . Px)$, and then express
	% properties of this nothing-ing predicate as appropriate. So even on his own terms,
	% given the advances in logic since the article was written, Carnap's argument fails.
	% He acknowledges as much in later writings (SOURCE), but still maintains that his
	% interpretation of Heidegger is correct. How does he claim this?

	% Even if we are to predicate over nothing, this form is highly unwieldy, and
	% does little to explain exactly what Heidegger is trying to do by
	% substantiating nothing.

	% \section{Putting the Question Rightly}

	% From the above discussion, it should be clear that our two thinkers are in agreement
	% about the following: (1) Science cannot capture values; (2) The way we speak is
	% deeply important to the philosophical aims we wish to pursue; (3) The Western
	% philosophical and political traditions need to be radically altered. Now we turn
	% to their disagreements, specifically the following points: (4) There is philosophically
	% meaningful, sensible discussion to be had beyond scientific discourse; (5) Logic,
	% especially logistics, is the key to rationality, and to knowledge generally; (6)
	% We must embrace technology as the key to our political aims.

	% Carnap and Heidegger both emphasized the importance of language to our understanding
	% of the world. While Heidegger explicitly used poetic language to speak about
	% what lies beyond science, Carnap believed that that which science cannot formalize
	% should explicitly be avoided in philosophical discourse. Any extrascientific
	% influence is akin to metaphysics: it cannot be formalized, and so should be
	% avoided. In fact, it can be actively dangerous. Carnap worried that metaphysical
	% language, though its claims are unverifiable, could have \blockquote[]{psychological impacts}
	% which could influence by an appeal to authority rather than to reason. This was
	% not without good cause: Gogarten and Heidegger would both endorse the Nazi
	% party. \autocite{dambock2022}

	% The logical positivists preferred to analyze the logical structure of scientific
	% language over material analysis. This is directly reflected in the title of
	% Carnap's first major work, \textit{The Logical Structure of the World} \nocite{carnap2003}.\footnote{I
	% will refer to this book as the \textit{Aufbau}.} In this book he would proceed
	% from the atomic structures of our reason to construct an entire empirical worldview
	% supposedly capable of recreating the world of science. His aim is to show the
	% unified basis of all the sciences, that is, to show that there is only one
	% proper science, and that the separation of the sciences into epistemologically
	% and ontologically distinct subfields is a mistake. In line with the rest of
	% the circle, this was to be done through conceptual reduction and reconstruction.
	% At the time of writing, Carnap's focus was to create a single constructional system
	% which would underlie the sciences. As his work matured, and by the time of his
	% later major work, \textit{The Logical Syntax of Language} \nocite{carnap2000},
	% he would shift his view to accommodate flexibility in the choice of the formal
	% system. He calls this the principle of tolerance. \blockquote[{\citetitle[{\S 17}]{carnap2000}}]{Everyone is at liberty to build up his own logic, i.e. his own form of language, as he wishes. All that is required of him is that, if he wishes to discuss it, he must state his methods clearly, and give syntactical rules instead of philosophical arguments.}
	% Nonetheless, the method of logical reduction and the autopsychological basis
	% of scientific theory holds. This basis is directly related to Husserl's through
	% the use of the \textit{epoche}, or the bracketing of psychological experiences
	% as real or unreal (see \citetitle[{\S 64}]{carnap2003}). Instead of proposing
	% a realm of abstract entities alongside the conceptual connections between them,
	% Carnap would propose only the logical structure, with the objects of that
	% structure taken to be entirely arbitrary.

	% Carnap and his circle's strict adherence to their scientific aims, and their rejection
	% of metaphysics, is directly influenced by their political goals. Carnap writes
	% in the preface to the \textit{Aufbau},
	% %
	% \blockquote[{\cite[xvii-xviii]{carnap2003}}]{our confidence ... stems from ... the belief that [metaphysics and theology] belong to the past. We feel that there is an inner kinship between the attitude on which our philosophical work is founded and the intellectual attitude which presently manifests in entirely different walks of life. ... It is an orientation which demands clarity everywhere, but which realizes that the fabric of life can never quite be comprehended. ... Our work is carried by the faith that this attitude will win the future.}
	% %
	% The work of the \textit{Aufbau} and the rest of his corpus can only be
	% explained by a pragmatic choice, the choice to pursue the method which has so far
	% given us the most progress in science, and which promises, according to the value
	% theory held by the group, to provide the most equitable and prosperous
	% conditions for mankind. His justification for the scientific worldview thus appeals
	% to the political and social aims of the leftist movements of the \nth{20}
	% century.

	%=========================================================================%
	\section{Conclusion}

	% Carnap and Hiedegger come to opposite conclusions from the same background. Both
	% wish to confront the incommensurability of facts and meaning. Carnap does so
	% by expunging meaning from scientific discourse, leaving it to the poets to sort
	% out. For Carnap, scientific discourse is central. Logic ought to guide us more
	% than the irrational whims of desire; it is left to the poets to express these \enquote{attitudes toward life.}
	% Although both agree that the sciences are preceeded by the emotions, and that in
	% life we are always already experiencing some attitude toward the world, they
	% disagree about what this means for human life. Heidegger takes the ontical primacy
	% of the emotions as evidence of their philosophical priority. If we are to
	% orient ourselves to the world in a meaningful manner, we must think through our
	% moods and our emotions; rather than viewing them as a useful but irrational
	% orientatating schema, we ought to consider them as precisely indicative of a relation
	% to the world. Only through an analysis of this sort can we come to understand
	% how we ought to relate to the world. However, the primacy of emotive, orienting
	% rationality is not to deny the categorical rationality of logic. The
	% difference in our author's positions becomes clear in their stances towards
	% language. For Carnap, formal syntax is the proper mode of scientific
	% communication, and thus the only way to properly enunciate an idea (where by
	% \enquote{idea} we mean the formal, Kantian notion of an idea: that of an empirical
	% statement about the world). %This is incorrect.

	% Political and geospatial differences fuelled the distance that Carnap and Heidegger
	% felt from each other, despite working from similar grounds. If we are able to overcome
	% these differences to see the throughline, perhaps we could reinvigorate a more
	% linguistic turn in analytic value theory. % Sellars.

	\printbibliography
\end{document}