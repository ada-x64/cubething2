%%%%%%%%%%%%%%%%%%%%%%%%%%%%%%% cubething.dev %%%%%%%%%%%%%%%%%%%%%%%%%%%%%%%%%

%preamble
\documentclass[leqno]{article}

%biblio
\usepackage[authordate, backend=biber, natbib]{biblatex-chicago}
\addbibresource{../../../library.bib}

% american mathematical society
\usepackage{amsmath}
\usepackage{amssymb}
\usepackage{amsthm}
\usepackage{mathtools} % extra symbols

% formatting
% \usepackage{CormorantGaramond}
\usepackage{csquotes} % smart quotes
\usepackage{enumerate} % enumeration tools
% \renewcommand{\thesection}{\S section} % fancy subsections
\makeatletter
%% See pp. 26f. of 'The LaTeX Companion,' 2nd. ed.
\def\@seccntformat#1{\@ifundefined{#1@cntformat}%
{\csname the#1\endcsname\quad}%      default
{\csname #1@cntformat\endcsname}}%   individual control
\newcommand{\section@cntformat}{\S\thesection\quad}
\newcommand{\subsection@cntformat}{\S\thesubsection\quad}
%\newcommand{\subsubsection@cntformat}{\S\thesubsubsection\quad}
%\newcommand{\paragraph@cntformat}{\S\theparagraph\quad}
%\newcommand{\subparagraph@cntformat}{\S\thesubparagraph\quad}
\makeatother % changes @ back to a special character

%opening
\title{Heidegger, Carnap, Husserl, and the Foundations of Rationality}
\author{Phoenix Ada Rose Mandala}

\begin{document}
	\maketitle

	\section{Heidegger against Husserl}

	Transcendentalist ego is itself too far removed from lived experience to be useful
	for a phenomenological investigation towards Being as such. We must proceed
	from deeper ends, namely our Being-in-the-world; our thrownness; our fundamental
	orientation towards the world in mood; etc. This is a \emph{hermeneutical}
	process because it involves an \emph{interpretation} of our lived experience.

	The goal of Husserl's investigations was epistemological - to find a firm basis
	for the sciences. But it is the history of philosophy - which gave rise to the
	sciences, and which praises them as the highest form of wisdom - which itself \emph{limits}
	the phenomenological method; its reliance on \emph{epistemology} and this
	field's reliance on Cartesian doubt - reconfigured in Husserl as the epoch\'e
	- its reliance on the ego as disconnected, as \emph{transcending} or \emph{ideal}
	- \emph{this} is what limits and dooms the Husserlian project. Only through an
	analysis of our \emph{factical, lived} experience will we be able to
	understand what drives us toward knowledge.

	Rorty says there is a gap left in the destruction of epistemology - this is
	where we \emph{discuss} - it is where the \textit{logos} happens. Carnap does not
	allow for fundamental ontology because of his dependence on a naturalistic,
	epistemological world-view. All of nature, reason, and logic is law-like,
	strict, non-contradictory. This is surely excellent for making predictions
	about the world, but it misses the way we actually live; it is a step \emph{beyond}
	the factical world and into transcendence. Metaphysics exists \emph{beyond} the
	epistemological. It exists prior to logic; it exists in \textit{logos}, in discussion,
	in interpretation, in dialogue or \emph{dialectics}. Knowledge proceeds in slow
	movement towards the true through contradiction and reconciliation - it is dialectical
	by nature. No epistemology with law-like characteristics - no naturalistic
	epistemology, no law of non-contradiction - can capture this.

	Symbolic formalism concretizes what is essentially a time-oriented process.
	Knowledge and limits do not exist outside of a time structure. The accumulation
	of knowledge is a slow \emph{progression}, and an \emph{unfolding} of truth. Truth
	may be eternal, but it is not \emph{static}. It depends on the time context
	and how truth is felt, seen, revealed to us is a matter of unfolding through
	time. It takes time even to formulate a thought.

	Science and math feel \textunderscore{static} in comparison to hermeneutics. Why?
	What is the distinction? This is contrasted to flow or wu-wei -- flow can
	happen in any case. There is a distinct static eternality to the universal.
	This is contrasted with the dynamic flow of the particular. The universal \emph{transcends}
	the particular and holds itself above in perfect stillness. What is the origin
	of this phenomenon?\footnote{This is a test footnote, don't mind me.}

	\section{On Kant}

	Transcendental ideas: God, the world, the soul - preconditions for rational
	knowledge; we cannot understand anything without presupposing these ideas.
	These cannot be \emph{known} because they exceed the boundaries of possible experience.

	Dialectical - to posit the limit is already to move beyond. The problem
	dissolves. Transcendental conditions of knowledge are revealed by dialectically
	opposed ideas; ideas which are rationally consistent but mutually exclusive. For
	example: freedom cannot be proven, but we act as if we are free; it is better
	if we act this way.

	"To make room for faith;" a belief in the absurd! - Connections to Kierkegaard.

	Todo: Draw on the Neokantian background for the origin of Husserlian phenomenology,
	the move away from Neokantianism towards the Vienna circle analytic thought (Marburg
	disillusionment with metaphysics, emphasis on physics, naturalism) and the
	Continental move towards Hegel (Southwest school, emphasis on history, humanities,
	dialectics).

	\section{Carnap against Heidegger}

	Carnap's conception of logic and of the world is constructionist. We build up the
	world from basic, phenomenological blocks, though these phenomenological building
	blocks are so minimal that they are hardly worth speaking of. For Carnap in
	the Aufbau, there is only a single building block. His criticism of Heidegger
	then can be clearly seen as an attack on Heidegger's violation of the logical
	basis of thought. It is an attack on psychologism, a trend which was deeply
	popular and, since Husserl, difficult to defend. Carnap is known to have been familiar
	with Husserl, even mentioning him in the Aufbau, and studying him closely before
	reformulating the Aufbau extensively around 19xx. However, Carnap, like many
	of his contemporaries, probably did not see the importance of the phenomenological
	project which underlies Husserl's prolegomena to the Logical Investigations. A
	quote from the excellent Introduction to Husserlain Phenomenology can
	elucidate this point.

	\begin{quote}
		If logic is naturally related to possible acts of thinking, then a precise, reciprocal
		demarcation of the conformity of such acts with logical as well as psychological
		laws is not merely the preparatory task of methodological work, rather it is
		the central task of logic itself. The rejection of logical "psychologism,"
		that is, the emancipation of "pure" logic from its ordination within (empricial-genetic)
		psychology, may not be separated from its positive counterpart, the cultivation
		of a novel, eidetic-descriptive "psychology." \autocite{bernet1993}
	\end{quote}

	Carnap, in rejecting psychologism, wishes to found our empirical understanding
	on as minimal a basis as possible. This is an admirable task. Heidegger, on
	the other hand, focuses heavily on the eidetic-descriptive task, outlining in
	detail various "modes of Being" under which we live our daily lives. For Heidegger,
	this is a primary task, it lies at a deeper root than logic can aspire to. For
	Carnap, a rich descriptive account of proto-logical thought is more or less what
	philosophers have already \emph{been} doing; it leads to confusion. Philosophy's
	task is to make life clear, and for Carnap this lies precisely in the logical-linguistic
	aspect of cognition. Heidegger's focus on emotive, poetic language is not only
	backwards, but impossible to formulate within a perfectly clear language. It is,
	strictly speaking, unthinkable, and therefore it is nonsense; worse, it is "bad
	poetry."

	\section{Metaphilosophical concerns}

	It would be helpful here to talk about The Task of Thinking. A summary: Philosophy
	as metaphysics - the study of the Being of beings - is nearing its end. It
	realizes this end in cybernetics, the total technoligization of human activity.
	By all accounts science is correct. What is left for philosophy? We have
	answered the call "to the things themselves," the call to examine subjectivity
	and its constitutive manifestations in full detail. we have received various
	answers and they have yielded from their speculation a totalizing science
	which itself subsumes the task philosophy set out to accomplish. But what remains
	is to understand where the underlying principles are grounded. What grounds
	rationality? What needs proof, and what does not? Heidegger speaks of a clearing,
	a space which opens for Lichtung to shine through. Lichtung is knowledge. But
	what is this clearing, what is it that shades and opens itself up so light can
	shine through? This is the task of thinking: the task of understanding \textit{al\=ethia}
	as unconcealment.

	Carnap, Husserl, Hegel and Heidegger himself were al misguided here. There is a
	"more originary questioning" which needs to occur.

	It's important to note the shared emphasis on science, logic, and technology betwene
	Hiedegger, Huserl, and Carnap. For Carnap, philosophy is the handmaiden of science.
	Philosophy's essential task was never metaphysics, but \emph{logic}. Epistemology
	is a part of this. Ontology plays no role; objects are constructed from simple
	sense perception and the rest follows a constructional systme. This is at
	least the view in the Aufbau; I do not believe he ever repudiated the main claim,
	that of constructionism. Heidegger says yes, you've got the idea: philosophy up
	to this point has been deeply focused on the Being of beings, in metaphysics. The
	end of metaphysics is science. To that end philosophy's goal - the end of
	philosophy - is cybernetcis, the full systemaization of human life. But there
	is a more oringary questioning. Logic is rationality emobdied - but logic itself
	admits of a psychological origin in phenomenology. The critique of
	psychologism cannot be complete without the admittance of a transcendental
	intuition, says Husserl; thus the sciences must be founded on phenomenology. Well
	and good says Heidegger, but let's dig deeper. What is it that allows us to be
	\emph{aware} of our intuition in the first place? On what grounds are the self-lighting,
	intuitive ideas made apparent? In a way, Heidegger is asking the hard problem:
	what is it that gives ground to consciousness? What grounds rationality?

	He does not give us an answer. To attempt that would be deeply immature. All
	we can hope for is a clarification of the question. This is what Heidegger hopes
	to give us, and it is this question itself which Heidegger believes all
	Western philosophy has until this point been blind to.

	--

	Heidegger's method relies on a historically grounded analysis of language, the
	development of language over time and an attempt to recreate the conditions of
	its origination in order to reveal the intention behind its usage. For
	Heidegger, this allows us to seek the underlying truths which are hidden behind
	layers of translation and historical obscurity. It allows us to see the
	foundational preconditions for philosophy as such, and so to form a critique of
	the foundational mode of philosophical knowledge. This is what is meant by the
	task of thinking: we are tasked with understanding how philosophy originated so
	that we can undergo this same process, perhaps to extend its purview or at
	least to expand the clearing by which we can see what really is. This is contrasted
	to Carnap's linguistic method, by which we attempt to get clear on concepts in
	a time-independent way. We abstract from language to an \emph{ideal} language,
	one which is perfect in its syntactical and semantic clarity. This is a
	wonderful approach, but one cannot make sense of it without the historical context
	in which it arose. A timeless, ideal language can only be conceptualized in the
	light of absolute Being, in light of the ideal. This conception is shaped by the
	Platonic Idea - this is apparent even in the name, "ideal." But what is an idea?
	From Descartes forward, an idea is a matter of subjectivity - so Carnap's goal
	is to find perfect clarity of rational mind: the objective principles by which
	subjectivity claims to understand the world. The basis of this questioning lies
	in the pursuit of \emph{truth.} But what is truth? Tarski would argue that truth
	is something system-dependent. Truth obtains within a logical system. The early
	Carnap and Heidegger, at which point the debate occurs, are both laboring under
	the notion of an absolute truth. It is unclear whether Carnap abandons this
	idea, but by the time he writes The Task of Thinking it is clear that Heidegger
	has moved beyond absolute truth. As explained above, Heidegger's pursuit has moved
	away from Being and Time toward Lighting and Clearing. What does he take truth
	to be?

	\begin{quote}
		Insofar as truth is understood in the traditionl "natural" sense as the correspondence
		of knowledge with beings, demonstrated in beings; but also insofar as truth is
		interpreted as the certainty of the knowledge of Being; \textit{al\=etheia},
		unconcealment in the sense of the clearing, may not be equated with truth.
		Rather, \textit{al\=etheia}, unconcealement thought as clearing, first
		grants the possibility of truth. For truth itself, like Being and thinking, can
		be what it is only in the element of the clearing. \autocite[446]{heidegger2008a}
	\end{quote}

	Here Heidegger is searching not only for the basis of the correspondence theory
	of truth, but also for the origin of fundamental ontology. He is looking for the
	basis of scientific thought as well as metaphysics. He takes truth to have
	multiple meanings, a meaning within science and a meaning within metaphysics.
	But how could it be that truth has multiple senses? There seems to be a rejection
	of the notion of absolute truth. Tarski would agree - and the later Carnap
	would famously propose a principle of tolerance, by which metaphysical confusion
	and confabulation can be replaced with a choice of a logical framework under which
	propositional statements (thus truth predicates) can be clearly and systematically
	articulated.

	\printbibliography
\end{document}