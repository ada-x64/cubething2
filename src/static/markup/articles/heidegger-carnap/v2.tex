%%%%%%%%%%%%%%%%%%%%%%%%%%%%%%% cubething.dev %%%%%%%%%%%%%%%%%%%%%%%%%%%%%%%%%

\documentclass[leqno, 12pt]{turabian-researchpaper}
\usepackage{../../cubething}
\addbibresource{../../library.bib}

\usepackage{pifont}

\renewcommand*{\bibfont}{\small}

%opening
\title{What Must Be Left Unsaid, and Saying It Anyways: \\ Logic and Metaphysics
in the Early \nth{20} Century}
\author{Phoenix Ada Rose Mandala}

\begin{document}
	\maketitle

	Carnap's 1932 \enquote{Elimination of Metaphysics} \nocite{carnap1966} is
	typically criticized as a poor reading of Heidegger. However, recent works
	have come to reject this reading. Although Carnap's paper itself fails on several
	fronts, it is indicative of the overall character of his project, and there is
	good reason to suppose that Carnap would read Heidegger seriously. In fact, the
	two authors share many of the same predispositions and assumptions. In this
	paper, I wish to explore these undercurrents, and to expose the origin of the
	difference between these giants of \nth{20} century philosophy.

	First, I will go over the historical and political movements which fuel their
	projects. The philosophers' positions will then be discussed against the then-dominant
	Neokantian school and the influence of Edmund Husserl's phenomenology. I will then
	discuss the matters on which they agree and find themselves in contrast. The
	discussion will proceed through an analysis of the authors' conceptions of fundamental
	epistemology, metaethics, and their relation to language.

	%=========================================================================%
	\section{Historical Background, Political Origins}

	% TODO: https://www.youtube.com/watch?v=RqESHNvmP20
	% The petit-borgeousie; material analysis

	Carnap and Heidegger were young men in 1914, when the world went to war. The horror
	of industrial warfare and the decline of Western religious practices scarred a
	generation, prompting a radical political split. The classic dilemma of is and
	ought had come to engross the age: What are we to do about the
	incommensurability of (empirical-scientific) facts and (theological-existential)
	meaning? In 1917, the sociologist Max Weber wrote the following.

	\begin{displayquote}
		[{\cite[139]{weber2014}}] \dots what is the meaning of science as a vocation,
		now after all these former illusions \dots have been dispelled? Tolstoy has given
		the simplest answer, with the words: \enquote{Science is meaningless because it gives %
		no answer to our question, the only question important for us: \enquote{%
		What shall we do and how shall we live?}}
		That science does not give an answer to this is indisputable. The only question
		that remains is the sense in which science gives \enquote{no} answer, and whether
		or not science might yet be of some use to the one who puts the question correctly.
	\end{displayquote}
	% This essay aims to clarify why science gives no answer.

	% rough
	For Carnap and his Vienna Circle, the limitations of science are themselves
	the limitations of knowledge. Anything which can be said about material things
	(including individuals and their societies) can be stated clearly, in
	scientific language. The rise of psychology and sociology as empirical disciplines
	alongside the development of formal logic provided further fuel for their attitude.
	Metaphysical pronouncements outside the scope of logical analysis -- about abstract
	essences or universal ethics -- were considered \enquote{nonsense.} This term is
	not to be taken as a complete dismissal, however; the scientific worldview is meant
	to \enquote{serve life}, not the other way around. Practical philosophy is to
	proceed from empirical grounds, as a sociopsychological science. But the
	Vienna Circle was not without their preferred conclusions.

	In \enquote{The Scientific Conception of the World,} \nocite{hahn1973} they
	explicitly link their worldview with the socialist movements of the early \nth{20}
	century. They acknowledge a widespread disenchantment among Weberian lines, connected
	with the industrial mode of production. The European public \blockquote[{\cite[21f]{hahn1973}}]{with their socialist attitudes tend to lean towards a down-to-earth empricist view. In previous times \emph{materialism} was the expression of this view; meanwhile, however, modern empiricism has ... taken a strong shape in the \emph{scientific world-conception}}.
	Their solution to the disenchantment can be found in the hope they place in science
	and collectivist movements taking shape throughout the continent.

	While scientific reason takes the priority for the positivists, Heidegger
	focuses on the lost sense of existential meaning. For him, the Western
	philosophical project is extremely fruitful, but overly restrictive. He aims
	to destruct the canon, clearing the space for a new philosophy. The scientific
	worldview is, after all, one among many.

	It is well-known that Heidegger took the rectorate at Freiburg under Nazi
	reign, and delivered several lectures expounding on the \enquote{inner greatness}
	of the Nazi party. Throughout his life, he was taciturn about his involvements
	in the party. In a late interview with \textit{der Spiegel}, published only posthumously,
	Heidegger claims his involvement with the Nazi party was at least partially
	out of a sense of obligation, in order to keep the university intact, and to
	secure his position as rector. Regardless of the nature of his involvement with
	fascism, it is clear that Heidegger was against communism in any form, especially
	the technocratic version which the Vienna Circle would undoubtedly espouse. He
	argues that communism and \enquote{Americanism} (what I assume to be
	industrial capitalism) are forms of global technicity, that is, a movement away
	from viewing human beings in their full nature towards the total organization and
	technologization of human behavior in cybernetics. In his language, all beings,
	even Dasein itself, become uprooted from the earth and become ready-to-hand,
	no longer capable of being seen in their true essence. Although he does not make
	his political end-goal explicit, it is against this technicity, and towards a \textit{Volkish}
	ideal that Heidegger strives \autocites{heidegger1981,heidegger2008c}.

	Heidegger and Carnap are both seated within the Western philosophical tradition,
	and both are attempting to overcome it. Their methods differ quite dramatically,
	however, as do their end goals. Carnap wishes to clear away the cruft of the
	past few millennia in order to see rightly; fundamentally, he believed the
	Western canon was on the right track. Heidegger, however, wants to destruct the
	method and the worldview which evolved from Plato through Kant. Both do so in
	order to found a worldview which is capable of overcoming the existential
	conditions which underlie false metaphysical inquiry. These conditions are immediately
	tied up in both material analysis and practical philosophy, in questions
	concerning both empirical facts and lived meaning. In the next section we will
	take a look at their shared philosophical background, after which we will
	analyze how they each moved past it.

	%=========================================================================%
	\section{Kant and Husserl}

	% Heidegger and Carnap were both educated in the Neokantian tradition. In keeping
	% with the post-Kantian idealists, the Neokantians rejected Kant's distinction
	% between the realms of pure sensibility and understanding. Over spatio-temporal
	% intuition, they preferred a pure logical basis for the object of knowledge. \autocite[28]{friedman2000}
	% The tradition was divided between several schools, most notably the Marburg
	% school and the Southwest school. While there are many differences between these
	% camps, the primary distinction I will focus on is their stance on the
	% distinction between logic and values.

	% Logic, being a discursive science aimed at proper thinking, was considered
	% \emph{normative.} For Rickert, a key player of the Southwest school, this leads
	% to the claim that logic belongs to the realm of transcendental value. Pure
	% logic is then to be distinguished from psychological knowledge, which proceeds
	% by empirical means. This leaves us with the gap between \emph{logic} as value,
	% and \emph{psychology} as fact \autocite[34ff]{friedman2000}.

	% Cassirer of the Marburg school rejected this distinction. Instead of cleaving
	% the mind between transcendental psychology and pure logic, Cassirer claims that
	% these gaps are \emph{moments} of cognition. They are not essential features which
	% then are synthesized in to a single cognition, but a unitary relation which
	% has been artificially analyzed into parts. Value and fact are analogous to the
	% universal and the particular, but for Cassirer this is a difference in matter
	% of \emph{viewpoint,} between understanding a thing in its \enquote{thinghood}
	% (as it relates to universal/transcendental forms), and understanding a thing in
	% its \enquote{context} (as a being among beings). This separation between viewpoints
	% is \enquote{artificial,} only occurring after the initial unitary relation
	% between mind and object. The originary perception finds these two aspects in unison
	% \autocite[34ff]{friedman2000}.

	% The parallels between Husserl and the Neokantian schools is fairly clear.
	% Rickert writes positively of the \textit{Logical Investigation's}\nocite{husserl2005}
	% polemic against psychologism, and there are clear parallels between the
	% unitary relation of cognition in Cassirer and Husserl's unitary field of
	% phenomenological consciousness (see \textit{Ideas} \S38)\nocite{husserl2012}. Akin
	% to Rickert, Husserl ends up proposing transcendental phenomenological ideas as
	% the basis of his ontology, but akin to Cassirer he proposes that value and fact
	% are part of the same conscious process.

	% The phenomenon was proposed by Husserl as a way of bridging the gap between the
	% conscious being and the thing-in-itself. If the phenomenon is understood in
	% Kant's way as the empirical intuition of a thing, then it could be said that Husserl's
	% phenomenon is closer to the transcendental intuition itself, the conditions
	% for the possibility of experience in the first place. Reality shows itself through
	% these conditions and through phenomenal appearances. It is the work of

	% Since at least Aristotle, metaphysics has been the study of the first
	% principles that underlie the various sciences and their most generic objects. In
	% addition to the special sciences, there is logic, which studies the analogous structures
	% of each special science. And, above them all lies metaphysics, the science of
	% first principles as such. Importantly, these first principles are accessible
	% as objects of intelligible intuition. That is, although we cannot access them
	% through our senses, we have access to them through our capacity to reason.

	% Kant criticized the traditional metaphysical system for its inherent
	% contradictions. Humans only have access to external objects through our sensible
	% intuition (i.e. through our access to space and time), and to internal,
	% psychological facts through our intellectual intuition. These sensible objects
	% (phenomena) are representations of the underlying 'real' objects (noumena),
	% which are purely intellectual. Although the phenomena are merely representations
	% which occur to perceiving creatures like us, the fact of their being is
	% reliant on the structure of our cognitive faculties. If the grounds of science
	% lie in transcendental principles and causes, then what Kant wishes to say is
	% that metaphysics ought to concern itself only with the way objects appear to
	% us, and not as they are in themselves. To Kant, we do not have direct access to
	% the thing-in-itself, the noumena as such. By making this distinction, Kant
	% rescues the foundations of science while supposedly ridding it of its
	% inherently contradictory character (where possible). However, his motivation
	% for this is not just to save the sciences from falling into skepticism. His primary
	% goal is to \enquote{limit reason in order to make room for faith.} This faith
	% is the rational belief in God as the foundational basis of our capacity for
	% freedom.

	\autocite{stone2017} recognizes three core aspects of Kant's critique of traditional
	metaphysics: (1.) Kant acknowledges that metaphysics correctly identifies the
	grounds of science in the transcendental causes and principles of reason. (2.)
	He denies any knowledge of the suprasensible subject-matter of metaphysics (noumena).
	(3.) His goal is to save practical philosophy by limiting our knowledge, to \enquote{limit reason in order to make room for faith.}
	But Kant's critiques left open two large concerns. First, the notion of
	noumena seems to be largely without basis. Why believe in a realm of suprasensible
	things-in-themselves if we are only given access to the phenomena which represent
	them? Second, the rational faculties are based on spatiotemporal sensory intuition,
	but there is concern about the primacy of these intuitions.

	In order to overcome the limitations of Kant's system, Husserl proposes to link
	our perceptions and the noumena, thus grounding our faculties and overcoming
	representationalism at once. Husserl's claim is that phenomena do not exist as
	objects-in-themselves, independently of our own interpretation. Rather,
	phenomena are those objects whose whole existence is dependent on our
	postulation through an intentional (directed) interpretation of sense
	experience. This transcendental process reveals to us the \textit{Urregion} of
	pure consciousness, containing the necessary and absolute entities (phenomena
	as such) which ground our human faculties. But here Husserl has solved the
	problems of noumena and intuition by the creation of a realm of suprasensible
	entities! He has, in effect, reinstated metaphysics. \autocite[4f]{stone2017}

	What about practical philosophy? For Husserl, ethics would appear as an objective
	science the same as any other. What is objectively good is what must
	necessarily appear good. Analogously to the postulation of sensory phenomena,
	evaluative phenomena are found in the \textit{Urregion} of consciousness insofar
	as they are postulated on top of volitional or emotional data. Ethics, then,
	takes on the shape of an objective science analogous to material phenomenology
	\autocite[5]{stone2017}.

	In overcoming metaphysics, Heidegger and Carnap are proceeding from, but
	overcoming, the phenomenological method which Husserl had laid out. Between
	them, there are a few shared concerns. Although they both wish to overcome
	metaphysics, I will focus on their stances on the nature of logic and value theory.
	In particular, I will show how for each author: (1.) Scientific thought is
	formal in nature and cannot capture values; (2.) Language is a key aspect of relating
	between the mind and the world; (3.) The manner in which we use language
	corresponds directly to our practical, political, and ethical aims.

	%-------------------------------------------------------------------------%
	\section{Rational Construction}

	For Carnap, the end of science is to \blockquote[{\citetitle[\S179]{carnap2003}}]{find and order the true statements about the objects of experience}.
	This can be split into two further goals: to create a constructional system
	corresponding to formal logic, and to investigate the relationships between objects
	of non-constructed experience. We then have two foundational aspects of
	construction theory: conventional stipulation of logical syntax and verification
	through empirical investigation.

	Carnap proceeds from unanalyzable phenomenological units which he calls basic experiences:
	these are whole-field conscious experiences which appear epistemically prior
	to the analysis of their constitutive parts. They are precisely the object of phenomenological
	analysis in Husserl's sense (\citetitle[\S64]{carnap2003}).\footnote{Carnap
	believed his system to be compatible with the three major epistemological
	movements: realism, idealism, and phenomenalism (\citetitle[\S177f]{carnap2003}).
	Nonetheless, he chose the Husserlian phenomena as his starting point.} However,
	basic experiences do not have any relation to a subject; the \enquote{for-me-ness}
	is analyzed into it after the fact. Similar to how we can only call integers
	\enquote{integers} in contradistinction to the real numbers once we have constructed
	them, we only understand our basic experiences as autopsychological in
	contrast with the later-constructed physical and heteropsychological objects. (\citetitle[{III.C \textit{passim.}}]{carnap2003}).

	Despite the inability to analyze the experiences themselves, he \enquote{quasi-analyzes}
	them by creating a formal allegory and examining the structures thereof. This
	amounts to a functionalization of objects, a switch from an ontological/material
	perspective to a structural/functional perspective. In this way he constructs
	the world through a purely formal analysis of basic experience. On this basis he
	claims to transform foundational epistemology from a speculative practice into
	a mathematically rigorous, logistical process.

	In the \textit{Aufbau,} Carnap aims to formulate a \enquote{one true language}
	from which to construct the world. In response to formal difficulties arising from
	the construction of the physical from the autopsychological, he is forced to
	reconsider his stance on logic, eventually realizing that there are many valid
	formal logics. Once we reach the \textit{Syntax,} this stipulatory nature is formalized
	into the principle of tolerance.
	%
	\begin{displayquote}
		[{\citetitle[\S17]{carnap2000}}] \emph{In logic, there are no morals.}
		Everyone is at liberty to build up his own logic, i.e. his own form of
		language, as he wishes. All that is required of him is that, if he wishes to
		discuss it, he must state his methods clearly, and give syntactial rules instead
		of philosophical arguments.
	\end{displayquote}
	%
	Since logistics is the ideal evaluative framework, but there are multiple valid
	logistical systems, it is up to us to find the ideal system to relate to our
	practical aims, whatever they may be.

	Analogous to the ontological and functional perception of gestalt phenomena,
	Carnap distinguishes between two modes of speech, the material and the formal.
	The distinction is defined by a translation between the formal-structural
	semantic statements of the kind discussed in the \textit{Syntax} and the material-ontological
	mode of speech which postulates properties of objects with little concern for structural
	soundness. Crucially, some material sentences, specifically those that propose
	spurious universal properties, cannot be translated into formal syntax. While
	Carnap claims that the material mode of speech is perfectly acceptable, and in
	fact unavoidable, he warns us that untranslatable material statements lead us
	to speak metaphysically. Metaphysical statements are a kind of expression, but
	they cannot be subject to claims about truth and falsity in the sense of
	logical-empirical validity, and so have no place in a scientific philosophy.
	\begin{displayquote}
		[{\citetitle[\S81]{carnap2000}}] \emph{Translatability into the formal mode
		of speech constitutes the touchstone for all philosophical sentences} \dots
		Sentences which do not give even a slight indication to determine their
		translation are outside the realm of the language of science and therefore
		incapable of discussion, no matter what depths or heights of feeling they may
		stir.
	\end{displayquote}
	So Carnap has his linguistic prescription: if you wish to be intelligible, you
	must speak in a manner consistent with the formal mode of speech.

	It is precisely this attitude which spells out the dangers of metaphysics. Metaphysical
	speech has the appearance of scientific rationality, while in fact only
	prescribing an orientation toward the world. This can lead to an unearned sense
	of authority for the individual. Carnap's argument against metaphysics - and
	especially against Heidegger - consists in this: speculative metaphysics is inherently
	authoritarian by lack of verifiably. In contrast, a social-scientific ethics undoubtedly
	places a large emphasis on the authority of the group and the sentiment of comradery
	which is essential for the scientific enterprise.

	Carnap claims in a 1932 lecture that ethical claims are in essence prescriptive
	statements, specifically imperatives. These statements are manifestations of an
	attitude toward life. Importantly, imperatives are \emph{not} translatable
	into formal language. We can see how easy it is to spin up spurious universals
	by analyzing ethical statements. \enquote{Do not kill,} for example, is
	clearly an imperative, while \enquote{killing is evil} is an assertion about a
	universal property which does not admit of empirical evidence \autocite[\S1.4]{carnap1996}.
	\enquote{Evil} is not a material thing, nor a property thereof. Carnap's own prescription,
	the scientific worldview, must then itself be an expression of an attitude toward
	life, and admits of no proper evidence other than a particular kind of faith
	in science.

	\begin{displayquote}
		[{\citetitle[xvii.f]{carnap2003}}] Whence then our confidence that our call for
		clarity, for a science that is free from metaphysics, will be heard? It
		stems from the knowledge or, to put it somewhat more carefully, from the
		belief that these opposing powers belong to the past. \dots Our work is carried
		by the faith that this attitude will win the future.
	\end{displayquote}

	The inability to formalize value statements leads Carnap directly into the famous
	argument against positivism, that the verification principle itself is a
	prescription, and cannot be verified. Any justification for a value must be
	made within the language of science, but the objects of its study are a matter
	of \emph{psychology} and not something which can be derived from universal
	principles of reason. Any prescriptive claim must take the shape of formal-logical
	arguments proceeding from \emph{volition}, ultimately amounting to an arbitrary
	(set?) of formal languages which capture the ethical feelings which express
	themselves as part of our basic experiences.

	In summary: (1.) Science is logistic in nature, proceeding from basic
	experiences to the construction of the world, while value postulation is a matter
	of irrational prescription, an animal reaction to chance situation. (2.) The material
	mode of speech is natural but insufficient, leading us into confusion. If one is
	to proceed scientifically, they must speak in a way which is translatable into
	the formal mode of speech. (3.) Material language has potentially deleterious psychological
	effects. The formal mode of speech most clearly aligns itself with a
	communitarian ethic, and specifically, the communist politic, while the material
	mode of speech aligns itself most easily with authoritarianism, specifically, the
	fascist politic.

	%-------------------------------------------------------------------------%
	\section{Beyond Logic}

	Heidegger, like Carnap, begins his analysis from phenomenological grounds. But
	his idea of a phenomenon is quite different. His system is very intricate, so we
	will only have the space to cover a small fraction of it here, mainly as it
	appears in \textit{Being and Time}.\footnote{References to Being and Time are
	to the original page numbers, prefixed with an H. in the cited edition.} We will
	focus only on what is necessary to contrast his understanding of scientific rationality
	from Carnap's.

	Heidegger's project is to study what it means for anything to be at all. This
	is the question of Being. For Heidegger, the question of Being is the proper
	aim of philosophy; it will take shape as a fundamental ontology. The goal of
	ontology is to set the \textit{a priori} foundations for the study of beings in
	general. In analogy with Carnap, Heidegger's approach is to analyze both the ontic,
	object-oriented sciences and the ontological, functional structure of consciousness
	in general. In doing so he and Carnap take on Husserl's goal of uniting the
	phenomena and the grounds of our rational faculties. However, in contrast to
	Carnap, in order to understand the object Heidegger believes we must understand
	it \emph{as} an object, i.e., as it appears. We cannot functionalize it, proceed
	by analogy, and consider this to be a full explanation. We must consider phenomena
	in its relation to our Being as a \emph{whole}, picking apart its various equiprimordial
	roots. In doing so we will come to a better understanding of the ambiguity
	inherent in the material mode of speech.

	We humans are exceptional in that our ontic existence is in fact ontological; that
	is, our Being is something we recognize, and which forms a problematic for us.
	This recognition makes us naturally capable of ontology. Because of this exceptional
	feature, it is through an analysis of \emph{our} Being, what Heidegger calls
	Dasein (lit. \enquote{being-there}) that we can come to an understanding of
	Being in general. Further, this analytic must occur as Dasein exists \enquote{proximally and for the most part,}
	in its everydayness. In so doing we will not only found the sciences, but all other
	meaningful orientation toward the world. (\citetitle[Int. I]{heidegger2008b})

	The method of this analysis is phenomenology. Though the term \enquote{phenomenology}
	implies a scientific approach, the form of \textit{l\'ogos} Heidegger has in
	mind is quite different from both the traditional syllogistic logic and the
	new mathematical logistics. What Heidegger means by \textit{l\'ogos} is closer
	to Hegel's dialectical logic, though he attains his definition through an etymological
	analysis of the original Greek terms. \textit{L\'ogos} is the manner in which a
	discussion allows the object of discourse to be seen (in Aristotle, \textit{apopha\'inesthai}).
	In this way, truth (\textit{al\'ethia}, lit. \enquote{not to escape notice}) becomes
	a manner of \enquote{disclosing} or \enquote{discovering} (\textit{pha\'inesthai},
	lit. \enquote{letting-be-seen}) the phenomenon, with falsity being a manner of
	\enquote{covering up} (\textit{pse\'udesthai}) to hide the \enquote{true} nature
	of the being as the object of discourse. So in this way, \textit{l\'ogos} as
	logistics cannot be the locus of truth. It is only in the phenomenon, or
	aesthetic experience, in which we find truth. \blockquote[{\citetitle[33]{heidegger2008b}}]
	{\textit{A\'isthesis}, %
	the sheer sensory perception of something, is \enquote{true} in the Greek sense, and indeed %
	more primordially than the \textit{l\'ogos} which we have been discussing}.
	\textit{A\'isthesis} cannot cover up, as it displays itself as itself (as \textit{noe\^in},
	cf Husserl's \textit{noema}); at worst it can be a kind of non-perceiving (\textit{agnoe\^in}).
	The object of \textit{a\'isthesis} is the \textit{\'idia} as, and for example,
	color is the object of sight. Though \textit{a\'isthesis} shows things as they
	are, covering-up occurs through synthesis, the showing of something as or
	through something else. \textit{A\'isthesis} does not occur through a faculty
	of cognition, but directly, thus avoiding the Kantian dilemmas. For Heidegger,
	rationality in the Kantian-scientific sense occurs only atop of apohpantic
	logos, which itself is derivative of aesthetic/phenomenal understanding. (\citetitle[Int.
	II.A, B]{heidegger2008b})

	So, the task of phenomenology is to (literally) discover the ideas (\textit{qua}
	principles) of aesthetic perception. So in contrast to Carnap, \enquote{for-me-ness}
	is an essential aspect of the phenomenon. It is clear then that the \textit{manner}
	in which things appear to us is of deep importance to Heidegger, and that the manner
	in which we express ourselves will be key to disclosing Being. Heidegger
	contends that our language has developed in accordance with the ontic
	perception. In order to overcome this, he requires us to rethink the manner in
	which we relate to, understand, and discuss the world.

	% To be clear, phenomenology cannot proceed in the form of propositional assertions
	% as in Carnap. This form of logic separates the essential character of Being
	% from its onto-historical context. By analyzing it in the abstract, the essence
	% \enquote{gets understood in an empty way} and \enquote{loses its indigenous character}
	% (\citetitle[36]{heidegger2008b}). In Carnap's system, we analyze basic
	% experiences which takes its character as free from any onto-historical or
	% personal identity. Any \enquote{for-me-ness} is analyzed \emph{into} basic experiences.
	% For Heidegger, \enquote{for-me-ness} is essential to the very character of
	% Dasein's Being. It is clear then that the \textit{manner} in which things
	% appear to us is of deep importance to Heidegger, and that the manner in which we
	% express ourselves will be key to disclosing Being.

	Things in the world can show themselves to us in different ways. Two of the
	most important orientations are readiness-to-hand and presence-to-hand. Ready-to-hand
	entities are visible in their tool-use: they make themselves apparent to us in
	their useful capacities toward some goal or another. But when the tool breaks,
	we stop to consider the entity as it is in itself, as a material thing with constitutive
	parts, etc. This viewpoint, when the entity is considered primarily \emph{as} entity,
	is presence-at-hand.

	Heidegger characterizes us as thrown into the world against our wills, made to
	orient ourselves. We do so through our state-of-mind, manifested as mood. Our
	moods turn us \emph{away} from ourselves and toward the world, thus revealing
	beings to us in their factical nature. Entities only appear to us as present-at-hand
	(in their utility) or otherwise because we are already predisposed by our moods
	to find them in this way (\citetitle[I.5 \P29]{heidegger2008b}). Alongside
	mood, we find the equally basic relation of understanding. The world discloses
	itself to us, and thus we to ourselves, in our possibilities, that is, in all
	the potential relations which they may hold. Potentiality is always future-oriented
	as something not yet actual, and never necessary. Understanding, then, as equiprimordial
	with state-of-mind, is a future-oriented disclosure of the potentiality of the
	world. This aspect of understanding is called projection. Our moods, as a turning-away,
	close off certain possibilities for us, thus restricting our sight. The manner
	in which projection works out its restricted possibilities is called
	interpretation, and the manner in which we then see something \emph{as}
	something is called articulation. But this potential-utility of the ready-to-hand
	object, for example, manifests itself through a \emph{totality} of
	involvements. Every potentiality of the ready-to-hand is such an involvement,
	and the ready-to-hand is only understood as relating to the entire mesh of
	these potential involvements.

	Since only what is understood can be articulated, all predication always already
	has the structure of understanding. As a matter of fixed predication, assertion
	is itself a kind of interpretation. But, since all understanding relates entities
	to Dasein and in so doing wraps up their being within Dasein's world, any
	\emph{meaning} as such will be discovered alongside the Being of Dasein. (\citetitle[\P31f]{heidegger2008b})

	So, Heidegger characterizes meaningful entities as just those entities which are
	understood in relation to our thrown projection, restricted in their
	possibilities (for utilization) by our state-of-mind. This interpretive structure
	is circular: we project meaningful possibility onto the ready-to-hand only
	because we already have a network of possibility structures in place by which
	to interpret them. Heidegger views this as a positive characteristic.
	\blockquote[{\citetitle[153]{heidegger2008b}}]{What is decisive is not to get out of the circle but to come into it in the right way. This circle of understanding is not an orbit in which any random kind of knowledge may move; it is the expression of the existential \emph{fore-structure} of Dasein itself. \dots In the circle is hidden a positive possibility of the most primordial kind of knowing.}

	Logistical analysis then becomes a kind of interpretation, a tearing-apart of
	what has been seen together in synthesis. He characterizes logic as operating on
	the basis of an ontic, \enquote{apophantical \enquote{as,}} in opposition to the
	\enquote{existential-hermeneutical \enquote{as}} of the ontological stance. In
	the ontic approach, \blockquote[{\citetitle[159]{heidegger2008b}}]{The judgement gets dissolved logistically into a system in which things are \enquote{co-ordinated} with one another; it becomes the object of a \enquote{calculus}; but it does not become a theme for ontological Interpretation}.
	That is to say, within the ontic mode of speech, we are unable to pursue an ontological
	discourse which can account for the socio-psychological aspects of our understanding.
	Apophantic logos misses the character of \enquote{thought} in Heidegger's sense,
	as it does not clear the space for an existential-hermeneutic analysis, and so
	is vastly limited in its possibilities.

	Now, finally, we come to characterize discourse or speech (\textit{Rede}) as
	equiprimordial with state-of-mind and understanding. As shown, meaning-making
	is a manner of interpretation, which takes the form of articulation. Meaning
	is that which is articulated. But what has been understood, and thus
	articulated, is always a totality of significations. The meaning of an
	utterance can be broken apart into separate significations, as in apophantic logos,
	and mapped to words. (\citetitle[161ff]{heidegger2008b})

	In order to communicate, we must be able both to speak and to listen.
	Listening takes the form of \emph{being quiet;} through this we open ourselves
	up to be \emph{with} others. But listening, that is, hearkening to the understanding
	which is to be communicated, can only occur because one is proximally
	alongside the Other. In this being-with, we find ourselves already with an
	understanding of the thing that any intelligible discourse is about. In order to
	be understood, one must already be in a shared state-of-mind (\citetitle[163ff]{heidegger2008b}).
	This is to say that understanding requires that one shares at least some portion
	of their \enquote{web of beliefs} with the other party, even if that understanding
	is as \enquote{minimal} as a shared natural language.

	Now, with all this mind we can understand Heidegger's call to destruct the Cartesian
	logical-empirical project. To Descartes, mathematical clarity, as an
	unchanging eternal ideal, is the arbiter of truth. To the extent that
	something truly \enquote{is,} its existence can be described in terms of mathematical
	rigor. But in order to do so, Descartes bases his ontology on the \textit{res
	extensa,} i.e. on Nature, an entity which to us is always proximally ready-to-hand.
	The mathematical mode of thought is apophantical, taking entities as they are
	present-to-hand, and reduced to their material-empirical properties, a deeply
	artificial state of mind. Descartes \blockquote[{\citetitle[96]{heidegger2008b}}]{prescribes for the world its \enquote{real} Being, as it were, on the basis of an idea of Being whose source has not been unveiled and which has not been demonstrated in its own right -- an idea in which Being is equated with constant presence-at-hand.}
	This second-hand knowledge - of material substance as constantly present-at-hand
	- forces us to avoid the question of value. As Carnap says, anything which goes
	beyond what can be empirically verified is nonsensical, i.e., cannot be
	formalized. This is exactly right, and it is exactly the issue.

	As in the \enquote{Question Concerning Technology,} \nocite{heidegger2008c}
	Heidegger claims that the empirical project, through modern technology, has
	dissociated humanity from its primordial nature. In addition to being a means to
	an end and a human activity, to Heidegger technology is a kind of revealing, an
	interpretive apparatus. Modern technology reveals the world as standing-reserve,
	available to us only in its possibility for consumption, i.e. translation into
	physical energy. Throughout his later work, as in \enquote{The Task of Thinking}\nocite{heidegger2008d}
	and \enquote{Only a God can Save Us,} Heidegger uses this argument in explicit
	opposition to socialism, which he calls \enquote{a form of planetary technicity}
	\autocite[206]{heidegger1981}. A Marxist might consider this strange: it is the
	capitalist class who exploits his fellow-man with technology in order to gain the
	excess wealth of his production. But Heidegger would respond: the worker and capitalist
	are both enframed, driven to view the world as standing-reserve irrespective
	of their class status. Socialism will not fix this problem because it does not
	understand the essence on which it is grounded.

	So, to conclude: (1.) Science is formal in nature, relying on an orientation
	which prefers ontic analysis over ontological. Values arise from existential considerations,
	which can be explicated only in terms of a hermeneutic analysis of Dasein. (2.)
	Our being-in-the-world, thus our understanding and our state-of-mind, is equiprimordial
	with and becomes intelligible through discourse, which reifies itself in
	language. (3.) The scientific worldview and its formal language lends itself to
	an ontic orientation which puts us at risk of dehumanization. Socialism and
	global capitalism, as technocratic regimes, reduce all Being into standing-reserve,
	potential for utilization toward an unexamined end.

	%=========================================================================%
	\section{Conclusion}

	The values which we hold can drastically impact the style and substance of our
	analyses. If we are prone to clarity and simplicity, a form of mathematical
	rigor can emerge. If we are prone to the ambiguous, a poetic form of speech
	more clearly fits our needs. Irrespective of how and why we value what we value,
	it is a key sociological finding that our social environments impact what we
	believe. Whether or not we intend them to, our values will find themselves
	leaking through to our philosophical discourse. If we are truly free to choose
	a logical system to suit our needs, then that choice will reflect pre-theoretical
	considerations. If we are to investigate what fuels those practical choices,
	it is essential to understand psychologically and existentially why we value
	what we do.

	Carnap unhelpfully throws his hands up at ethics, claiming that, because it is
	non-cognitive, we cannot say anything definite about its fundamental constitution,
	and thus ought to remain silent. Heidegger gives us an analytic which provides
	a method for grounding ethics, yet his understanding may lead us to
	obscurantism and an authoritarian worldview. If we are to overcome Heidegger's
	politics and Carnap's irrationalism, we must think \emph{with} these thinkers \emph{against}
	these thinkers. Perhaps we should take Carnap's penchant for semantic clarity and
	combine it with Heidegger's analytic of the non-cognitive, admitting that some
	ambiguity is necessary in the pursuit of transcendental knowledge. If we are to
	found ethics, we must think what cannot be said, and say it anyways.

	\centerline{\ding{166}} \break

	\printbibliography
\end{document}