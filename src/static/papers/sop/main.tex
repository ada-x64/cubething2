\documentclass{letter}

\signature{Phoenix Mandala}
\address{}

\newcommand{\BODY}[1]{ body goes here #1 }

\begin{document}
	\begin{letter}
		{CUNY}

		\opening{} \BODY \closing{All the best,}
	\end{letter}
\end{document}

% How do our emotions, social interactions, and embodied modes of cognition
% contribute to our understanding of the world? How is it that consciousness can arise
% from physical phenomena (if it does at all)? Can we explore and communicate
% these themes exclusively through language? I want to undertake the work of revealing
% unspoken assumptions and hidden labor, and of challenging hegemonic beliefs
% about the creation and dissemination of knowledge and wisdom. I want to investigate
% philosophy as a global tradition and as a practice. This is a wide-reaching
% enterprise, and it requires a wide-reaching study.

% I am primarily interested in phenomenology, comparative philosophy, philosophy of
% mind, and aesthetics. I am also interested in semiotics, critical theory, and value
% theory more generally. My interests converge in the study of embodied
% consciousness with influence from Buddhist conceptions of mind, and through the practice
% of interactive art.

% Currently, my career is as a software engineer. I am only able to pursue this because
% of the advances in symbolic logic and mathematics in the early 20th century, so
% naturally these are of great interest to me. Much of my undergraduate career was
% spent studying mathematics (my other major), logic, computer science, and the
% philosophy of mathematics. Studying the philosophy of natural science changed my
% perspective, giving me reason to doubt that empirical science is the one true
% method for investigating reality. In particular I am interested in the semantics
% and semiotics of science and mathematics – how do we translate from logic and empirical
% data to something meaningful? As Nancy Cartwright says, the truth doesn’t explain
% much.

% My math degree was focused on discrete mathematics, especially computational
% theory, number theory, and abstract algebra. In my senior year I took an independent
% study in the philosophy of mathematics which focused on intuitionism and the foundations
% of mathematics. At the same time, I wrote my senior thesis, which focused on the
% Heidegger-Carnap debate. I argued against Carnap for an interpretation of Heidegger’s
% das Nichts as an ontic basis for scientific reason and logical analysis.
% Additionally, I won the Paretsky Award for Creativity, a university-wide
% competition, for a paper applying the early Heidegger’s phenomenological method to
% the absent qualia problem.

% My move from a scientistic philosophy to one that attempts to reach beyond it did
% not come as a complete turn-around of my perspective. While I did believe in a
% purely empirical worldview then, I have also been interested in mindfulness practices
% and Eastern philosophy from a young age. I find Theravada and Zen Buddhism
% especially insightful on the nature of consciousness and selfhood. In particular
% I find the meontological convictions of the Kyoto School, and especially
% Nishitani, to be insightful. Phenomenology and existentialism helped lead me to
% them. I would like to improve my background in the history of phenomenology. In
% particular, I have recently been reading through Husserl’s Logical Investigations
% and I hope to study his works in further depth.

% In addition to the scientific aspirations of phenomenology, I find it is a good
% tool for the exploration of aesthetics and value theory quite generally. To this
% end I am interested in the work of Merleau-Ponty, Levinas, and critical theorists
% quite generally. No study on 20th century cultural theory could be complete
% without reference to Adorno, Benjamin, Foucault, and Deleuze: what would it mean
% to be authentic and live a meaningful life in an era of alienation? How do our lived
% experiences relate to the histories that surround us, and which we embody? I believe
% that the study of authenticity has a special role to play in understanding meaning
% in life, and this study can best be undertaken through an existentialist lens.

% In aesthetics I have been especially focused on existential ludology, and have studied
% C. Thi Nguyen, Ian Bogost, and Stefano Gualaeni. I am interested in understanding
% whether and how alternative media are capable of ‘doing’ philosophy. I believe
% that videogames hold a special potential. Games in general allow us to take on roles
% we would otherwise never encounter; they allow us to try out new experiences in
% an especially engrossing way. Videogames in particular expose some fundamental phenomenological
% problems, for example perception and representation. There are quite clear ties to
% semiotics through artistic symbolism, and I believe we could take lessons from
% experiential philosophies and apply them to the sciences as well.

% Buddhist philosophers believe that the true basis of consciousness, nibbana,
% cannot be spoken of, only discerned through unconditioned experience. As
% Heidegger says, the truth must show itself in the clearing. In the case of
% videogames, we can use their perceptual potency to guide our audience towards a conclusion,
% to psychologically prime them for the experiential argument we hope to make. If there
% is much that words “cannot speak,” we need to reach beyond them, “to the things
% themselves” – and further, beyond things; if what lies beyond things is
% “nothing” then we must ask – “how is it with the Nothing?”

% To conclude, I hope to spend my time doing a close study of the phenomenological
% method as a foundation of philosophical and scientific theory. In addition to the
% traditional forms of essay-writing and pedagogy, I hope to explore alternative
% media as a form of philosophical expression.

% Thank you for taking the time to read my materials, and I hope to meet you soon.