% FROM https://www.overleaf.com/latex/templates/rendercv-sb2nov-theme/gdspgtsnfncm

\documentclass[10pt, letterpaper]{article}

% Packages:
\usepackage[
	ignoreheadfoot, % set margins without considering header and footer
	top=2 cm, % seperation between body and page edge from the top
	bottom=2 cm, % seperation between body and page edge from the bottom
	left=2 cm, % seperation between body and page edge from the left
	right=2 cm, % seperation between body and page edge from the right
	footskip=1.0 cm, % seperation between body and footer
	% showframe % for debugging
]{geometry} % for adjusting page geometry
\usepackage{titlesec} % for customizing section titles
\usepackage{tabularx} % for making tables with fixed width columns
\usepackage{array} % tabularx requires this
\usepackage[dvipsnames]{xcolor} % for coloring text
\definecolor{primaryColor}{RGB}{0, 79, 144} % define primary color
\usepackage{enumitem} % for customizing lists
\usepackage{fontawesome5} % for using icons
\usepackage{amsmath} % for math
\usepackage[
	pdftitle={Phoenix Mandala Resume},
	pdfauthor={Phoenix Mandala},
	pdfcreator={LaTeX with RenderCV},
	colorlinks=true,
	urlcolor=primaryColor
]{hyperref} % for links, metadata and bookmarks
\usepackage[pscoord]{eso-pic} % for floating text on the page
\usepackage{calc} % for calculating lengths
\usepackage{bookmark} % for bookmarks
\usepackage{lastpage} % for getting the total number of pages
\usepackage{changepage} % for one column entries (adjustwidth environment)
\usepackage{paracol} % for two and three column entries
\usepackage{ifthen} % for conditional statements
\usepackage{needspace} % for avoiding page brake right after the section title
\usepackage{iftex} % check if engine is pdflatex, xetex or luatex
\usepackage{dirtytalk}
\usepackage{datetime}

% Ensure that generate pdf is machine readable/ATS parsable:
\ifPDFTeX
\input{glyphtounicode}
\pdfgentounicode=1
% \usepackage[T1]{fontenc} % this breaks sb2nov
\usepackage[utf8]{inputenc}
\usepackage{lmodern}
\fi

% Some settings:
% \AtBeginEnvironment{adjustwidth}{\partopsep0pt} % remove space before adjustwidth environment
\pagestyle{empty} % no header or footer
\setcounter{secnumdepth}{0} % no section numbering
% \setlength{\parindent}{0pt} % no indentation
% \setlength{\topskip}{0pt} % no top skip
% \setlength{\columnsep}{0cm} % set column seperation
\makeatletter
\let\ps@customFooterStyle\ps@plain % Copy the plain style to customFooterStyle
\patchcmd{\ps@customFooterStyle}{\thepage}{ \color{gray}\textit{\small Phoenix Mandala - Page \thepage{} of \pageref*{LastPage}} }{}{} % replace number by desired string
\makeatother
\pagestyle{customFooterStyle}

\titleformat{\section}{\needspace{4\baselineskip}\bfseries\large}{}{0pt}{}[
\vspace{1pt}
\titlerule]

\titlespacing{\section}{
% left space:
-1pt }{
% top space:
0.3 cm }{
% bottom space:
0.2 cm } % section title spacing

\renewcommand{\labelitemi}{$\circ$} % custom bullet points
\newenvironment{highlights}{ \begin{itemize}[ topsep=0.10 cm, parsep=0.10 cm, partopsep=0pt,
itemsep=0pt, leftmargin=0.4 cm + 10pt ] }{ \end{itemize} } % new environment for highlights

\newenvironment{highlightsforbulletentries}{ \begin{itemize}[ topsep=0.10 cm,
parsep=0.10 cm, partopsep=0pt, itemsep=0pt, leftmargin=10pt ] }{ \end{itemize} } % new environment for highlights for bullet entries

\newenvironment{onecolentry}{ \begin{adjustwidth}{ 0.2 cm + 0.00001 cm }{ 0.2 cm + 0.00001 cm }
}{ \end{adjustwidth}} % new environment for one column entries

\newenvironment{twocolentry}[2][]{ \onecolentry \def\secondColumn{#2} \setcolumnwidth{\fill, 10 cm}
\begin{paracol}{2} }{ \switchcolumn \raggedleft \secondColumn \end{paracol}
\endonecolentry } % new environment for two column entries

\newenvironment{header}{
\setlength{\topsep}{0pt}
\par\kern\topsep
\centering
\linespread{1.5} }{ \par\kern\topsep } % new environment for the header

\newdateformat{monthyeardate}{%
\monthname[\THEMONTH], \THEYEAR}
\newcommand{\placelastupdatedtext}{% \placetextbox{<horizontal pos>}{<vertical pos>}{<stuff>}
\AddToShipoutPictureFG*{% Add <stuff> to current page foreground
\put( \LenToUnit{\paperwidth-2 cm-0.2 cm+0.05cm}, \LenToUnit{\paperheight-1.0 cm} ){\vtop{{\null}\makebox[0pt][c]{ \small\color{gray}\textit{Last updated in \monthyeardate\today}\hspace{\widthof{Last updated in \monthyeardate\today}} }}}%
}%
}%

% save the original href command in a new command:
\let\hrefWithoutArrow\href

% new command for external links:
\renewcommand{\href}[2]{\hrefWithoutArrow{#1}{\ifthenelse{\equal{#2}{}}{ }{#2 }\raisebox{.15ex}{\footnotesize \faExternalLink*}}}

\usepackage{setspace}

\begin{document}
	\newcommand{\AND}{\unskip \cleaders\copy\ANDbox\hskip\wd\ANDbox \ignorespaces }
	\newsavebox{\ANDbox}
	\sbox{\ANDbox}{}

	\placelastupdatedtext
	\begin{header}
		\textbf{\fontsize{24 pt}{24 pt}\selectfont Phoenix Ada Rose Mandala}

		\vspace{0.3 cm}

		\normalsize
		\mbox{{\color{black}\footnotesize\faMapMarker*}\hspace*{0.13cm}Lawrence, KS}%
		\kern 0.25 cm%
		\AND%
		\kern 0.25 cm%
		\mbox{\hrefWithoutArrow{mailto:ada.mandala@pm.me}{\color{black}{\footnotesize\faEnvelope[regular]}\hspace*{0.13cm}ada.mandala@pm.me}}%
		\kern 0.25 cm%
		\AND%
		\kern 0.25 cm%
		\mbox{\hrefWithoutArrow{https://cubething.dev/}{\color{black}{\footnotesize\faLink}\hspace*{0.13cm}cubething.dev}}%
		\kern 0.25 cm%
		\AND%
		\kern 0.25 cm%
		\mbox{\hrefWithoutArrow{https://linkedin.com/in/ada-mandala}{\color{black}{\footnotesize\faLinkedinIn}\hspace*{0.13cm}ada-mandala}}%
		\kern 0.25 cm%
		\AND%
		\kern 0.25 cm%
		\mbox{\hrefWithoutArrow{https://github.com/ada-x64}{\color{black}{\footnotesize\faGithub}\hspace*{0.13cm}ada-x64}}%
	\end{header}

	\begin{doublespace}
		\noindent
		\textbf{thatgamecompany,}

		I am excited to inform of you of my application to your Build Engineer role.
		I hope this letter finds you well.

		Firstly, I want to emphasize my long-standing desire to work in the arts,
		and to pursue liberation and personal change through creative endeavors. After
		experiencing the atmosphere of Sky, Journey, and Flower, feeling the awesome
		power of open spaces, and the curious, playful joy of limited anonymous
		interaction, I believe thatgamecompany's vision of creating empathetic and meditative
		social experiences is well in line with that goal.

		Back in the workplace, I thrive in collaborative environments that practice
		shared ownership and which engage in challenging software work. In prior roles
		I have shared responsibilities across all aspects of the tech stack, from
		front-end in React and full-stack Rust to container-based DevOps with AWS
		and Azure. Additionally, I am strong in test-driven development with
		Playwright, Jest, pytest, and Rust. I have collaborated on all aspects of
		business, including designing and developing client platforms and utilizing Agile
		best practices such as Kanban task tracking, twice-daily stand-ups, and daily
		code pair sessions. At Prospective, Valorem Reply/Disney, and in my
		freelancing career I have been on many client-facing calls, and have successfully
		translated customer requirements into tangible results.

		A good tool is invisible, and a bad tool is like a broken arm. I have strived
		for good DevEx throughout my career, contributing to the build process of
		various open-source and commercial codebases. Some projects of note are \href{https://github.com/ada-x64/testy}{my
		work for OpenSesame's e-learning platform} (a solo project), the development
		of ESPN+ on the PS5, including work on automated dev kit deployments, and my
		work on \href{https://github.com/finos/perspective}{Perspective}, a WASM-based,
		in-browser data engine built with CMake. At Prospective, our enterprise product
		was built with \href{https://bazel.build}{Bazel}, and handled complex and
		interoperating C++, Rust, Python, and TypeScript builds. I am knowledgeable
		of the intricacies of compilation, linking, and archiving executables on
		Linux and Windows as well as the development, testing, continuous
		integration and hosting of websites.

		In my free time I have developed a
		\href{https://github.com/ada-x64/sundile-rs}{custom WGPU rendering engine}, and
		I am currently working on a \href{https://github.com/ada-x64/qproj}{game
		project} in \href{https://bevyengine.org}{bevy}. Since I am working on WSL2
		and
		\href{https://github.com/ada-x64/qproj/pull/3}{importing C++ dependencies}, this
		requires a custom build process. I have written my build scripts
		\href{https://github.com/ada-x64/qproj/pull/5}{in python} and
		\href{https://github.com/ada-x64/qproj/pull/4}{tailored my build process for
		remote execution}
		so that I can build and run the game from my laptop while using my desktop's
		powerful resources. This required creating a systemd service on my raspberrypi,
		setting up an SSH proxy jump, and a simple secure static and websocket server
		to communicate between my devices. Additionally, I have \href{https://github.com/ada-x64/qproj/actions}{PR
		checks set up through GH Actions,} and plan on adding tests and package distribution
		once development proceeds far enough. Moreover, I have experience publishing
		to DigitalOcean (an AWS-based service) through my \href{https://github.com/ada-x64/cubething2}{personal
		website} (and a custom Minecraft server) and I am Azure certified.

		My background in philosophy and mathematics has equipped me with strong analytical
		skills, a knack for tutoring, and a first-principles mindset that helps me navigate
		and explain complex concepts. This foundation, combined with my self-taught engineering
		journey, demonstrates my commitment to continuous learning and growth —
		qualities I would proudly bring to thatgamecompany's engineering culture.

		Thank you for taking the time to review my application materials. I hope to hear
		from you soon!

		\noindent
		\textbf{All the best,}

		\noindent
		\textit{- Phoenix Mandala (they/them)}
	\end{doublespace}
\end{document}