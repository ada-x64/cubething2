% FROM https://www.overleaf.com/latex/templates/rendercv-sb2nov-theme/gdspgtsnfncm

\documentclass[10pt, letterpaper]{article}

% Packages:
\usepackage[
	ignoreheadfoot, % set margins without considering header and footer
	top=2 cm, % seperation between body and page edge from the top
	bottom=2 cm, % seperation between body and page edge from the bottom
	left=2 cm, % seperation between body and page edge from the left
	right=2 cm, % seperation between body and page edge from the right
	footskip=1.0 cm, % seperation between body and footer
	% showframe % for debugging
]{geometry} % for adjusting page geometry
\usepackage{titlesec} % for customizing section titles
\usepackage{tabularx} % for making tables with fixed width columns
\usepackage{array} % tabularx requires this
\usepackage[dvipsnames]{xcolor} % for coloring text
\definecolor{primaryColor}{RGB}{0, 79, 144} % define primary color
\usepackage{enumitem} % for customizing lists
\usepackage{fontawesome5} % for using icons
\usepackage{amsmath} % for math
\usepackage[
	pdftitle={Phoenix Mandala Resume},
	pdfauthor={Phoenix Mandala},
	pdfcreator={LaTeX with RenderCV},
	colorlinks=true,
	urlcolor=primaryColor
]{hyperref} % for links, metadata and bookmarks
\usepackage[pscoord]{eso-pic} % for floating text on the page
\usepackage{calc} % for calculating lengths
\usepackage{bookmark} % for bookmarks
\usepackage{lastpage} % for getting the total number of pages
\usepackage{changepage} % for one column entries (adjustwidth environment)
\usepackage{paracol} % for two and three column entries
\usepackage{ifthen} % for conditional statements
\usepackage{needspace} % for avoiding page brake right after the section title
\usepackage{iftex} % check if engine is pdflatex, xetex or luatex
\usepackage{dirtytalk}
\usepackage{datetime}

% Ensure that generate pdf is machine readable/ATS parsable:
\ifPDFTeX
\input{glyphtounicode}
\pdfgentounicode=1
% \usepackage[T1]{fontenc} % this breaks sb2nov
\usepackage[utf8]{inputenc}
\usepackage{lmodern}
\fi

% Some settings:
\AtBeginEnvironment{adjustwidth}{\partopsep0pt} % remove space before adjustwidth environment
\pagestyle{empty} % no header or footer
\setcounter{secnumdepth}{0} % no section numbering
\setlength{\parindent}{0pt} % no indentation
\setlength{\topskip}{0pt} % no top skip
\setlength{\columnsep}{0cm} % set column seperation
\makeatletter
\let\ps@customFooterStyle\ps@plain % Copy the plain style to customFooterStyle
\patchcmd{\ps@customFooterStyle}{\thepage}{ \color{gray}\textit{\small Phoenix Mandala - Page \thepage{} of \pageref*{LastPage}} }{}{} % replace number by desired string
\makeatother
\pagestyle{customFooterStyle}

\titleformat{\section}{\needspace{4\baselineskip}\bfseries\large}{}{0pt}{}[
\vspace{1pt}
\titlerule]

\titlespacing{\section}{
% left space:
-1pt }{
% top space:
0.3 cm }{
% bottom space:
0.2 cm } % section title spacing

\renewcommand{\labelitemi}{$\circ$} % custom bullet points
\newenvironment{highlights}{ \begin{itemize}[ topsep=0.10 cm, parsep=0.10 cm, partopsep=0pt,
itemsep=0pt, leftmargin=0.4 cm + 10pt ] }{ \end{itemize} } % new environment for highlights

\newenvironment{highlightsforbulletentries}{ \begin{itemize}[ topsep=0.10 cm,
parsep=0.10 cm, partopsep=0pt, itemsep=0pt, leftmargin=10pt ] }{ \end{itemize} } % new environment for highlights for bullet entries

\newenvironment{onecolentry}{ \begin{adjustwidth}{ 0.2 cm + 0.00001 cm }{ 0.2 cm + 0.00001 cm }
}{ \end{adjustwidth}} % new environment for one column entries

\newenvironment{twocolentry}[2][]{ \onecolentry \def\secondColumn{#2} \setcolumnwidth{\fill, 10 cm}
\begin{paracol}{2} }{ \switchcolumn \raggedleft \secondColumn \end{paracol}
\endonecolentry } % new environment for two column entries

\newenvironment{header}{
\setlength{\topsep}{0pt}
\par\kern\topsep
\centering
\linespread{1.5} }{ \par\kern\topsep } % new environment for the header

\newdateformat{monthyeardate}{%
\monthname[\THEMONTH], \THEYEAR}
\newcommand{\placelastupdatedtext}{% \placetextbox{<horizontal pos>}{<vertical pos>}{<stuff>}
\AddToShipoutPictureFG*{% Add <stuff> to current page foreground
\put( \LenToUnit{\paperwidth-2 cm-0.2 cm+0.05cm}, \LenToUnit{\paperheight-1.0 cm} ){\vtop{{\null}\makebox[0pt][c]{ \small\color{gray}\textit{Last updated in \monthyeardate\today}\hspace{\widthof{Last updated in \monthyeardate\today}} }}}%
}%
}%

% save the original href command in a new command:
\let\hrefWithoutArrow\href

% new command for external links:
\renewcommand{\href}[2]{\hrefWithoutArrow{#1}{\ifthenelse{\equal{#2}{}}{ }{#2 }\raisebox{.15ex}{\footnotesize \faExternalLink*}}}

\begin{document}
	\newcommand{\AND}{\unskip \cleaders\copy\ANDbox\hskip\wd\ANDbox \ignorespaces }
	\newsavebox{\ANDbox}
	\sbox{\ANDbox}{}

	\placelastupdatedtext
	\begin{header}
		\textbf{\fontsize{24 pt}{24 pt}\selectfont Phoenix Ada Rose Mandala}

		\vspace{0.3 cm}

		\normalsize
		\mbox{{\color{black}\footnotesize\faMapMarker*}\hspace*{0.13cm}Lawrence, KS}%
		\kern 0.25 cm%
		\AND%
		\kern 0.25 cm%
		\mbox{\hrefWithoutArrow{mailto:ada.mandala@pm.me}{\color{black}{\footnotesize\faEnvelope[regular]}\hspace*{0.13cm}ada.mandala@pm.me}}%
		\kern 0.25 cm%
		\AND%
		\kern 0.25 cm%
		\mbox{\hrefWithoutArrow{https://cubething.dev/}{\color{black}{\footnotesize\faLink}\hspace*{0.13cm}cubething.dev}}%
		\kern 0.25 cm%
		\AND%
		\kern 0.25 cm%
		\mbox{\hrefWithoutArrow{https://linkedin.com/in/ada-mandala}{\color{black}{\footnotesize\faLinkedinIn}\hspace*{0.13cm}ada-mandala}}%
		\kern 0.25 cm%
		\AND%
		\kern 0.25 cm%
		\mbox{\hrefWithoutArrow{https://github.com/ada-x64}{\color{black}{\footnotesize\faGithub}\hspace*{0.13cm}ada-x64}}%
	\end{header}

	% \vspace{0.3 cm - 0.3 cm}
	% \section{Quick Guide}

	% \begin{onecolentry}
	% 	\begin{highlightsforbulletentries}
	% 		\item Each section title is arbitrary and each section contains a list of entries.

	% 		\item There are 7 unique entry types: \textit{BulletEntry}, \textit{TextEntry},
	% 		\textit{EducationEntry}, \textit{ExperienceEntry}, \textit{NormalEntry},
	% 		\textit{PublicationEntry}, and \textit{OneLineEntry}.

	% 		\item Select a section title, pick an entry type, and start writing your section!

	% 		\item \href{https://docs.rendercv.com/user_guide/}{Here}, you can find a
	% 		comprehensive user guide for RenderCV.
	% 	\end{highlightsforbulletentries}
	% \end{onecolentry}

	%%%%%%%%%%%%%%%%%%%%%%%%%%%%%%%%%%%%%%%%%%%%%%%
	%%                EXPERIENCE                 %%
	%%%%%%%%%%%%%%%%%%%%%%%%%%%%%%%%%%%%%%%%%%%%%%%
	\begin{samepage}
		\section{Experience}

		%{employer}{dates}{title}{skills}
		\newenvironment{experienceSection}[4]{
		\begin{twocolentry}
			{ \textbf{#4}

			\textit{#2}} \textbf{#3}

			\textit{#1}
		\end{twocolentry}

		\vspace{0.10 cm}
		\begin{onecolentry} \begin{highlights} }{ \end{highlights} \end{onecolentry}
		\vspace{0.2cm}
		}

		%% A.Team
		\begin{experienceSection}
			{A.Team Contractor}{Jan 2025 -- Current}{Software Engineer}{Typescript, Rust/WASM, Graphics APIs}
			\item Selected to join an exclusive software development contracting
			platform for senior engineers. \item Developed various software projects,
			ranging from full-stack greenfield solo projects to large-scale team
			enterprises.
		\end{experienceSection}

		%% Prospective
		\begin{experienceSection}
			{Prospective}{June 2023 -- June 2024}{Software Engineer}{Rust, WASM, Typescript, Python, CI/CD}
			\item Designed and developed the Prospective data visualization dashboard
			using the Perspective library, integrating with AWS, Azure, Google Cloud,
			and DigitalOcean.

			\item Created the plugin API and enhanced visual features; collaborated on
			JupyterLab extension, C++ data engine rewrite, and integrations with Kafka
			and Kerberos.
		\end{experienceSection}

		%% Valorem Reply / Disney
		%% Prospective
		\begin{experienceSection}
			{Valorem Reply / Disney}{June 2022 -- May 2023}{Software Engineer}{Rust, WASM, Python, Embedded}
			\item Contracted with Valorem Reply to bring ESPN+ to PlayStation 5 using a
			custom Rust framework.

			\item Developed infrastructure, implemented concurrent algorithms, and simplified
			analytics with a custom macro library; credited with \say{saving the project.}
		\end{experienceSection}

		%% Roll20
		\begin{experienceSection}
			{Roll20}{October 2019 -- August 2021}{Front-End Engineer}{Javscript, SCSS, HTML5, NodeJS}
			\item Developed front-end for user-driven IP BurnBryte, utilizing the Charactermancer
			feature and providing feedback to improve the API and developer experience.

			\item Created a local development environment mimicking the Roll20 frontend,
			boosting team productivity.
		\end{experienceSection}

		%% Freelance
		\begin{experienceSection}
			{Freelance}{June 2018 -- July 2022}{Software Engineer}{Javascript, PHP, C\#, Angular, React}
			\item Designed, developed, and integrated full-stack solutions to meet
			client expectations.

			\item Worked with clients on short-term contracts in various industries
			including game development, healthcare, and business applications.

			\item Built green and brownfield projects on web, mobile, and desktop.
		\end{experienceSection}
	\end{samepage}
	% \section{Publications}

	% \begin{samepage}
	% 	\begin{twocolentry}
	% 		{ Jan 2004 } \textbf{3D Finite Element Analysis of No-Insulation Coils}

	% 		\vspace{0.10 cm}

	% 		\mbox{Frodo Baggins}, \mbox{\textbf{\textit{John Doe}}},
	% 		\mbox{Samwise Gamgee}
	% 	\end{twocolentry}

	% 	\vspace{0.10 cm}

	% 	\begin{onecolentry}
	% 		\href{https://doi.org/10.1109/TASC.2023.3340648}{10.1109/TASC.2023.3340648}
	% 	\end{onecolentry}
	% \end{samepage}

	%%%%%%%%%%%%%%%%%%%%%%%%%%%%%%%%%%%%%%%%%%%%%%%
	%%                 PROJECTS                  %%
	%%%%%%%%%%%%%%%%%%%%%%%%%%%%%%%%%%%%%%%%%%%%%%%

	\begin{samepage}
		\section{Projects}

		% {name}{skills}{github link?}{demo href?}
		\newenvironment{project}[4]{
		\begin{twocolentry}
			{%
			\textbf{#2}

			\textit{#3}%
			}%
			\textbf{#1}

			\textit{#4}
		\end{twocolentry}
		\vspace{0.10 cm}
		\begin{onecolentry} \begin{highlights} }{ \end{highlights} \end{onecolentry}

		\vspace{0.2 cm}
		}

		%% Sundile
		\begin{project}
			{Custom Web Game Engine}%
			{Rust, WASM, WebGPU}%
			{\hrefWithoutArrow{https://github.com/ada-x64/sundile_rs}{ada-x64/sundile\_rs \faGithub}}
			{}%
			\item Developed a Rust-based graphics engine with WGPU for local and
			browser rendering. Utilizes WASM.

			\item Features include perspective cameras, .obj model rendering, and
			Blinn-Phong lighting.

			\item Implemented a custom asset engine for dynamic asset loading; project
			built from scratch for learning purposes.
		\end{project}
	\end{samepage}

	%%%%%%%%%%%%%%%%%%%%%%%%%%%%%%%%%%%%%%%%%%%%%%%
	%%                 EDUCATION                 %%
	%%%%%%%%%%%%%%%%%%%%%%%%%%%%%%%%%%%%%%%%%%%%%%%
	\begin{samepage}
		\section{Education}

		\begin{twocolentry}
			{ \textbf{GPA: 3.8}

			\textit{Aug 2014 -- Dec 2017}} \textbf{University of Kansas}

			\textit{BA Philosophy (Hnrs.), BA Mathematics}
		\end{twocolentry}

		\vspace{0.10 cm}
		\begin{onecolentry}
			\begin{highlights}
				\item \textbf{Coursework:} Computational Theory, Philosophy of Mathematics,
				Formal Logic, Senior Essay
			\end{highlights}
		\end{onecolentry}
	\end{samepage}
\end{document}