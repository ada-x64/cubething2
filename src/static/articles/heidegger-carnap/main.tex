\documentclass[leqno, 12pt]{turabian-researchpaper}
\usepackage{cubething}
\addbibresource{library.bib}

\usepackage{pifont}

\renewcommand*{\bibfont}{\small}

%opening
\title{Technology and Disenchantment in Heidegger and Carnap}
\author{Phoenix Ada Rose Mandala}

\begin{document}
	\maketitle

	The famous contestation between Heidegger and Carnap is often seen as an
	exemplary early example of the divide between the analytic and continental
	traditions in philosophy. Recent research has indicated the increasing importance
	of understanding the political and sociocultural backgrounds of our two thinkers
	in order to make sense of their positions. This paper will discuss the shared
	background and show how it expresses itself in terms of our thinkers' metaethical
	philosophies, and how their metaethics is shaped by their epistemological
	projects and conceptions of language. I will argue that Heidegger's theory of dwelling
	provides a better basis for metatheoretical discussion than Carnap's embrasure
	of scientific-technological thinking.

	%=========================================================================%
	\section{Technology and Disenchantment}

	% Section overview.
	\autocite{dambock2022} argues that Carnap and Heidegger follow Weber in maintaining
	an existentialist ethical decisionism. Carnap's preference for scientific
	rationality results in a non-cognitivism, essentially shifting the ethical into
	a psychological field of study. Heidegger will argue that, while scientific (cognitive)
	reasoning cannot capture values, we still have a method for determining which
	values to choose through resolute action toward an archetypal ideal. Damb\"ock
	claims that Heidegger, in moving away from the Enlightenment value of
	instrumental reasoning, wants to remove science from philosophy altogether.
	While I agree with Damb\"ock on Carnap, I will argue his understanding of Heidegger
	is na\"ive in this respect.

	% Definition of disenchantment and the shared Weberian background.
	Carnap and Heidegger were young men in 1914 when the world went to war. The
	horror of industrial warfare and the decline of Western religious practices
	prompted a youth movement whose primary concern was the overcoming of
	Enlightenment sensibility and the disenchantment that came with it. In 1917,
	Max Weber addressed the following to students of the youth movement:
	\begin{displayquote}
		[{\cite[p139]{weber2014}}] \dots what is the meaning of science as a vocation,
		now after all these former illusions \dots have been dispelled? Tolstoy has given
		the simplest answer, with the words: \enquote{Science is meaningless because it gives %
		no answer to our question, the only question important for us: \enquote{%
		What shall we do and how shall we live?}}
		That science does not give an answer to this is indisputable. The only question
		that remains is the sense in which science gives \enquote{no} answer, and whether
		or not science might yet be of some use to the one who puts the question correctly.
	\end{displayquote}
	So in disenchantment the classic dilemma of is and ought had come to characterize
	modernity: What are we to do about the incommensurability of empirical-scientific
	facts, as embodied by technology, and theological-existential meaning, as embodied
	by faith? This split was felt politically as well, with the scientifically-oriented
	drawn to socialism and the theologically concerned drawn to fascism.\footnote{This
	movement to overcome modernity may alternately be described as a Nietzschean
	\enquote{revaluation of all values.} Heidegger's involvement with Nietzsche is
	well-known. Carnap's interest in Nietzsche is less obvious, but he is known to
	have read and thought rather highly of him. See \autocite{sachs2011a},
	\autocite[n22]{stone2017} for Nietzschean interpretations of Carnap's faith in
	science.}

	% Overview of logical positivism, summary of the Uberwindung.
	For Carnap and his Vienna Circle, the limitations of science are themselves the
	limitations of knowledge. Taking physics as their model, they believed that anything
	which can be said about material things can and must be stated in logical-empirical
	language. The guiding star of their approach was the verification principle: Any
	meaningful statement must be subject to logical-empirical verification. If a
	statement fails to meet this criterion, it is considered \enquote{nonsense} (\textit{Sinnloss},
	lit. without sense). Metaphysical statements are inherently unverifiable, so the
	positivists claim, and so are not subject to proper scientific-philosophical
	discourse. Instead, they are considered to have only expressive meaning. This attitude
	is not to be taken as strictly austere, however, as the scientific attitude is
	meant to serve life.\footnote{This paragraph summarizes the primary argument
	used against Heidegger in Carnap's famous \citetitle{carnap1966}\autocite{carnap1966}.
	Although Carnap's logical arguments fall flat on a modern reading, the argument's
	focus on the verification principle and the notion of metaphysics as nonsense remains.}
	It is from an existential and political concern with the well-being of the common
	man that the Vienna Circle proclaims their attitude. To them, metaphysics not only
	poses a threat to science, but to organized human life.

	% Link from positivism to socialism and technicity as the solution to disenchantment.
	In \enquote{The Scientific Conception of the World} \autocite{hahn1973}, the Circle
	explicitly link their worldview with contemporaneous socialist movements. They
	acknowledge a widespread disenchantment among Weberian lines, connected with the
	industrial mode of production. They claim that \blockquote[{\cite[p21f]{hahn1973}}]{The European public with their socialist attitudes tend to lean towards a down-to-earth empricist view. In previous times \emph{materialism} was the expression of this view; meanwhile, however, modern empiricism has ... taken a strong shape in the \emph{scientific world-conception}}.
	Their solution to disenchantment can be found in the hope they place in science
	and collectivist movements taking shape throughout the continent. If
	disenchantment is defined as the crisis emerging from the loss of theological
	and royal authority, then the Circle's solution is to \emph{lean in,} to root
	out the old and make room for the new.

	% Acknowledgement of Heidegger's politiccs, decision to focus on a critique of socialism.
	It is well-known that Heidegger took the rectorate at Freiburg under Nazi
	reign and delivered lectures on the \enquote{inner greatness} of the Nazi party,
	but that he was later taciturn about his involvements in the regime. While Heidegger's
	fascism and antisemitism are subjects of much debate and are themselves worthy
	of discussion, we will leave aside any positive account of his politics in
	order to focus on his critique of socialism.

	% Definition of technicity, link between technicity and socialism.
	Heidegger argues that communism and \enquote{Americanism} (what I assume to be
	liberal-democratic industrial capitalism) are forms of \enquote{planetary technity}
	\autocite[p206]{heidegger1981}. This critique is not a row against science or technology
	in the sense of tool-use or objectivity. Rather, it is an expression of disenchantment
	which takes the form of a desire for a secularized sense of divinity. Modern
	technology reveals itself as an enframing (\textit{Ge-stell}) and challenges mankind
	to see all beings as essentially physical. Instead of allowing the Being of
	beings to speak to us through \textit{poesis}, the enframing aspect of
	technology forces us to view all being as potentially technological. This ultimately
	leads to a technological conception of mankind itself, as expressed in the
	science of cybernetics \autocite{heidegger2008c}. It is exactly the scientific
	world-view which Heidegger is opposing, and which Carnap represents.

	% Heidegger's metaethic.
	In \textit{Being and Time,} Heidegger argues for a metaethic of resolute
	decision. In essence, he will argue that what founds ethics is our ability to
	choose ourselves, to make an authentic decision about who we want to be within
	the society we live in. His metaethic is then both highly personal while
	allowing for a cultural ethic to emerge. The later Heidegger, especially in
	\enquote{Building Dwelling Thinking} \autocite{heidegger2008e}, will argue for
	a metaethic based on \enquote{dwelling within the fourfold.} This is a
	foundational relation to the metaphorical entiteis of earth, sky, divinities, and
	mortals. In dwelling, Being reveals itself to us as Being; the question of
	Being becomes possible to pose. I read dwelling as a being-alongside beings in
	the world, to the extent that we allow Being to show itself in the fourfold. I
	will this conception in more detail later, but what is important for our
	overview is that the scientific worldview enframes us in such a way that the
	question of being becomes impossible for us. We are no longer able to dwell,
	and in this way we lose the safeguarding of the fourfold, and so we lose the potentiality
	for being as we truly are.

	% Section conclusion and segue to next.
	As shown, thinkers are wrapped up in an encounter with disenchantment, and both
	are looking for metaethical solutions to the social problems it causes. But
	their methods differ as dramatically as their political goals. Carnap wants to
	bring the Enlightenment project to a conclusion with the total empiricization of
	philosophy. He does so in order to clear space for a collectivist world free
	from metaphysics. Heidegger, however, wants to destruct this method and worldview
	in order to open up the possibility of a new, poetic kind of philosophy.
	% In order to understand how these thinkers came to opposite conclusions
	% concerning language and metaethics, we will need to provide a brief overview of
	% their philosophical background. We will focus on their shared Neokantian
	% education along the lines of \autocite{friedman2000}, though I will emphasize
	% with \autocite{stone2017} their similarities to Husserl over Cassirer.
	%=========================================================================%
	% \section{Kant and Husserl}

	% Heidegger and Carnap were both educated in the Neokantian tradition. In keeping
	% with the post-Kantian idealists, the Neokantians rejected Kant's distinction
	% between the realms of pure sensibility and understanding. Over spatio-temporal
	% intuition, they preferred a pure logical basis for the object of knowledge. \autocite[p28]{friedman2000}
	% The tradition was divided between several schools, most notably the Marburg
	% school and the Southwest school. While there are many differences between these
	% camps, the primary distinction I will focus on is their stance on the
	% distinction between logic and values.

	% Logic, being a discursive science aimed at proper thinking, was considered
	% \emph{normative.} For Rickert, a key player of the Southwest school, this leads
	% to the claim that logic belongs to the realm of transcendental value. Pure
	% logic is then to be distinguished from psychological knowledge, which proceeds
	% by empirical means. This leaves us with the gap between \emph{logic} as value,
	% and \emph{psychology} as fact \autocite[p34ff]{friedman2000}.

	% Cassirer of the Marburg school rejected this distinction. Instead of cleaving
	% the mind between transcendental psychology and pure logic, Cassirer claims that
	% these gaps are \emph{moments} of cognition. They are not essential features which
	% then are synthesized in to a single cognition, but a unitary relation which
	% has been artificially analyzed into parts. Value and fact are analogous to the
	% universal and the particular, but for Cassirer this is a difference in matter
	% of \emph{viewpoint,} between understanding a thing in its \enquote{thinghood}
	% (as it relates to universal/transcendental forms), and understanding a thing in
	% its \enquote{context} (as a being among beings). This separation between viewpoints
	% is \enquote{artificial,} only occurring after the initial unitary relation
	% between mind and object. The originary perception finds these two aspects in unison
	% \autocite[p34ff]{friedman2000}.

	% The parallels between Husserl and the Neokantian schools is fairly clear.
	% Rickert writes positively of the \textit{Logical Investigation's}\nocite{husserl2005}
	% polemic against psychologism, and there are clear parallels between the
	% unitary relation of cognition in Cassirer and Husserl's unitary field of
	% phenomenological consciousness (see \textit{Ideas} \S38)\nocite{husserl2012}. Akin
	% to Rickert, Husserl ends up proposing transcendental phenomenological ideas as
	% the basis of his ontology, but akin to Cassirer he proposes that value and fact
	% are part of the same conscious process.

	% The phenomenon was proposed by Husserl as a way of bridging the gap between the
	% conscious being and the thing-in-itself. If the phenomenon is understood in
	% Kant's way as the empirical intuition of a thing, then it could be said that Husserl's
	% phenomenon is closer to the transcendental intuition itself, the conditions
	% for the possibility of experience in the first place. Reality shows itself through
	% these conditions and through phenomenal appearances. It is the work of

	% Since at least Aristotle, metaphysics has been the study of the first
	% principles that underlie the various sciences and their most generic objects. In
	% addition to the special sciences, there is logic, which studies the analogous structures
	% of each special science. And, above them all lies metaphysics, the science of
	% first principles as such. Importantly, these first principles are accessible
	% as objects of intelligible intuition. That is, although we cannot access them
	% through our senses, we have access to them through our capacity to reason.

	% Kant criticized the traditional metaphysical system for its inherent
	% contradictions. Humans only have access to external objects through our sensible
	% intuition (i.e. through our access to space and time), and to internal,
	% psychological facts through our intellectual intuition. These sensible objects
	% (phenomena) are representations of the underlying 'real' objects (noumena),
	% which are purely intellectual. Although the phenomena are merely representations
	% which occur to perceiving creatures like us, the fact of their being is
	% reliant on the structure of our cognitive faculties. If the grounds of science
	% lie in transcendental principles and causes, then what Kant wishes to say is
	% that metaphysics ought to concern itself only with the way objects appear to
	% us, and not as they are in themselves. To Kant, we do not have direct access to
	% the thing-in-itself, the noumena as such. By making this distinction, Kant
	% rescues the foundations of science while supposedly ridding it of its
	% inherently contradictory character (where possible). However, his motivation
	% for this is not just to save the sciences from falling into skepticism. His primary
	% goal is to \enquote{limit reason in order to make room for faith.} This faith
	% is the rational belief in God as the foundational basis of our capacity for
	% freedom.

	% \autocite{stone2017} recognizes three core aspects of Kant's critique of traditional
	% metaphysics: (1.) Kant acknowledges that metaphysics correctly identifies the
	% grounds of science in the transcendental causes and principles of reason. (2.)
	% He denies any knowledge of the suprasensible subject-matter of metaphysics (noumena).
	% (3.) His goal is to save practical philosophy by limiting our knowledge, to \enquote{limit reason in order to make room for faith.}
	% But Kant's critiques left open two large concerns. First, the notion of
	% noumena seems to be largely without basis. Second, the rational faculties are
	% based on spatiotemporal sensory intuition, but there is concern about the primacy
	% of these intuitions.

	% Heidegger and Carnap were both educated in the Neokantian tradition. In
	% keeping with the post-Kantian idealists, the Neokantians rejected Kant's distinction
	% between the realms of pure sensibility and understanding. Over spatio-temporal
	% intuition, they preferred a pure logical basis for the object of knowledge \autocite[p28]{friedman2000}.
	% Logic, being a discursive science aimed at proper thinking, was considered
	% \emph{normative.} For Rickert of the Southwest school, this leads to the claim
	% that logic belongs to the realm of transcendental value. Pure logic is then to
	% be distinguished from psychological knowledge, which proceeds by empirical
	% means. This leaves us with the gap between \emph{logic} as value, and \emph{psychology}
	% as fact \autocite[p34ff]{friedman2000}.

	% Cassirer of the Marburg school rejected this distinction. Instead of cleaving the
	% mind between transcendental psychology and pure logic, Cassirer claims that
	% these gaps are \emph{moments} of cognition. They are not essential features
	% which then are synthesized in to a single cognition, but a unitary relation which
	% has been artificially analyzed into parts. Value and fact are analogous to the
	% universal and the particular, but for Cassirer this is a difference in matter of
	% \emph{viewpoint,} between understanding a thing in its \enquote{thinghood} (as
	% it relates to universal/transcendental forms), and understanding a thing in
	% its \enquote{context} (as a being among beings). This separation between
	% viewpoints is \enquote{artificial,} only occurring after the initial unitary relation
	% between mind and object. The originary perception finds these two aspects in
	% unison \autocite[{\textit{ibid.}}]{friedman2000}.

	% The parallels between Husserl and the Neokantian schools is fairly clear. Rickert
	% writes positively of the \textit{Logical Investigation's}\nocite{husserl2005}
	% polemic against psychologism, and there are clear parallels between the unitary
	% relation of cognition in Cassirer and Husserl's unitary field of phenomenological
	% consciousness (see \textit{Ideas} \S38)\nocite{husserl2012}. Akin to Rickert, Husserl
	% ends up proposing transcendental phenomenological ideas as the basis of his ontology,
	% but akin to Cassirer he proposes that value and fact are part of the same
	% conscious process.

	% The phenomenon was proposed by Husserl as a way of bridging the gap between
	% the conscious being and the thing-in-itself. If the phenomenon is understood in
	% Kant's way as the empirical intuition of a thing, then it could be said that
	% Husserl's phenomenon is closer to the transcendental intuition itself, the conditions
	% for the possibility of experience. Reality shows itself through these
	% conditions and through phenomenal appearances.

	% In order to overcome the limitations of Kant's system, Husserl proposes to
	% link our perceptions and the noumena, thus grounding our faculties and overcoming
	% representationalism at once. Husserl's claim is that phenomena do not exist as
	% objects-in-themselves, independently of our own interpretation. Rather, phenomena
	% are those objects whose whole existence is dependent on our postulation through
	% an intentional (directed) interpretation of sense experience. This
	% transcendental process reveals to us the \textit{Urregion} of pure consciousness,
	% containing the necessary and absolute entities (phenomena as such) which
	% ground our human faculties. But here Husserl has solved the problems of noumena
	% and intuition by the creation of a realm of suprasensible entities! He has, in
	% effect, reinstated metaphysics. \autocite[p4f]{stone2017}

	% What about practical philosophy? For Husserl, ethics would appear as an objective
	% science the same as any other. What is objectively good is what must
	% necessarily appear good. Analogously to the postulation of sensory phenomena,
	% evaluative phenomena are found in the \textit{Urregion} of consciousness insofar
	% as they are postulated on top of volitional or emotional data. Ethics, then,
	% takes on the shape of an objective science analogous to material phenomenology
	% \autocite[p5]{stone2017}.

	% Stone argues that Carnap's \enquote{The Overcoming of Metaphysics}\nocite{carnap1966}
	% and Heidegger's \citetitle{heidegger2008b} overcome metaphysics in the same way
	% as Kant. For Heidegger, science has no limits, it will tell us everything
	% there is to know about beings. This knowledge is based in our empirical faculties,
	% as per the usual story. However, our knowledge of metaphysical entities comes
	% to us not through a faculty of reason but directly through our phenomenal
	% experiences. In this way, Heidegger, with Husserl, overcomes the Kantian
	% dilemmas of sensory intuition and noumenal perception. However, for Heidegger,
	% our understanding of the transcendent nature of being is directly accessible to
	% us through a non-cognitive faculty of perception, i.e. through the fundamental
	% mood of anxiety through which we hold ourselves out into the possibility of our
	% own death. In this nihilative force, we find a calling toward authenticity,
	% that is, to find an identity. Metaphysics, to Heidegger, is this calling
	% toward authenticity in which we find our freedom.

	% Carnap's overcoming similarly extends the purview of science. As noted above,
	% his commitment to the verification principle presses metaphysics into the
	% realm of an oblique expressionism. His notion of freedom lies in his principle
	% of tolerance, wherein we are free to choose the logistic (thus normative)
	% system by which we describe the world.

	% Heidegger and Carnap are proceeding from, but overcoming, the phenomenological
	% method of Husserl and the logical conclusions of the Neokantians. While
	% Husserl takes a radical philosophical leap from the Neokantian tradition, he remains
	% culturally indebted to his generation's critical method. Further, Friedman's contention
	% that Cassirer may hold the key to resolving the gap between our two thinkers misses
	% the cultural link between them. Cassirer's methods are still steeped in
	% Enlightenment-era philosophical aesthetics. His Kantian method is a representation
	% of his commitment to conciliatory liberal republicanism \autocite[p3f]{friedman2000}.
	% Although we can look to Cassirer for a remarkable fusion between the
	% philosophies of Heidegger and Carnap, we cannot look to him for an overcoming of
	% modernity.

	Next we will focus on our thinkers' stances on the nature of logic and value
	theory. I will show how for each author: (1.) Scientific thought is formal in
	nature and cannot capture values; (2.) Language is a key aspect of relating between
	the mind and the world; (3.) The manner in which we use language corresponds
	directly to our practical, political, and ethical aims.

	%-------------------------------------------------------------------------%
	\section{Logic, Science, and Social Democracy}

	For Carnap, the end of science is to \blockquote[{\citetitle[\S179]{carnap2003}}]{find and order the true statements about the objects of experience}.
	This can be split into two further goals: to create a constructional system
	corresponding to formal logic, and to investigate the relationships between objects
	of non-constructed experience. We then have two foundational aspects of
	construction theory: conventional stipulation of logical syntax and verification
	through empirical investigation.

	Carnap proceeds from unanalyzable phenomenological units which he calls basic experiences.
	These are whole-field conscious experiences which appear epistemically prior
	to the analysis of their constitutive parts. They are precisely the object of phenomenological
	analysis in Husserl's sense (\citetitle[\S64]{carnap2003}).\footnote{Carnap
	believed his system to be compatible with the three major epistemological
	movements: realism, idealism, and phenomenalism (\citetitle[\S177f]{carnap2003}).
	Nonetheless, he chose the Husserlian phenomena as his starting point.} However,
	basic experiences do not have any relation to a subject; the \enquote{for-me-ness}
	is analyzed into it after the fact. Similar to how we can only call integers
	\enquote{integers} in contradistinction to the real numbers once we have constructed
	them, we only understand our basic experiences as autopsychological in
	contrast with the later-constructed physical and heteropsychological objects. (\citetitle[{III.C \textit{passim.}}]{carnap2003}).

	Despite the inability to analyze the experiences themselves, he \enquote{quasi-analyzes}
	them by creating a formal allegory and examining the structures thereof. This
	amounts to a functionalization of objects, a switch from an ontological/material
	perspective to a structural/functional perspective. In this way he constructs
	the world through a purely formal analysis of basic experience. On this basis he
	claims to transform foundational epistemology from a speculative practice into
	a mathematically rigorous, logistical process.

	In the \textit{Aufbau,} Carnap aims to formulate a \enquote{one true language}
	from which to construct the world. In response to formal difficulties arising from
	the construction of the physical from the autopsychological, he is forced to
	reconsider his stance on logic, eventually realizing that there are many valid
	formal logics. Once we reach the \textit{Syntax,} this stipulatory nature is formalized
	into the principle of tolerance.
	%
	\begin{displayquote}
		[{\citetitle[\S17]{carnap2000}}] \emph{In logic, there are no morals.}
		Everyone is at liberty to build up his own logic, i.e. his own form of
		language, as he wishes. All that is required of him is that, if he wishes to
		discuss it, he must state his methods clearly, and give syntactial rules instead
		of philosophical arguments.
	\end{displayquote}
	%
	Since logistics is the ideal evaluative framework, but there are multiple valid
	logistical systems, it is up to us to find the ideal system to relate to our
	practical aims, whatever they may be.

	Analogous to the ontological and functional perception of gestalt phenomena,
	Carnap distinguishes between two modes of speech, the material and the formal.
	The distinction is defined by a translation between the formal-structural
	semantic statements of the kind discussed in the \textit{Syntax} and the material-ontological
	mode of speech which postulates properties of objects with little concern for structural
	soundness. Crucially, some material sentences, specifically those that propose
	spurious universal properties, cannot be translated into formal syntax. While
	Carnap claims that the material mode of speech is perfectly acceptable, and in
	fact unavoidable, he warns us that untranslatable material statements lead us
	to speak metaphysically. Metaphysical statements are a kind of expression, but
	they cannot be subject to claims about truth and falsity in the sense of
	logical-empirical validity, and so have no place in a scientific philosophy.
	\begin{displayquote}
		[{\citetitle[\S81]{carnap2000}}] \emph{Translatability into the formal mode
		of speech constitutes the touchstone for all philosophical sentences} \dots
		Sentences which do not give even a slight indication to determine their
		translation are outside the realm of the language of science and therefore
		incapable of discussion, no matter what depths or heights of feeling they may
		stir.
	\end{displayquote}
	So Carnap has his linguistic prescription: if you wish to be intelligible, you
	must speak in a manner consistent with the formal mode of speech.

	It is precisely this attitude which spells out the dangers of metaphysics. Metaphysical
	speech has the appearance of scientific rationality, while in fact only
	prescribing an orientation toward the world. This can lead to an unearned sense
	of authority for the individual. Carnap's argument against metaphysics - and
	especially against Heidegger - consists in this: speculative metaphysics is inherently
	authoritarian by lack of verifiably. In contrast, a social-scientific ethics undoubtedly
	places a large emphasis on the authority of the group and the sentiment of comradery
	which is essential for the scientific enterprise.

	Carnap claims in a 1932 lecture that ethical claims are in essence prescriptive
	statements, specifically imperatives. These statements are manifestations of an
	attitude toward life. Importantly, imperatives are \emph{not} translatable
	into formal language. We can see how easy it is to spin up spurious universals
	by analyzing ethical statements. \enquote{Do not kill,} for example, is
	clearly an imperative, while \enquote{killing is evil} is an assertion about a
	universal property which does not admit of empirical evidence \autocite[\S1.4]{carnap1996}.
	\enquote{Evil} is not a material thing, nor a property thereof. Carnap's own prescription,
	the scientific worldview, must then itself be an expression of an attitude toward
	life, and admits of no proper evidence other than a particular kind of faith
	in science.

	\begin{displayquote}
		[{\citetitle[xvii.f]{carnap2003}}] Whence then our confidence that our call for
		clarity, for a science that is free from metaphysics, will be heard? It
		stems from the knowledge or, to put it somewhat more carefully, from the
		belief that these opposing powers belong to the past. \dots Our work is carried
		by the faith that this attitude will win the future.
	\end{displayquote}

	The inability to formalize value statements leads Carnap directly into the famous
	argument against positivism, that the verification principle itself is a
	prescription, and cannot be verified. Any justification for a value must be
	made within the language of science, but the objects of its study are a matter
	of \emph{psychology} and not something which can be derived from universal
	principles of reason. Carnap was not ignorant of this problem. Although he
	cannot ground the scientific worldview on rational grounds, he argues that approaching
	the world scientifically is the only way to ensure the survival of the sciences.

	\autocite{dambock2022a} proposes three key aspects of Carnap's relationship
	between science and practical philosophy. (Note that these are exactly the
	points Weber makes in \enquote{Science as a Vocation.}) First, scientific attitudes
	allow us to evaluate the consistency of our beliefs. Second, they can show us the
	causal relations between events, thus giving us a good idea of the outcomes of
	our decisions. Third, science can reveal to us a relationship between our
	means and ends. This final point is crucial for us, as it is here Carnap's
	noncognitivism bridges into socialism. In order to understand this we need to understand
	what makes an ethical statement sound.

	A proper ethical statement must be logically consistent, and the feeling which
	underlies it must be temporally persistent. We need to discard temporary feelings
	in favor of long-term deliberative consensus. This is akin to a thermodynamic
	process, where stochastic motion of particles (in this case, feelings)
	eventually evens out into a long-term \enquote{temperature} (attitude).
	Rational discourse and deliberation plays a key factor in this process. In addition
	to private deliberation, we must be able to reconcile our beliefs with those
	of others. In order to do so objectively, we must remain open to emotive and
	organizational challenges. Damb\"ock gives the examples of the staunch Christian,
	who must remain committed to their religious doctrine, and the Neoliberal, who
	is closed to centralized organization \autocite[p515]{dambock2022a}. If we are
	to adopt a scientific attitude everywhere, then it is only atheistic Marxist
	material analysis which can provide a causal account of the consequences of
	our actions and thus to determine the most consistent evaluative structure and
	course of action. Any evaluative discourse based on dogma will inevitably lead
	to an authoritarian politic.

	In summary: (1.) Science is logistic in nature, proceeding from basic experiences
	to the construction of the world, while value postulation is a matter of
	irrational prescription, an animal reaction to chance situation. (2.) The
	material mode of speech is natural but insufficient, leading us into confusion.
	If one is to proceed scientifically, they must speak in a way which is translatable
	into the formal mode of speech. (3.) Material language has potentially
	deleterious psychological effects. The formal mode of speech most clearly aligns
	itself with a communitarian ethic, and specifically, the communist politic,
	while the material mode of speech aligns itself most easily with
	authoritarianism, specifically, the fascist politic.

	%-------------------------------------------------------------------------%
	\section{Beyond Logic}

	Heidegger, like Carnap, begins his analysis from phenomenological grounds. But
	his idea of a phenomenon is quite different. His system is very intricate, so
	we will only have the space to cover a small fraction of it here, mainly as it
	appears in \textit{Being and Time}.\footnote{References to Being and Time are
	to the original page numbers, prefixed with an H. in the cited edition.} We
	will focus only on what is necessary to contrast his understanding of
	scientific rationality from Carnap's through a reading of his critique of
	Descartes.

	Heidegger's project is to study what it means for anything to be at all. This is
	the question of Being. For Heidegger, the question of Being is the proper aim of
	philosophy; it will take shape as a fundamental ontology. The goal of ontology
	is to set the \textit{a priori} foundations for the study of beings in general.
	In analogy with Carnap, Heidegger's approach is to analyze both the ontic, object-oriented
	sciences and the ontological, functional structure of consciousness in general.
	In doing so he and Carnap take on Husserl's goal of uniting the phenomena and the
	grounds of our rational faculties. However, in contrast to Carnap, in order to
	understand the object Heidegger believes we must understand it \emph{as} an object,
	i.e., as it appears. We cannot functionalize it, proceed by analogy, and consider
	this to be a full explanation. We must consider phenomena in its relation to
	our Being as a \emph{whole}, picking apart its various equiprimordial roots. For
	Heidegger, this entails an essential relation of Being to the horizon of time,
	and especially to death. In considering Being we will come to a better
	understanding of the ambiguity inherent in the material mode of speech.

	We humans are exceptional in that our ontic existence is in fact ontological; that
	is, our Being is something we recognize, and which forms a problematic for us.
	This recognition makes us naturally capable of ontology. Because of this exceptional
	feature, it is through an analysis of \emph{our} Being, what Heidegger calls
	Dasein (lit. \enquote{being-there}) that we can come to an understanding of
	Being in general. Further, this analytic must occur as Dasein exists \enquote{proximally and for the most part,}
	in its everydayness. In so doing we will not only found the sciences, but all other
	meaningful orientation toward the world. (\citetitle[Int. I]{heidegger2008a})

	The method of this analysis is phenomenology. Though the term \enquote{phenomenology}
	implies a scientific approach, the form of \textit{l\'ogos} Heidegger has in
	mind is quite different from both the traditional syllogistic logic and the
	new mathematical logistics. What Heidegger means by \textit{l\'ogos} is
	attained through an etymological analysis of the original Greek terms. \textit{L\'ogos}
	is the manner in which a discussion allows the object of discourse to be seen
	(in Aristotle, \textit{apopha\'inesthai}). In this way, truth (\textit{al\'ethia},
	lit. \enquote{not to escape notice}) becomes a manner of \enquote{disclosing} or
	\enquote{discovering} (\textit{pha\'inesthai}, lit. \enquote{letting-be-seen})
	the phenomenon, with falsity being a manner of \enquote{covering up} (\textit{pse\'udesthai})
	to hide the \enquote{true} nature of the being as the object of discourse. So in
	this way, \textit{l\'ogos} as logistics cannot be the locus of truth. It is
	only in the phenomenon, or aesthetic experience, in which we find truth. \blockquote[{\citetitle[p33]{heidegger2008a}}]
	{\textit{A\'isthesis}, %
	the sheer sensory perception of something, is \enquote{true} in the Greek sense, and indeed %
	more primordially than the \textit{l\'ogos} which we have been discussing}. \textit{A\'isthesis}
	cannot cover up, as it displays itself as itself (as \textit{noe\^in}, cf
	Husserl's \textit{noema}); at worst it can be a kind of non-perceiving (\textit{agnoe\^in}).
	The object of \textit{a\'isthesis} is the \textit{\'idia} as, and for example,
	color is the object of sight. Though \textit{a\'isthesis} shows things as they
	are, covering-up occurs through synthesis, the showing of something as or through
	something else. \textit{A\'isthesis} does not occur through a faculty of cognition,
	but directly. For Heidegger, rationality in the scientific sense occurs only
	atop of apohpantic logos, which itself is derivative of aesthetic/phenomenal understanding.
	(\citetitle[Int. II.A, B]{heidegger2008a})

	So, the task of phenomenology is to (literally) discover the ideas (\textit{qua}
	principles) of aesthetic perception. In this way, \textit{contra} Carnap, \enquote{for-me-ness}
	is an essential aspect of the phenomenon. It is clear then that the \textit{manner}
	in which things appear to us is of deep importance to Heidegger, and that the manner
	in which we express ourselves will be key to disclosing Being. He contends
	that our language has developed in accordance with the ontic perception. In order
	to overcome this, he requires us to rethink the manner in which we relate to, understand,
	and discuss the world.

	% To be clear, phenomenology cannot proceed in the form of propositional assertions
	% as in Carnap. This form of logic separates the essential character of Being
	% from its onto-historical context. By analyzing it in the abstract, the essence
	% \enquote{gets understood in an empty way} and \enquote{loses its indigenous character}
	% (\citetitle[p36]{heidegger2008a}). In Carnap's system, we analyze basic
	% experiences which takes its character as free from any onto-historical or
	% personal identity. Any \enquote{for-me-ness} is analyzed \emph{into} basic experiences.
	% For Heidegger, \enquote{for-me-ness} is essential to the very character of
	% Dasein's Being. It is clear then that the \textit{manner} in which things
	% appear to us is of deep importance to Heidegger, and that the manner in which we
	% express ourselves will be key to disclosing Being.

	Things in the world can show themselves to us in different ways. Two of the
	most important orientations are readiness-to-hand and presence-to-hand. Ready-to-hand
	entities are visible in their tool-use: they make themselves apparent to us in
	their useful capacities toward some goal or another. But when the tool breaks,
	we stop to consider the entity as it is in itself, as a material thing with constitutive
	parts, etc. This viewpoint, when the entity is considered primarily \emph{as} entity,
	is presence-at-hand.

	Heidegger characterizes us as thrown into the world against our wills, made to
	orient ourselves. We do so through our state-of-mind, manifested as mood. Our
	moods turn us \emph{away} from ourselves and toward the world, thus revealing
	beings to us in their factical nature. Entities only appear to us as present-at-hand
	(in their utility) or otherwise because we are already predisposed by our moods
	to find them in this way (\citetitle[I.5 \P29]{heidegger2008a}). Alongside
	mood, we find the equally basic relation of understanding. The world discloses
	itself to us, and thus we to ourselves, in our possibilities, that is, in all
	the potential relations which they may hold. Potentiality is always future-oriented
	as something not yet actual, and never necessary. Understanding, then, as equiprimordial
	with state-of-mind, is a future-oriented disclosure of the potentiality of the
	world. This aspect of understanding is called projection. Our moods, as a turning-away,
	close off certain possibilities for us, thus restricting our sight. The manner
	in which projection works out its restricted possibilities is called
	interpretation, and the manner in which we then see something \emph{as}
	something is called articulation. But this potential-utility of the ready-to-hand
	object, for example, manifests itself through a \emph{totality} of
	involvements. Every potentiality of the ready-to-hand is such an involvement,
	and the ready-to-hand is only understood as relating to the entire mesh of
	these potential involvements.

	Since only what is understood can be articulated, all predication always already
	has the structure of understanding. As a matter of fixed predication, assertion
	is itself a kind of interpretation. But, since all understanding relates entities
	to Dasein and in so doing wraps up their being within Dasein's world, any
	\emph{meaning} as such will be discovered alongside the Being of Dasein. (\citetitle[\P31f]{heidegger2008a})

	So, Heidegger characterizes meaningful entities as just those entities which are
	understood in relation to our thrown projection, restricted in their
	possibilities (for utilization) by our state-of-mind. This interpretive structure
	is circular: we project meaningful possibility onto the ready-to-hand only
	because we already have a network of possibility structures in place by which
	to interpret them. Heidegger views this as a positive characteristic.
	\blockquote[{\citetitle[p153]{heidegger2008a}}]{What is decisive is not to get out of the circle but to come into it in the right way. This circle of understanding is not an orbit in which any random kind of knowledge may move; it is the expression of the existential \emph{fore-structure} of Dasein itself. \dots In the circle is hidden a positive possibility of the most primordial kind of knowing.}

	Logistical analysis then becomes a kind of interpretation, a tearing-apart of
	what has been seen together in synthesis. He characterizes logic as operating on
	the basis of an ontic, \enquote{apophantical \enquote{as,}} in opposition to the
	\enquote{existential-hermeneutical \enquote{as}} of the ontological stance. In
	the ontic approach, \blockquote[{\citetitle[p159]{heidegger2008a}}]{The judgement gets dissolved logistically into a system in which things are \enquote{co-ordinated} with one another; it becomes the object of a \enquote{calculus}; but it does not become a theme for ontological Interpretation}.
	That is to say, within the ontic mode of speech, we are unable to pursue an ontological
	discourse which can account for the socio-psychological aspects of our understanding.
	Apophantic logos misses the character of \enquote{thought} in Heidegger's sense,
	as it does not clear the space for an existential-hermeneutic analysis, and so
	is vastly limited in its possibilities.

	Now, finally, we come to characterize discourse or speech (\textit{Rede}) as
	equiprimordial with state-of-mind and understanding. As shown, meaning-making
	is a manner of interpretation, which takes the form of articulation. Meaning
	is that which is articulated. But what has been understood, and thus
	articulated, is always a totality of significations. The meaning of an
	utterance can be broken apart into separate significations, as in apophantic logos,
	and mapped to words. (\citetitle[p161ff]{heidegger2008a})

	In order to communicate, we must be able both to speak and to listen.
	Listening takes the form of \emph{being quiet;} through this we open ourselves
	up to be \emph{with} others. But listening, that is, hearkening to the understanding
	which is to be communicated, can only occur because one is proximally
	alongside the Other. In this being-with, we find ourselves already with an
	understanding of the thing that any intelligible discourse is about. In order to
	be understood, one must already be in a shared state-of-mind (\citetitle[p163ff]{heidegger2008a}).
	This is to say that understanding requires that one shares at least some portion
	of their \enquote{web of beliefs} with the other party, even if that understanding
	is as \enquote{minimal} as a shared natural language, or the same night sky.

	Now, with all this mind we can understand Heidegger's call to destruct the
	Cartesian logical-empirical project. To Descartes, mathematical clarity, as an
	unchanging eternal ideal, is the arbiter of truth. To the extent that something
	truly \enquote{is,} its existence can be described in terms of mathematical
	rigor. But in order to do so, Descartes bases his ontology on the \textit{res
	extensa,} i.e. on Nature. The mathematical mode of thought is apophantical,
	taking entities as they are present-to-hand, and reduced to their material-empirical
	properties, a deeply artificial state of mind. Descartes \blockquote[{\citetitle[p96]{heidegger2008a}}]{prescribes for the world its \enquote{real} Being, as it were, on the basis of an idea of Being whose source has not been unveiled and which has not been demonstrated in its own right -- an idea in which Being is equated with constant presence-at-hand.}
	This second-hand knowledge - of material substance as constantly present-at-hand
	- forces us to avoid the question of value. As Carnap says, anything which goes
	beyond what can be empirically verified is nonsensical, i.e., cannot be
	formalized. This is exactly right, and it is exactly the issue.

	As in the \enquote{Question Concerning Technology,} \nocite{heidegger2008c}
	Heidegger claims that the empirical project, through modern technology, has
	dissociated humanity from its primordial nature. In addition to being a means to
	an end and a human activity, to Heidegger technology is a kind of revealing, an
	interpretive apparatus. Modern technology reveals the world as standing-reserve,
	available to us only in its possibility for consumption, i.e. translation into
	physical energy. Throughout his later work, as in \enquote{The Task of Thinking}\nocite{heidegger2008d}
	and \enquote{Only a God can Save Us,} Heidegger uses this argument in explicit
	opposition to socialism, which he calls \enquote{a form of planetary technicity}
	\autocite[p206]{heidegger1981}. A Marxist might consider this strange: it is the
	capitalist class who exploits his fellow-man with technology in order to gain the
	excess wealth of his production. But Heidegger would respond: the worker and capitalist
	are both enframed, driven to view the world as standing-reserve irrespective
	of their class status. Socialism will not fix this problem because it does not
	understand the essence on which it is grounded.

	His alternative is found in the concept of dwelling. In \textit{Being and Time,}
	the question of Being is phrased in terms of the meaning of Being. This
	provides an ontological analysis, which essentially takes the shape of a traditional
	scientific treatise. In this treatise the final fundamental aspect of Dasein
	is revealed as temporality. The second, unfinished part of \textit{Being and
	Time} was to be a reversal of the analysis, \enquote{Time and Being,} in which
	he was to destruct the history of Being as understood in the Western philosophical
	canon. He found that the scientific-ontological mode of language was
	inappropriate for the task at hand. Scientific thought imposes our image on nature;
	it is inherently technic; our interpretation of the world is too \enquote{loud.}
	When the question of Being shifted to ask, \enquote{how does Being unfold?}, Heidegger
	likewise shifted to a poetic form of language, one which is more suited to
	\textit{listening.} Under this new questioning, Being reveals itself to us
	through the fourfold character of dwelling. This fourfold character is unified,
	and reveals itself in the ecstatic (phenomenally prominent) characters of the
	earth, the sky, mortals, and gods. Each has its own characteristic interaction;
	we save the earth, we receive the sky, await the divinities, and initiate the
	mortals. Right away it is apparent that the scientific mode of language is inappropriate
	here. It is, in fact, inappropriate for the kind of at-home-ness that
	Heidegger is attempting to articulate.

	The question for Heidegger, then, becomes the manner in which we ought to
	relate to Being as a whole. The task is to think \blockquote[{\cite[p437]{heidegger2008d}}]{of the possibility that the world civilization that is just now beginning might one day overcome its technological-scientific-industrial character as the sole criterion of man's world sojourn}.
	In order to do so, we must save the earth (by reconsidering our relation to
	spatiality), receive the sky (by dwelling temporally), await the gods (by
	finding essentially new social archetypes within a natural-technical world),
	and initiate mortals (by dwelling poetically and authentically toward death). Each
	of these essential relations to Being have parallels in \textit{Being and Time,}
	in Being-in-the-world, the historicality of Being, our fallen and projective
	character toward Others, and finally our Being-toward-death and the character of
	resolute authenticity. Unfortunately, we do not have the time to discuss any of
	this further.

	% As stated, Heidegger's proposed metaethic relies on an authentic choice of a
	% socially-bound identity. Dasein, in being-with-others, is considered fallen insofar
	% as it finds itself alongside others in the world. In its fallenness Dasein is
	% inauthentic. This is not a morally negative characteristic, but an opposition
	% to authentic Dasein. We only take a concern with our Being when experience a shift
	% in our state-of-mind, for example in anxiety (\textit{Angst}). In anxiety, we withdraw
	% from the other. We are taken out of our thrown projection into the comfortable
	% dwelling of society and forced to reckon with our identity. \enquote{Anxiety makes manifest in Dasein its \emph{Being towards} its ownmost [authentic] potentiality-for-Being -- that is, its \emph{Being-free for} the freedom of choosing itself and taking hold of itself}
	% (\citetitle[p188]{heidegger2008a}). In this authentic turning-towards-the-self
	% we find our potentiality for freedom. This is the nihilating force of the
	% Nothing on which Heidegger expounds in \enquote{What is Metaphysics?,} and it is
	% this potentiality for personal being which underscores his metaethical theory. The technological mode of Being fundamentally discloes

	So, to conclude: (1.) Science is formal in nature, relying on an orientation which
	prefers ontic analysis over ontological. Values arise from existential
	considerations, which can be explicated only in terms of a hermeneutic analysis
	of Dasein. (2.) Our being-in-the-world, thus our understanding and our state-of-mind,
	is equiprimordial with and becomes intelligible through discourse, which reifies
	itself in language. (3.) The scientific worldview and its formal language
	lends itself to an ontic orientation which puts us at risk of dehumanization. Socialism
	and global capitalism, as technocratic regimes, reduce all Being into standing-reserve,
	potential for utilization toward an unexamined end. The challenge for post-modernity
	is to understand its relation to the new, technical world, and to create archetypal
	relations in which Being may reveal itself to us anew.

	%=========================================================================%
	\section{Conclusion}

	I have argued that Heidegger and Carnap are motivated by a Weberian
	disenchantment with modern technological existence. While Carnap strives to for
	a purely scientific world-view and philosophical discipline with social-democractic
	values, essentially leaning in to the Enlightenment project of world
	technicity, Heidegger argues for a poetic sense of dwelling in which we are to
	challenge our relationship to technological thought in order to build new relations
	to Being.

	Carnap's scientific attitude proposes a non-cognitivism which relies on conciliatory
	democratic principles. He argues that Heidegger's brand of assertive material
	speech is dangerous because of its ability to propose spurious metaphysical
	properties and tendency toward authoritarianism. He favors a purely structural
	account on the grounds that logistics is the foundational mode of rationality.
	These arguments are based in Enlightenment-era thought, and are seen as a way to
	bring them to their completion. In this way Carnap tackles the problem of
	disenchantment by proposing a life-reform project based on scientific
	rationality, leaning into the primacy of rationality.

	Heidegger's poetic attitude proposes a sort of non-cognitive contemplation which
	relies on a fundamental relationship to ourselves and our environment. He asks
	us to listen to Being rather than press our will on it through technology. The
	essence of technology is found in an enframing which forces us to view the world,
	and even ourselves, as a sort of resource, ready to be utilized to an end
	which we cannot examine through its lens. His analysis is at first based on a phenomenological
	hermeneutic of our lived experience, and later turned toward an analysis of
	Being as it expresses itself through poetic language. In this way Heidegger tackles
	the problem of disenchantment by leaning into the \enquote{more primoridal truth}
	of poetic thinking.

	It has been proposed that the gap between these two thinkers may be
	irreconcilable. On the contrary, Heidegger's position can account for the
	scientific presuppositions of Carnap's position. However, Carnap has a good
	eye for the sociopsychological character of group deliberation and the
	importance of democratic consensus, which Heidegger does not discuss. If there
	is to be a resolution, it must proceed from a natural-technic ground a la Heidegger,
	and must account for Carnap's sharp eye for the social good.

	It has also been proposed that we might find a solution to our thinkers' differences
	in Cassirer. However, this proposal misses the crucial cultural link between them.
	Cassirer's methods are still steeped in Enlightenment-era philosophical
	aesthetics. His Kantian method is a representation of his commitment to conciliatory
	liberal republicanism \autocite[p3f]{friedman2000}, the very same which failed
	to account for modern technology's existential challenge. Although we can look
	to Cassirer for a remarkable fusion between the philosophies of Heidegger and Carnap,
	we cannot look to him for an overcoming of modernity.

	Heidegger is not against science, and Carnap is not against poetry. These two
	thinkers are not as at odds as one may think. If Heidegger offers a critique of
	the scientific worldview, it is not to say that he is not a scientific realist,
	only that he sees the technological mode of Being as one among many. Instead
	of proceeding from a highly developed technological world-view to reconstruct
	the world, Heidegger suggests we sit and listen, to allow the world to speak
	to us as it exists in is and alongside us.

	\printbibliography

	\centerline{\ding{166}}
\end{document}